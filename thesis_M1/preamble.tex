% Some basic packages
\usepackage[utf8]{inputenc}
\usepackage[T1]{fontenc}
\usepackage{textcomp}
\usepackage[french]{babel}
\usepackage{url}
\usepackage{graphicx}
\usepackage{float}
\usepackage{booktabs}
\usepackage{enumitem}

% \usepackage[a4paper,left=1.5cm,right=1.5cm,top=1.5cm,bottom=1.5cm]{geometry}

\usepackage{hyperref}
\hypersetup{
	colorlinks=true,
	linkcolor=black,
	filecolor=magenta,
	urlcolor=cyan,
	% pdftitle={pdftitle}, title of the window
	bookmarks=true,
	pdfpagemode=FullScreen,
}

\pdfminorversion=7

% Don't indent paragraphs, leave some space between them
\usepackage{parskip}

% Hide page number when page is empty
\usepackage{emptypage}
\usepackage{subcaption}
\usepackage{multicol}
\usepackage{xcolor}

% Other font I sometimes use.
% \usepackage{cmbright}

% Math stuff
\usepackage{amsmath, amsfonts, mathtools, amsthm, amssymb, wasysym}
% Fancy script capitals
\usepackage{mathrsfs}
\usepackage{cancel}
% Bold math
\usepackage{bm}
% Some shortcuts
\newcommand\N{\ensuremath{\mathbb{N}}}
\newcommand\R{\ensuremath{\mathbb{R}}}
\newcommand\Z{\ensuremath{\mathbb{Z}}}
\renewcommand\O{\ensuremath{\emptyset}}
\newcommand\Q{\ensuremath{\mathbb{Q}}}
\newcommand\C{\ensuremath{\mathbb{C}}}
\newcommand\K{\ensuremath{\mathbb{K}}}

% Some new math command french made
\newcommand\congru{\ensuremath{\equiv}}
\newcommand\et{\ensuremath{\wedge}}
\newcommand\ou{\ensuremath{\vee}}
% Some new command french made
\newcommand\guillemeg{\guillemotleft}
\newcommand\guillemed{\guillemotright}

% New math operator french made
\DeclareMathOperator{\pgcd}{pgcd}
\DeclareMathOperator{\ppcm}{ppcm}
\DeclareMathOperator{\card}{Card}
\DeclareMathOperator{\trace}{trace}
\DeclareMathOperator{\id}{Id}
\DeclareMathOperator{\trans}{Trans}
\DeclareMathOperator{\bij}{Bij}
\DeclareMathOperator{\vect}{Vect}
\DeclareMathOperator{\aff}{Aff}
\DeclareMathOperator{\affeuclid}{AffEuclid}
\DeclareMathOperator{\simp}{Sim^+}

% Standard math operator
\DeclareMathOperator{\dx}{dx}
\DeclareMathOperator{\dt}{dt}
\DeclareMathOperator{\ds}{ds}

% Easily typeset systems of equations (French package)
\usepackage{systeme}

% Put x \to \infty below \lim
\let\svlim\lim\def\lim{\svlim\limits}

%Make implies and impliedby shorter
\let\implies\Rightarrow
\let\impliedby\Leftarrow
\let\iff\Leftrightarrow
\let\epsilon\varepsilon
\let\phi\varphi
\let\vec\overrightarrow

% Add \contra symbol to denote contradiction
\usepackage{stmaryrd} % for \lightning
\newcommand\contra{\scalebox{1.5}{$\lightning$}}

% Command for short corrections
% Usage: 1+1=\correct{3}{2}

\definecolor{correct}{HTML}{009900}
\newcommand\correct[2]{\ensuremath{\:}{\color{red}{#1}}\ensuremath{\to }{\color{correct}{#2}}\ensuremath{\:}}
\newcommand\green[1]{{\color{correct}{#1}}}

% horizontal rule
\newcommand\hr{
    \noindent\rule[0.5ex]{\linewidth}{0.5pt}
}

% hide parts
\newcommand\hide[1]{}

% si unitx
\usepackage{siunitx}
\sisetup{locale = FR}

% Environments
\makeatother
% For box around Definition, Theorem, \ldots
\usepackage{mdframed}
\mdfsetup{skipabove=1em,skipbelow=0em}
\theoremstyle{definition}
% \newmdtheoremenv[nobreak=true]{definition}{Définition}
\newtheorem{th}{Théorème}[chapter]
\newtheorem{def}[theoreme]{Définition}
\newtheorem{propo}[theoreme]{Proposition}
\newtheorem{propri}[theoreme]{Propriété}
\newtheorem{lm}[theoreme]{Lemme}
\newtheorem{coro}[theoreme]{Corollaire}
\newtheorem{defthm}[theoreme]{Définition et théorème}

\newtheorem{exo}{Exercice}

% without numbering
\newtheorem*{demonstration}{Démonstration}
\newtheorem*{preuve}{Preuve}
\newtheorem*{but}{But}
\newtheorem*{consequence}{Conséquence}
\newtheorem*{rappel}{Rappel}
\newtheorem*{regle}{Règle de calcul}
\newtheorem*{postulat}{Postulat}
\newtheorem*{conclusion}{Conclusion}
\newtheorem*{bonus}{Bonus}
\newtheorem*{conjecture}{Conjecture}
\newtheorem*{recurrence}{Récurrence}
\newtheorem*{intermede}{Intermède}
\newtheorem*{observation}{Observation}
\newtheorem*{application}{Application}
\newtheorem*{probleme}{Problème}
\newtheorem*{terminologie}{Terminologie}
\newtheorem*{question}{Question}
\newtheorem*{exemple}{Exemple}
\newtheorem*{contrex}{Contre-Exemple}
\newtheorem*{notation}{Notation}
\newtheorem*{vuavant}{Comme vu précédemment}
\newtheorem*{remarque}{Remarque}
\newtheorem*{intuition}{Intuition}
\newtheorem*{motivation}{Motivation}
\newtheorem*{resume}{Résumé}

% End example and intermezzo environments with a small diamond (just like proof
% environments end with a small square)
\usepackage{etoolbox}
\AtEndEnvironment{exemple}{\null\hfill$\diamond$}%
\AtEndEnvironment{intermede}{\null\hfill$\diamond$}%
\AtEndEnvironment{demonstration}{\null\hfill$\square$} 
\AtEndEnvironment{preuve}{\null\hfill$\square$} 

% Fix some spacing
% http://tex.stackexchange.com/questions/22119/how-can-i-change-the-spacing-before-theorems-with-amsthm
\makeatletter
\def\thm@space@setup{%
  \thm@preskip=\parskip \thm@postskip=0pt
}


% Exercise 
% Usage:
% \exercice{5}
% \subexercice{1}
% \subexercice{2}
% \subexercice{3}
% gives
% Exercice 5
%   Exercice 5.1
%   Exercice 5.2
%   Exercice 5.3
\newcommand{\exercice}[1]{%
    \def\@exercice{#1}%
    \section{Exercice #1}
}

\newcommand{\subex}[1]{%
    \subsection{Exercice #1}
}

% \lecture starts a new lecture (les in dutch)
%
% Usage:
% \lecture{1}{di 12 feb 2019 16:00}{Inleiding}
%
% This adds a section heading with the number / title of the lecture and a
% margin paragraph with the date.

% I use \dateparts here to hide the year (2019). This way, I can easily parse
% the date of each lecture unambiguously while still having a human-friendly
% short format printed to the pdf.

\usepackage{xifthen}
\def\testdateparts#1{\dateparts#1\relax}
\def\dateparts#1 #2 #3 #4 #5\relax{
    \marginpar{\small\textsf{\mbox{#1 #2 #3 #5}}}
}

\def\@lecture{}%
\newcommand{\lecture}[3]{
    \ifthenelse{\isempty{#3}}{%
        \def\@lecture{Lecture #1}%
    }{%
        \def\@lecture{Lecture #1: #3}%
    }%
    \subsection*{\@lecture}
    \marginpar{\small\textsf{\mbox{#2}}}
}

\def\@td{}%
\newcommand{\td}[3]{
    \ifthenelse{\isempty{#3}}{%
        \def\@td{Td #1}%
    }{%
        \def\@td{Td #1: #3}%
    }%
    \subsection*{\@td}
    \marginpar{\small\textsf{\mbox{#2}}}
}


% These are the fancy headers
\usepackage{fancyhdr}
\pagestyle{fancy}

\fancyhead[LE,RO]{Yann-Arby BEBBA}

\fancyhead[RO,LE]{\@lecture} % Right odd,  Left even
\fancyhead[RO,LE]{\@td} % Right odd,  Left even
\fancyhead[RE,LO]{}          % Right even, Left odd

\fancyfoot[RO,LE]{\thepage}  % Right odd,  Left even
\fancyfoot[RE,LO]{}          % Right even, Left odd
\fancyfoot[C]{\leftmark}     % Center

\makeatother




% Todonotes and inline notes in fancy boxes
\usepackage{todonotes}
\usepackage{tcolorbox}

% Make boxes breakable
\tcbuselibrary{breakable}

% Verbetering is correction in Dutch
% Usage: 
% \begin{verbetering}
%     Lorem ipsum dolor sit amet, consetetur sadipscing elitr, sed diam nonumy eirmod
%     tempor invidunt ut labore et dolore magna aliquyam erat, sed diam voluptua. At
%     vero eos et accusam et justo duo dolores et ea rebum. Stet clita kasd gubergren,
%     no sea takimata sanctus est Lorem ipsum dolor sit amet.
% \end{verbetering}
\newenvironment{correction}{\begin{tcolorbox}[
    arc=0mm,
    colback=white,
    colframe=green!60!black,
    title=Correction,
    fonttitle=\sffamily,
    breakable
]}{\end{tcolorbox}}

% Noot is note in Dutch. Same as 'verbetering' but color of box is different
\newenvironment{note}[1]{\begin{tcolorbox}[
    arc=0mm,
    colback=white,
    colframe=red!60!black,
    title=#1,
    fonttitle=\sffamily,
    breakable
]}{\end{tcolorbox}}




% Figure support as explained in my blog post.
\usepackage{import}
\usepackage{xifthen}
\usepackage{pdfpages}
\usepackage{transparent}
\newcommand{\incfig}[1]{%
    \def\svgwidth{\columnwidth}
    \import{./figures/}{#1.pdf_tex}
}

% Fix some stuff
% %http://tex.stackexchange.com/questions/76273/multiple-pdfs-with-page-group-included-in-a-single-page-warning
\pdfsuppresswarningpagegroup=1


% My name
\author{Yann-Arby BEBBA}
