\chapter{Définitions générales sur les courbes elliptiques}

\section{Définition}

La définition générale d'une courbe est la suivante
\begin{definition}
    Soit un corps $K$. Une courbe elliptique sur $K$ est une cubique, non singulière,
    définie comme l'ensemble des solutions du plan projectif $\mathbb{P}_{2}(K)$ de
    l'équation de Weierstrass homogène suivante:
\[
\label{eq:geneEll}
E: Y^2Z+a_1XYZ+a_3YZ^2 = X^3 +a_2X^2Z+a_4XZ^2+a_6Z^3
\] 
avec $a_1,a_2,a_3,a_4$ et $a_6$ dans $K$.
\end{definition}
Le terme non singulière, signifie que la courbe est lisse. Ce qui signifie que si on écrit
l'équation précédente sous la forme d'une équation homogène $F(X,Y,Z)=0$, alors les dérivées
partielles de $F$ ne doivent pas s'annuler simultanément en un point de la courbe.

Autrement dit, il n'existe pas de point $P = \left[ x_0,y_0,z_0 \right] \in \mathbb{P}^2$ tel
que, en posant
\[
F(x,y,z) = y^2z+a_1xyz+a_3yz^2 - x^3-a_2x^2z-a_4xz^2-a_6Z^3
,\] 
on ait
\[
F(x,y,z)=\frac{\partial{F}}{\partial{x}} = \frac{\partial{F}}{\partial{y}} =
\frac{\partial{F}}{\partial{z}} = 0
.\] 
En d'autre termes, on peut définir une unique tangente à la courbe au point $P$.

On peut par changement linéaire de variable lorsque $\car(K) \neq 2,3$ se ramener à la forme
simplifié de l'équation de Weierstrass que l'on utilise dans la définition suivante.

C'est pourquoi, dans la totalité de ce qui suit la lettre $K$ désignera un corps de caractéristique $0$ ou un
corps fini de caractéristique distincte de $2$ et $3$. Autrement dit, on peut voir $K$
comme étant l'un des corps commutatif suivant $\Q,\ \R,\ \C$ ou $\mathbb{F}_{q}$.

On désignera la clôture algébrique de $K$, choisi implicitement, par la notation $\overline{K}$.

\begin{definition}
    \label{def:ell}
    Une courbe elliptique définie sur $K$ est une courbe projective plane d'équation
    \begin{align}
        \label{eq:ell}
    y^2z=x^3+axz^3+bz^2
    .\end{align}
    où $a$ et $b$ sont des éléments de $K$ vérifiant la condition
    \begin{align}
        \label{eq:delta}
    4a^3+27b^2\neq 0
    .\end{align}
\end{definition}

Comme par définition, c'est une courbe sur le plan projectif, on dispose du polynôme homogène
de degré $3$, $F(X,Y,Z)$ dans l'anneau de polynôme $K[X,Y,Z]$. Étant données $ a,b \in K$. 

Posons 
\[
F=Y^2Z-\left( X^3+aXZ^2+bZ^3 \right) 
.\] 

Comme $F$ est homogène de degré $3$. Par la définition \eqref{def:courbe}, on a donc
$E$ la courbe sur le plan projectif $\mathbb{P}_{2}$ qui est l'ensemble des solutions de l'équation
polynômiale
\[
E: \quad F(X,Y,Z) = 0
,\] 
Si $\left( x,y,z \right) $ est un éléments non nul de $\overline{K}^3$, cette condition
ne dépend que de sa classe dans $\mathbb{P}_{2}(\overline{K})$.

Soit $P = \left[ x,y,z \right] $ un point de $\mathbb{P}^2(\overline{K})$. On
dit que $P$ est un zéro de $F$ dans $\overline{K}$, ou plus simplement un zéro de $F$, si
l'on a $F(x,y,z)=0$. On signifie par, courbe projective plane d'équation \eqref{eq:ell},
l'ensemble des zéros de $F$ dans $\overline{K}$.

Quant à la condition \eqref{eq:delta}, elle signifie que les racines dans $\overline{K}$ du
polynôme
\[
f = X^3 + aX + b
\] 
sont simples.

Le lemme suivant nous garanti que la courbe est lisse.
\begin{lemme}
    \label{lem:lemme1}
    Le discriminant de $f$ est $\Delta = -(4a^3 + 27b^2)$. En particulier, les racines de $f$ sont simples, si et seulement si $\Delta \neq 0$.
\end{lemme}

Pour demontrer ce lemme, on utilise la proposition suivante:
\begin{proposition}
    \label{prop:discriminant}
    Soit $g$ un polynôme unitaire à coefficients dans $K$ de degré $n \ge 1$. Soient
    $\alpha_1,\ldots,\alpha_{n}$ ses racines dans $\overline{K}$ comptées avec
    multiplicités. Le discriminant $\Delta$ de $g$ est défini par l'égalité
    \[
    \Delta = \prod_{i<j}^{} \left( \alpha_{i} - \alpha_{j} \right) ^2 
    .\] 
    C'est un élément de $K$.
\end{proposition}

\begin{demonstration}
    Montrons tout d'abord que le discriminant de $f$ est $\Delta= -(4a^3 + 27b^2)$.

    Soit $\Delta$ le discriminant de $f$. Soient $\alpha, \beta, \gamma $ les racines de $f$ dans $\overline{K}$ et $f'$ le polynôme dérivé de $f$.

    À l'aide de la proposition \ref{prop:discriminant},on veut montrer que le discriminant est de la forme
    suivante: 
    \begin{align*}
        \Delta &= (-1) ( \alpha - \beta )^2 ( \alpha - \gamma )^2 ( \beta - \gamma )^2 \\
          &= - f(\alpha)'f(\beta )'f(\gamma)'
    .\end{align*}

    Vérifions que c'est bien le cas.

    D'aprés le théorème d'Alembert-Gauss comme $f \in \overline{K}$, on dispose de la
    forme scindé de $f$.
    \[
        f = \left( X - \alpha \right) \left( X - \beta \right) \left( X - \gamma \right) 
    .\] 
    En dérivant $f$ sous cette forme on obtient :
    \begin{align*}
        f &= ( X - \alpha ) \left( ( X - \beta ) ( X - \gamma ) \right)'  + ( X - \beta ) ( X - \gamma )\\
          &= ( X - \alpha ) \left( ( X - \beta ) + ( X - \gamma ) \right) + ( X - \beta ) ( X - \gamma ) 
    .\end{align*}

    Donc  
\[
f' = ( X - \alpha ) ( X - \beta ) + ( X - \alpha ) ( X - \gamma ) + ( X - \beta ) ( X - \gamma )
.\] 

On a alors successivement : 
\[
    f(\alpha)' = ( \alpha - \beta) ( \alpha - \gamma )
,\] 
\[
f(\beta )' = ( \beta - \alpha) ( \beta - \gamma)
,\] 
et
\[
f(\gamma)' = ( \gamma - \alpha) ( \gamma - \beta)
.\] 

En multipliant ces trois expressions, on obtient :
\begin{align*}
    f(\alpha)' f(\beta )' f(\gamma)' &= ( \alpha - \beta ) ( \alpha - \gamma ) ( \beta - \alpha ) ( \beta - \gamma) ( \gamma - \alpha ) ( \gamma - \beta ) \\
&= \left( X - \alpha \right) \left( \alpha - \gamma \right) \left( -1 \right) \left( \alpha - \beta  \right) \left( \beta - \gamma \right) \left( -1 \right) \left( \alpha - \gamma \right) \left( -1 \right) \left( \beta - \gamma \right) \\
&= \left( -1 \right) ^3 \left( \alpha - \beta  \right) ^2 \left( \alpha - \gamma  \right) ^2 \left( \beta - \gamma \right) ^2\\
 &= - \Delta
.\end{align*}

Et donc 

\[
\Delta = - f'(\alpha) f'(\beta ) f'(\gamma)
.\] 

En partant de la forme $f : x^3 + ax + b$, on remarque que $f' : 3x^2 + a$. Par suite on obtient,
\begin{align*}
    \Delta &= - f'(\alpha) f'(\beta ) f'(\gamma) \\
      &= - \left( 3 \alpha^2 + a \right) \left( 3 \beta^2 + a \right) \left( 3 \gamma^2 + a \right) 
.\end{align*}
Ce qui donne :
\begin{align*}
    \Delta  &= - \left( ( 9 \alpha^2 \beta^2 + 3a ( \alpha^2 + \beta^2 ) + a^2 ) ( 3 \gamma^2 + a ) \right)  \\
       &= - \left( 27 \left( \alpha \beta \gamma \right)^2  + 9a \left( \alpha^2 \beta^2 + \alpha^2 \gamma^2 + \beta^2 \gamma^2 \right) + 3a^2 \left( \alpha^2 + \beta^2 + \gamma^2 \right) + a^3 \right) 
.\end{align*}
On peut écrire
\[
\alpha^2 + \beta^2 + \gamma^2 = \left( \alpha + \beta + \gamma \right)^2 - 2 \left( \alpha \beta + \alpha \gamma + \beta \gamma \right)
,\] 
\[
\alpha^2 \beta^2 + \alpha^2 \gamma^2 + \beta^2 \gamma^2 = \left( \alpha \beta + \alpha \gamma + \beta \gamma \right)^2 - 2\alpha \beta \gamma \left( \alpha + \beta + \gamma \right)
.\] 
Donc d'après les relations entre coefficients et racine (i.e relation de Viète), pour un polynôme de la forme $ax^3 + bx^2 + cx + d$, on a :
\[
\alpha + \beta + \gamma = - \frac{b}{a}
,\] 
\[
\alpha \beta + \alpha \gamma + \beta \gamma = \frac{c}{a}
,\] 
\[
\alpha \beta \gamma = - \frac{d}{a}
.\] 
Ici dans $f$ on a $a = 1$, $b = 0$, $c = a$ et $d = b$.

D'où,
\[
\alpha + \beta + \gamma = 0 \text{, } \alpha \beta + \alpha \gamma + \beta \gamma = a \text{ et } \alpha \beta \gamma = - b
.\] 
Ce qui donne : 
\begin{align*}
    \alpha^2 + \beta^2 + \gamma^2 &= 0^2 - 2a = -2a \\
    \alpha^2 \beta^2 + \alpha^2 \gamma^2 + \beta^2 \gamma^2 &= a^2 + 2b \times 0
.\end{align*}

Donc le discriminant vaut :
\begin{align*}
    \Delta &= - \left( 27b^2 + 9a^3 - 6a^3 + a^3  \right) \\
        &= - \left( 4a^3 + 27b^2  \right)
.\end{align*}

Montrons maitenant que les racines de $f$ sont simple, si et seulement si, $\Delta \neq 0$

Raisonnons par contraposition et montrons que les racine de $f$ sont multiples, si et
seulement si, $\Delta = 0$. 

Supposons que $\Delta = 0$. On a alors :
 \begin{align*}
     - \left( 4a^3 + 27^2 \right) = 0 &\iff - f(\alpha)' f(\beta )' f(\gamma)' = 0 \\
                                      & \iff \left( f(\alpha)' = 0 \right) \ou \left( f(\beta )' = 0 \right) \ou \left( f(\gamma)' = 0 \right) \\
                                      &\iff \alpha \text{ ou } \beta \text{ ou } \gamma \text{ est une racine multiple}
.\end{align*}

D'où le résultat.
\end{demonstration}

    On dit que la courbe elliptique d'équation \ref{eq:ell} est définie sur $K$ pour préciser
    que $a$ et $b$ sont dans $K$. Ceci pour $a$ et $b$ vérifiant la condition
    \eqref{eq:delta}  

\begin{remarque}
\end{remarque}

\section{Partie affine et point à l'infini}

\textbf{parle de $\mathbb{P}^2(\R)$}

Posons 
\[
U = \left\{ [x,y,z] \in \mathbb{P}^2(\overline{K}) \mid z \neq 0 \right\} 
.\] 

On dispose de l'application $\phi : U \to \overline{K}^2$ définie par
\[
\phi([x,y,z])=\left( \frac{x}{z},\frac{y}{z} \right) 
.\] 

C'est une bijection, dont l'application réciproque est donnée par la formule
\[
\phi^{-1}(x,y)=\left[ x,y,1 \right] 
.\] 

Considérons des éléments $a$ et $b$ de $K$ tels que $4a^3+27b^2 \neq 0$. Soit $E$ la courbe elliptique définie sur $K$ d'équation 
\[
y^2z=x^3+axz^2+bz^3
.\] 

L'ensemble des points $[x,y,z] \in E$ tels que $z=0$ est réduit au singleton $\left\{ O \right\} $ où
\[
O = [0,1,0]
.\] 

En effet, dans l'équation \eqref{eq:ell}, il vient
\[
y^2\times 0 = x^3 + ax \times 0^2 + b \times 0^3
,\] 
donc $x=0$, ainsi on peut prendre pour représentant de classe de cet élément la classe de $O$.

Par ailleurs, $E \cap U$ s'identifie via $\phi$ à l'ensemble des éléments $(x,y)$ de $\overline{K}^2$ vérifiant l'égalité

\begin{align}
    \label{eq:ell1}
y^2 = x^3 + ax + b
.\end{align}

On dira que $E \cap U$ est la partie affine de $E$ et que $O$ est le point à la l'infini de $E$.

Dans toute la suite, on identifira $E \cap U$ et le sous-ensemble de $\overline{K}^2$ formé des éléments $(x,y)$ vérifiant \eqref{eq:ell1}. Avec cette identification, on a 
\[
E = \left\{ (x,y) \in \overline{K} \times \overline{K} \mid y^2=x^3+ax+b \right\} \cup \left\{ O \right\} 
.\] 
\textbf{les objets qui compose $\mathbb{P}^2(E)$}

Ainsi, $E$ est la courbe affine d'équation \eqref{eq:ell1} à laquelle on adjoint le point à l'infini $O$. C'est pourquoi on définira souvent une courbe elliptique par sa partie affine, sans préciser le point $O$.

\begin{remarque}
    On retiendra qu'une courbe affine d'équation de la forme \eqref{eq:ell1} est une courbe
    elliptique si et seulement si, par définition, la condition \eqref{eq:delta} est satisfaite.
\end{remarque}

\section{Points rationnels d'une courbe elliptique}
\textbf{pas encore bien clair}

Soit $L$ une extension de $K$ dans $\overline{K}$.

\begin{definition}
    Soit $P=\left[ x,y,z \right] $ un point de $\mathbb{P}^2$. On dit que $P$ est rationnel sur $L$ s'il existe $\lambda \in \overline{K}^{*}$ tel que $\lambda x$, $\lambda y$ et $\lambda z$ soient dans $L$. On note $\mathbb{P}^2(L)$ l'ensemble des points de $\mathbb{P}^2$ rationnels sur $L$.

\end{definition}

D'aprés la définition, un point non nul $P$ est dans $\mathbb{P}_{2}(L)$, si sa classe est dans $L$. Autrement
dit, 
\[
\mathbb{P}_{2}(L)=\left\{ P \in \overline{K}^{3} \mid \exists \lambda \in \overline{K}^{*}, P =
\lambda P\right\} 
.\] 

Cela justifie la notation $\mathbb{P}^2 = \mathbb{P}^2(\overline{K})$.

\begin{remarque}
    Étant donné un point $[x_1,x_2,x_3] \in \mathbb{P}^2$, le fait qu'il soit rationnel sur $L$ n'implique pas que les $x_{i}$ soient dans $L$. Cela signifie qu'il existe $i$ tel que $x_{i}$ soit non nul, et que chaque $\frac{x_{j}}{x_{i}}$ appartienne à $L$.

    En effet, soit un point $P \in \mathbb{P}_{2}$ non nul. Si $P \in \mathbb{P}_{2}(L)$,
    comme il est non nul, il existe $x \neq 0$, et pour $\lambda = x$, on a $P =
    [1,\frac{y}{x},\frac{z}{x}]$ et on a bien $\frac{y}{x},\frac{z}{x} \in L$ et pourtant ce
    sont des variables indéterminées de $\overline{K}$.
\end{remarque}

Soit $E$ une courbe elliptique définie sur $K$ d'équation \eqref{eq:ell}.

\begin{definition}
    Un point de $E$ est dit rationnel sur $L$ s'il appartient à $E \cap \mathbb{P}^2(L)$. On note $E(L)$ l'ensemble des points de $E$ rationnels sur $L$.
\end{definition}



Par définition, on a donc
\[
E = E(\overline{K})
.\] 

Le point $O = [0,1,0]$ appartient à $E(K)$. Soit $(x,y) \in \overline{K}^2$ un point de la partie affine de $E$. Par définition, il est rationnel sur $L$ si et seulement si $x$ et $y$ sont dans $L$. Il en résulte que l'on a 
\[
E = \left\{ (x,y) \in L \times L \mid y^2 = x^3+ax+b \right\} \cup \left\{ O \right\} 
.\] 

\begin{exemple}
    Soit la courbe $E$ définie sur $\mathbb{F}_{5}$ d'équation
    \[
    y^2 =x^3+x+1
    .\] 

    Cette courbe vérifie bien la condition \eqref{eq:delta}.
    
    En effet, on a $\Delta = -(4 \times 1^3 + 27 \times 1) = -31$.

    L'ensemble des points de la courbe est le suivant:
    \[
    E(\mathbb{F}_{5})= \left\{ (0,1),(0,4),(2,1),(2,4),(3,1),(3,4),(4,2),(4,3) \right\} \cup
    \left\{ \mathcal{O} \right\} 
    .\] 
    En effet, comme la courbe est symétrique, il nous suffit de vérifier que pour tous les
    valeurs de $x$ dans $\mathbb{F}_{5}$, lesquelles sont un carré dans $\mathbb{F}_{5}$.

    Par exemple, pour $x=0$, on a $y^2=0^3+0+1$ donc $y = \pm 1$, ce qui nous donne
    les points d'abiscisse $x=0$ et d'ordonnées $y=\pm 1$ dans $\mathbb{F}_{5}$, par
    conséquent les points $(0,1)$ et $(0,4)$ vérifient l'équation $f(x,y) = 0$ et sont
    donc des points de la courbe.

    Maintenant si $x=1$, l'équation de la courbe nous donne $y^2=1^3+1+1=3$ donc il faut
    chercher si dans $\mathbb{F}_{5}$, s'il existe un carré modulo $5$ égal à 3. Ce qui n'est
    pas le cas. En effet, dans $\mathbb{F}_{5}$ les éléments sont $\left\{
    \overline{0},\overline{1},\overline{2},\overline{3},\overline{4} \right\} $ et on a
    succecssivement dans $\mathbb{F}_{5}$
    \begin{align*}
        y^2 &=0^2=0 \\
        y^2 &= 1^2 = \pm 1 \\
        y^2 &= 2^2=4=-1 \\
        y^2 &= 3^2=9=-1 \\
        y^2 &= 4^2=16=1 \\
    .\end{align*}
    Ainsi, il n'existe pas $y \in \mathbb{F}_{5}$ qui vérifient l'équation de la
    courbe. Donc la courbe $E$ ne possède pas de point de d'abscisse $x=1$.

    À voir si je mets $\mathbb{F}_{25}$ je crois j'ai pigé faut que je vérifie ça ce midi
\end{exemple}
