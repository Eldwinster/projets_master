\section{sources}

Cours Delaunay: 
\begin{itemize}
    \item def 1: lisse
    \item prop 1: raison pour lem 7.1
    \item equation réduite raison de pourquoi $k$ diff de $3$ et $2$
    \item exemple des eq pour  $\car(k)=2$ ou $3$
    \item def 3: ensembles des points $k$-rationnels
    \item def 4: log
    \item th 5,6 et def 5: point rationnels et courbe supersingulière
    \item rq: pour parler de courbe singulière il faut introduire le morphisme de frobenius
    \item recherche des points rationnels différent algo
    \item protocol signature :
        \begin{itemize}
            \item algo choix de courbe

                cherche c'est quoi $l$, $t$ et $MOV$
            \item algo ecdsa (signature et vérification)
        \end{itemize}
    \item factorisation exemple d'algo info etc peut-être en parler brièvement tout en invitant
        à aller consulter pour cela il est peut-être bon d'en faire un résumer trés bref des
        différents lien.

        Je compte parler de l'algo de lenka et peut-être ce qu'il y a dans ce cours
\end{itemize}

Crypto asy et ce (chabanol)
\begin{itemize}
    \item el gammal basé sur $\left( \Z / p\Z \right) ^{*}$ muni de la multiplication
        constitue un groupe abélien
    \item enigma mm machine assortiede la clé (reglage de la machine) permet de dé/chiffrer
    \item désavantage de sym partage de la clé
    \item avantage de asy partage de la clé
    \item exemple concret paiement en ligne
    \item problème log discret version vulgarisé (oral)
    \item exponentiation rapide exemple rapport avec l'addition de $E$
    \item principe asym fonction à sens unique
    \item gros moyen mis en oeuvre pour dvp des algo de facto
    \item on ne sait pas prouver qu'une fonction est difficile à inverser
    \item plus sur que RSA
    \item histoire ec
    \item symétrie par rapport à $x$ 
    \item solution de $f$ sont les abscisses des points d'intersection de cette courbe
        avec l'axe $x$
    \item $\delta = 0$
    \item  $\delta > 0$ 
    \item $\delta < 0$
    \item example associativité
    \item remarque rationnels à partir d'un point $P$ au coordonnées rationnel on obtient
        que des multiple à coordonnées rationnel
    \item question ouverte cardinal
    \item désavantage face à RSA (donne les raisons voir delaunay)
    \item étude de l'allure de ces courbes
\end{itemize}

bib
\begin{itemize}
    \item courbe ell
        \begin{itemize}
            \item coordonnées selon équation de droites vérifie que c'est bien la même chose que ce
        que tu as et si c'est pas le cas pourquoi
            \item structure et ordre difficile à connaitre
            \item analogue plus compliqué que $\left( \Z / p\Z \right)^*$
        \end{itemize}
    \item chiffré à l'aide des c-e
        \begin{itemize}
            \item point de vue de eve
            \item difficulté de transmission de texte voir livre
            \item avantage inconvénients
                \begin{itemize}
                    \item log plus compliqué
                    \item taille des clé réduite
                    \item carte à puce peu de puissance, influence de la clé sur
                        performances
                    \item récent 
                    \item bcp de brevet donc potentiellement plus couteux
                \end{itemize}
        \end{itemize}
    \item factorisation c-e
        \begin{itemize}
            \item méthode $p-1$ de pollard
            \item facto via courbe elliptiques (Lenstra)
                \begin{itemize}
                    \item algo
                    \item pourquoi ca marche
                \end{itemize}
        \end{itemize}
    \item facto entier
        \begin{itemize}
            \item crucial par rapport à RSA
            \item différent algo possible
            \item principe derrière la facto
        \end{itemize}
\end{itemize}

Crypto et ordi quantique (Kachigar)
\begin{itemize}
    \item intro
        \begin{itemize}
            \item intro crypto
            \item antiquité romaine
            \item crypto moderne = crypto à clé publique
            \item hyp fonction à sens unique
            \item algo shor facto
            \item algo simon 1990
            \item casser les cryptosys
            \item discussion comment resister aux ordi quantique
            \item exemple sur impact sur la crypto à clé secrète
        \end{itemize}
    \item crypto état des lieux
        \begin{itemize}
            \item césar
            \item la vrai diff entre sym et asym c'est la difficulté de trouver $f^{-1}$
            \item def crypto systeme
            \item fonction à sens unique pas de preuve existence
            \item exemple détaillé RSA
            \item nombre de clé nécessaires
            \item rapidité sym $>$ asym
            \item exemple pratique d'hybride en footnote
        \end{itemize}
    \item asym vs quantique
        \begin{itemize}
            \item type d'attaque
            \item algo shor attaque de type math
            \item tps de calcul rsa rapport entre sec et facto
            \item explication compléxité rsa
            \item explication entre temps de calcul pour dé/chiffrer
            \item problème de recherche de période
            \item simon 1994 inspire shor ordi quantique efficaces problème de recherche
                de période
            \item shor permet de ramener le probleme de facto à un probleme de période grace a
                des propritété d'arithmétique
            \item exploite le fait que le problème soit doté d'une structure
            \item au contraire d'une simple recherche d'éléments
            \item raison calcul quantique capable de tel calcul
            \item resumé sur $P=NP$ liens avec quantique et crypto
        \end{itemize}
    \item sym vs quantique
        \begin{itemize}
            \item résultat de claude shanon longueur du texte et de la clé égale pour une
                sécurité parfaite
            \item parle un peu du principe moderne de sym
            \item présente le shéma d'enven-mansour (preuve de concept) de comment ramener
                les problèmes lier à la crypto sym à des problème de période
        \end{itemize}
\end{itemize}

wiki
\begin{itemize}
    \item el gamal
        \begin{itemize}
            \item GNUpg
            \item info
        \end{itemize}
    \item crypto hybride
        \begin{itemize}
            \item GnuPG
            \item PGP
            \item TLS
        \end{itemize}
    \item log
        \begin{itemize}
            \item pas d'algo pour reciproque mais addition nb de multi log en taille de l'arg
            \item dans certain groupe log difficile alors que expo facile algo expo rapide
            \item exemple d'algo pour resoudre log
        \end{itemize}
    \item crypto sur c-e
        \begin{itemize}
            \item intro quelques info utile peut-être
            \item histoire et motivation
                \begin{itemize}
                    \item algo ss-expo pour log (crible gene du corps de nb)
                    \item travail dans corps assez large implique cout implementation,
                        transmission et tps de calcul augment
                    \item tout par du groupes points rationnels de $E$ 
                    \item variétés algébrique collection de points satisfaisant une équation
                        à plusieurs indéterminées
                    \item groupe d'ordre premier donc cyclique cad qu'il existe $P$ qui
                        engendre le groupe. conséquence on peut passer des versions
                        classique aux version ell sur les algo type el et dh
                    \item seuls les algo générique comme l'algo de shanks pouvaient
                        résoudre le log discret dans $E$
                    \item cela rendait l'attaque bcp plus diff sur $E$ que sur un corps de
                        nombre
                    \item conséquence niveau de sec satisfaisant, diminution de la
                        taille des données manipulées
                    \item gain de vitesse reduction des couts d'implé et transmi
                    \item attaque MOV c-e supersingulière exemple d'attaque
                \end{itemize}
        \end{itemize}
    \item probleme dh
        \begin{itemize}
            \item def
            \item relation avec log
        \end{itemize}
    \item plongement de veil
        \begin{itemize}
            \item degré de plongement attaque mov
        \end{itemize}
    \item echange cle dh
        \begin{itemize}
            \item 2015 prix turing
            \item fondement math
        \end{itemize}
    \item calculateur quantique
        \begin{itemize}
            \item partie sur la crypto
        \end{itemize}
    \item crypto sur c-e
        \begin{itemize}
            \item liste d'algo
            \item liste de courbe
            \item liste de primaire cryptographique
        \end{itemize}
    \item echange de clé
        \begin{itemize}
            \item TLS le plus utilisé
            \item prix turing 2016 pour dh
            \item avant clé publique
            \item info protocole dh
            \item info signature
        \end{itemize}
    \item signature
    \item TLS
    \item fonction de hachage
    \item EC multiplication
    \item curve25519
    \item théoreme de bézout geo algé (en et fr)
    \item expo by squaring
    \item one-way function
    \item diophante 
    \item eq diophantienne
    \item time complexity
    \item asymptotic analysis
    \item chiffrement par substitution
    \item droite à l'infini
    \item plan pro réel
    \item plan pro
    \item espace pro
    \item droite pro
    \item géo pro
\end{itemize}

miximum sur algo courbe elliptique avec un exemple sur les signature

handbook sur les c-e presentation général du sujet notamment sur le plan projectif et
l'intersection des axe $x,y,z$ au point $\mathcal{O}$
 
une page sur stack exchange avec le meme principe qu'au dessus et une autre encore peut etre
encore plus simple à comprendre

image pour les transformation projective (perspective)

j'ai l'article de kerckhoff en lien


\section{Plan}

\begin{itemize}
    \item introduction
        \begin{itemize}
            \item crypto
                \begin{itemize}
                    \item origine grecs
                    \item considération historique
                    \item étymologie, signification porté historique (guerre)
                    \item crypto sym
                        \begin{itemize}
                            \item césar
                            \item vigenere
                            \item cryptanalise
                            \item enigma
                            \item ce qui amène kerckhoff
                        \end{itemize}
                    \item principe de kerckhoff
                    \item crypto asym
                        \begin{itemize}
                            \item intro principe kerckhoff
                            \item cryptosys
                            \item D-H 1976
                            \item intro asym
                            \item principe termes mathématiques
                            \item enjeux asym
                        \end{itemize}
                    \item prob fac et log
                \end{itemize}
        \end{itemize}
    \item courbes elliptiques
        \begin{itemize}
            \item parallèle historique dans le sens en même temps
        \end{itemize}
    \item plan projectif
        \begin{itemize}
            \item espace pro
            \item droite projective
                \begin{itemize}
                    \item def
                    \item classe d'équivalence (explicite ce que signifie les coordonnées pro
                        non ?)
                    \item point de la droite
                    \item division en deux type
                        \begin{itemize}
                            \item $[x,1]$ droite affine
                            \item $[x,0]$ point à l'infini
                        \end{itemize}
                    \item plan projectif
                        \begin{itemize}
                            \item def (à changer pour être dans la continuité de la def
                                précédente.
                        \end{itemize}
                \end{itemize}
        \end{itemize}
    \item def g
        \begin{itemize}
            \item def de la courbe
            \item poly homogène $F$
            \item conséquence def : zero de $F$ et $f$ racine simples
            \item lemme sur delta
            \item demo lemme complète
            \item partie fine et point à l'infini
            \item $U,\phi,\mathcal{O},E$
            \item def point rationnel sur L
            \item remarque pas trop compris
            \item rationnel sur E
            \item exemples pas encore mis à choisir
        \end{itemize}
    \item loi de groupe
        \begin{itemize}
            \item def droite dans $\mathbb{P}^2$
            \item lemme unicité et existence de la droite
            \item démo pas fini!
            \item tangente rajoute des info si possible
            \item lemme je crois c'est par rapport à l'existence d'une unique tangente
            \item demo complète
            \item def equation tangente
            \item lemme des cas tangente
            \item demo complète
            \item prop cordes-tangentes
            \item demo complète
            \item th loi de groupe
            \item demo complète
            \item exemple et remarque à choisir
        \end{itemize}
    \item applications
        \begin{itemize}
            \item application domaine pratique et th
            \item exemples
            \item ce qui m'intéresse
            \item application crypto
            \item DH
            \item el-gammal
            \item ecsda
        \end{itemize}
\end{itemize}

% pistes:
% \begin{itemize}
%     \item Factorisation (algo $\N$ et $E$)
%     \item
% \end{itemize}



% Dans ce mémoire, nous allons nous intéresser aux courbes elliptiques et plus particulièrement
% au groupe abélien des courbes elliptiques. Ce qui va nous permettre d'utiliser ce groupe en
% cryptographie et présenter une façon plus efficace d'utiliser l'algorithme d'El-Gamal (EG.) (date) et le
% protocol de Diffie-Hellman (D-H.) (date).

% Dans un monde en constant évolution, notamment technique. Il est crucial de pouvoir
% améliorer, réinventer, ou même changer, des principes qui ont révolutionner à leur époque.
% C'est pourquoi, en (date) Klobnitz, à présenter une façon concrète d'utiliser les courbes
% elliptiques dans le cadre de la cryptographie. Ceci a permit d'apporter une nouvelle façon
% de faire de la cryptographie, tout en conservant des concepts eprouvé basé sur le problème
% du logarithme discret. Cette nouvelle approche, que l'on nommera
% version elliptique, contrairement à la version dite classique de chiffrement, qui est basé
% sur le groupe multiplicatif $\left( \Z /p\Z \right) ^{*}$ et sa commutativité, celle ci
% présente une diversité et complexité non négligeable. 

% En effet, que ce soit l'algorithme d'El Gamal ou le protocol de Diffie-Hellman, leur version
% classique est basé sur le générateur du groupe $\left( \Z / p\Z \right) ^{*}$, alors que
% leur version elliptique est basé sur les courbes elliptiques qui comme on le verra sont en grand
% nombre pour leur part. De plus, par construction du groupe des courbes elliptiques, plus
% abstrait, la résolution du logarithme discret est quasiment impossible sans l'aide d'ordinateur
% quantique extremement puissant (ref article Mme Abdelatif).

% \section{plan ?}

% \begin{itemize}
%     \item explication cryptographie
%         \begin{itemize}
%             \item sym et asym
%             \item rsa
%             \item El Gamal et D-H
%         \end{itemize}
%     \item histoire des fonctions elliptiques 
% \end{itemize}

% \section{La cryptographie}
% L'application première de notre construction étant la cryptographie, il me semble nécessaire de
% poser les bases de cette branche des mathématiques. Ceci nous permettra d'avoir une idée clair
% des différents concepts et enjeux qui la compose.

% Tout d'abord, définition ce qu'est la cryptographie. Métaphysiquement, c'est le fait de vouloir
% communiquer des messages, entre diverses entités et ceci de façon à ce que seul ces dernières
% n'aient connaissances du contenue du message.

% Cette définition personnel et trivial est basé sur comment dans l'histoire la cryptographie est
% apparu.

% De nos jours, le concept c'est énormément diversifié. La transmission de message reste un élément
% majeur de ce qu'est la cryptographie mais 
