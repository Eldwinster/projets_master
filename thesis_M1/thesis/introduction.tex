\chapter{Introduction}

Dans ce mémoire, je vais m'intéresser au groupe des courbes elliptiques et son application dans le
domaine de la cryptographie.

L'application première de notre construction étant la cryptographie, il me semble nécessaire de
poser les bases de cette branche des mathématiques. Ceci nous permettra d'avoir une idée clair
des différents concepts et enjeux qui la compose.

\section{Cryptographie}

La cryptographie trouve ses origines avec l'invention de l'écriture, en effet on en
retrouve des traces dés l'époque des Égyptiens vers 2000 a.v. J.C.

Elle a longtemps été considéré comme un art. Un art bien souvent en relation avec l'art de
la guerre.

À ce stade, on est en droit de ce demander ce que signifie la cryptographie. C'est un mot
d'étymologie d'origine grec. On peut le traduire par le fait de cacher ce qui est écrit.

On peut donc en conclure que c'est l'intention de transmettre un message de façon secret.
Autrement dit, on souhaiterai transmettre par écrit un message dont seul le destinataire
et l'expéditeur connaisse la signification du dit "message secret". 

On comprend donc tout l'importance de la cryptographie et son rôle important avec la
guerre.

Un première exemple bien connu de cryptographie est appelé le chiffrement de César. 
\textbf{date et attribution}

Cette méthode consiste à
prendre les lettres de l'alphabet et d'effectuer une transposition $n \in \N$, on peut
ainsi définir sur $\Z / 26\Z$ une bijection entre les lettres de l'alphabet et ce groupe.
L'entier $n$ est alors ce que l'on appelle la clé secrète, qui va permettre à la fois de
chiffrer et de déchiffrer un éventuel message.

il a d'autres exemples, comme le chiffrement de Vigenère, inventé par Blaise de Vigenère en
1586 dans le traité des chiffres paru en 1586 (on retrouve cependant une méthode analogue
dans un court traité de Giovan Battista Bellano paru en 1553), le chiffrement de Vigenère repose sur le même principe que
le chiffrement de César. À ceci prés que que la transposition $n$ est remplacé explicitement
par une clé secret que l'on peut noter $k$, qui est un mot secret ou bien une suite de lettre.
Ainsi on effectue la même opération que pour le chiffrement de César à la différence prés que
notre $n$ cette fois ci varie dans $\Z / 26\Z$ selon les lettres qui compose notre clé secrète $k$.

C'est deux exemples ne sont plus sûr. En effet, bien que le chiffrement de Vigenère essaye de
contourner le problème de l'analyse de fréquence d'apparition des lettres, qui permet de rendre
inefficace les chiffrement du type chiffrement de César avec un seul alphabet. Il aura tout de
même fallu III
siècle après son apparition, pour qu'en 1863 le major prussien Friedrich Kasiski publie une
méthode pour percé le chiffrement de Vigenère.

Cependant encore récemment, la machine enigma utilisé par les Allemands lors de la seconde
guerre mondial utilisé encore le principe liés au chiffrement Vigenère que l'on nomme chiffrement
par substitution polyalphabétique.

On retiendra que ces méthodes n'ont pas résisté à l'analyse de leurs
fonctionnement.

Ceci m'amène donc à parler d'un principes fondateur sur lequel est basé la cryptographie
moderne, qui repose essentiellement sur l'avènement de l'informatique qui à permit à la
cryptographie un renouveau historique. En effet, aujourd'hui elle n'est plus considérer comme
un art mais une vrai science avec tout le formalise que l'on est en droit d'attendre.

On appelle ce principe, le principe de Kerckhoffs, énoncé par Augustus Kerckhoffs en 1883 dans
un article en deux parties, "La cryptographie millitaire". Ce principe nous dit que la sécurité
ne dépend pas de la méthode de chiffrement mais sur le secret de la clé. Autrement dit, d'après
Kerckhoffs, une bonne méthode de chiffrement, ne doit pas se reposer sur le secret de sa
méthode mais sur le fait que même si elle est connue tant que l'on ne peut pas à partir de
celle-ci en déduire une méthode efficace pour retrouver la clé. Notre système
cryptographique est considérer comme sûr. 

C'est ainsi, qu'en 1976 W.Diffie et M.Hellman, lors de la National Computer Conference,
énonce une nouvelle méthode basé sur le principe de Kerckhoffs, sans pouvoir
cependant en fournir un exemple d'application.
\textbf{à vérifier}

Cette nouvelle méthode est la pierre fondatrice de la cryptographie moderne basé sur le
principe de clé publique et clé secret, qui sont deux clés distinctes. On parle alors de cryptographie asymétrique ou
cryptographie à clé publique.
L'asymétrie, ici est une asymétrie de l'information entre les clé ou l'une est publique, donc
connue, et l'autre non publique donc inconnue. De plus, chaque clé à sa propre fonction,
c'est à dire que la clé publique sert au chiffrement et la clé secrète au déchiffrement.

On peut se représenter le principe, en considérant deux personnes, traditionnellement
nommées
Alice, Bob.

Soit $\mathcal{M}$ un ensemble de chiffrements. On prend souvent pour $\mathcal{M}$
l'ensemble $\Z / n \Z$ ou bien un corps fini comme $\mathbb{F}_{q}$. Alice souhaite pouvoir se faire envoyer des
messages chiffrés de $\mathcal{M}$ de façon privée. Elle choisit une bijection $f_{A}\ :\
\mathcal{M}\to \mathcal{M}$ qui sera rendu publique, et elle seul en connaît la réciproque
$f_{A}^{-1}$. Le principe repose sur la grande difficulté de trouver $f_{A}^{-1}$ à partir de
$f_{A}$. 

Dans la situation où Bob envoie un message $x \in \mathcal{M}$. Il lui suffit
d'envoyer à Alice en clair l'élément $y = f_{A}(x)$. Pour déchiffrer le message Alice
calcul donc $f_{A}^{-1}(y)$, et retrouve le message $x$ de Bob. On appel ce genre de
fonction des fonctions à sens unique, car leurs réciproques sont difficiles à expliciter.

L'enjeu de la cryptographie à clé publique est donc de trouver ce type de fonction.
C'est à dire des opérations faciles à calculer mais dont le cheminement inverses est le plus
difficiles possible.

La cryptographie d'aujourd'hui est basée sur une hypothèse mathématiques éprouvé et sur deux
problèmes issu de la théorie des nombres. On a d'un côté l'hypothèse qu'il existe des
fonctions à sens unique, c'est à dire dont la réciproque est inexistante. Et de l'autre, on a
le problème de la factorisation d'un entier et celui du logarithme discret.

Le problème de la factorisation est basé sur le fait qu'il est facile de multiplier des
entiers pour en trouver d'autres mais il est difficile d'effectuer l'opération inverse à
savoir trouver les facteurs premier d'un entier.

Le problème du logarithme discret est le suivant: 

Soit $(G,.)$ un groupe abelien. Étant donné $g \in G$ et $n \in \N^*$, connaissant $g$ et
$g^{n}$, trouver $n$.

Ces deux problèmes sont la base sur lesquelles s'appuit bon nombres de système
cryptographique. On peut citer notamment RSA, protocole Diffie-Hellman ou encore
l'algorithme d'El-Gamal.


\section{Les courbes elliptiques}

En parallèle de l'histoire de la cryptographie. Se déroulait deux histoires tout aussi
anciennes liées à deux problèmes qui trouve leurs sources dans l'antiquité grec.

\subsection{Cercles et courbes elliptiques}
La première histoire est celle du cercle. En effet, depuis l'antiquité grecque, l'homme
s'est fortement intéressé à l'étude du cercle. Très vite, il s'est posé la question de
connaître la longueur du rayon d'un cercle et plus généralement la longueur d'un arc
de cercle. 

Cette idée fut réintroduite, mais cette fois ci dans un contexte encore plus générale
avec celui du calcul de la longueur d'un arc d'ellipse. En
effet, entre le XVIIème et XVIIIème siècle les mathématiciens Abel () et Jacobi
(), pour ne citer qu'eux, se sont mis à l'étude 
Et par la suite Weierstrass introduisis ce qu'on appelle aujourd'hui, les équations
normales de Weierstrass. 
\textbf{explique eq W}

En étudiant ce problème dans le corps des nombres complexes, on arrive à dégager une
structure de groupes sur les points rationnels de la courbe.

En 1985, indépendamment l'un de l'autre N.Koblitz et , on fournit un exemple d'application
possible du groupes des courbes elliptiques dans un corps fini à la cryptographie. C'est ce
groupes qui va nous intéresser.

\subsection{Diophante et courbes elliptiques}

La seconde histoire est celle des équations diophantiennes qui sont attribué à Diophante.

Le principe est de trouver tous les solutions entières d'une équation polynômiale à une ou
plusieurs inconnues dont les solutions sont des entiers.

Par exemple, une des équation diophantienne les plus simples à résoudre est l'équation $ax+by = c$ avec
les coefficients $a,b,c \in \Z$ et les inconnues $x,y \in \Z$ également. Sa résolution s'appuie
sur l'algorithme d'Euclide, le théorème de Bachet-Bézout et le lemme de Gauss.

Cependant, certaines équation diophantiennes ont nécessité les efforts conjugués de nombreux
mathématiciens sur plusieurs siècles pour les résoudre.

Ainsi, comme on peut s'en douter elle joue un rôle prépondérant dans la
cryptographie moderne qu'il s'agisse des plus connues comme l'équation présenté si dessus, ou
des plus sophistiquées, comme celles étudiées par L.Mordell du type $y^2 = x^3 + ax+b$ qui va
nous intéresser.

Le groupe des courbes elliptiques est le fruit de la rencontre entre ces trois
histoires.


% \[
% u = \int_{0}^{t} \frac{\dx}{\sqrt{1-x^2} }, \quad t \in ]-1,1[
% .\] 

% Qui n'est autre que la fonction $u = \arcsin(x)$. Or c'est plutôt la fonction inverse
% $\sin(x)$ qui est intéressante.

% Via la méthode de Jacobi, on peut montrer que la fonction
% \[
% t \to u = \int_{0}^{t} \frac{\dx}{\sqrt{1-x^2} }
% ,\] 
% admet une fonction réciproque $t(u)=\sin(u)$ définie sur tout le plan complexe, périodique de
% période $2\pi$, solution de l'équation différentielle :
% \[
% \left( \frac{\dt}{\du} \right) ^2 = 1 - u^2
% .\] 

% Au XVIIème et XVIIIème siècles, les mathématiciens se sont attaqué à la généralisation du
% problème précédent. En effet, ils ont cherché à déterminer la longueur d'un arc d'ellipse,
% d'équation
% \[
% \frac{x^2}{a^2}+\frac{y^2}{b^2}= 1
% .\] 

% Ce qui revient à calculer l'intégral, en posant $e = \sqrt{1 - \left( \frac{b}{a} \right)
% ^2} $, l'exentricité de l'ellipse :
% \[
% u = \int_{0}^{t} \frac{\sqrt{1-e^2x^2}}{\sqrt{1-x^2} } \dx, \quad t \in ]-1,1[
% .\] 

% La solution qui s'imposa fut de considérer cette intégrale comme une fonction à
% part entière.

% image

% , c'est à dire, où l'abscisse peut prendre plusieurs valeurs en ordonné. Ce
% qui ouvre l'étude à de nouvelle courbe comme le lemniscate de Benouilli. 

% Ainsi, Abel en 1827 et Jacobi en 1829 on indépendament attaqué la question de cette
% intégrale associée à une ellipse du point de vue des fonctions complexes, en
% considérant sa réciproque.


% Si Alice souhaite envoyer un message à Bob, elle chiffre sont message avec la clé publique de
% Bob, et ce dernier avec sa clé secrète, qu'il est le seul à connaître, peut déchiffrer le
% message d'Alice. Alice et Bob peuvent donc s'échanger des messages chiffré, via un canal
% non sécurisé, et ceux ci sans posséder de secret commun. Dans l'hypothèse qu'Eve interceptent
% les communications sur le canal non sécurisé, il lui est impossible un en temps
% raisonnable de déterminer la clé secrète à partir de la clé publique.

% Quand on parle
% de temps raisonnable ou d'efficacité, ceci est lié à la façon dont nos ordinateurs calcul. En effet, chaque
% opération arithmétique possède un certain coût en temps de calcul. Quand on parle de
% compléxité en temps, on calcul dans le pire des cas possible combien d'opération, il est
% nécessaire d'effectuer dans un algorithme. C'est pourquoi, un algorithme efficace est un
% algorithme qui coût peu cher en temps de calcul.


% Il n'aurat pas fallut longtemps pour trouver un exemple basé sur le principe énoncé par Diffie
% et Helman. En effet, dès 1977, Ronald \textbf{R}ivet, Adi \textbf{S}hamir et Leonard
% \textbf{A}dleman invente le cryptosystème asymétrique RSA.

% Il est basé sur le problème de la factorisation d'entier. En effet, pour trouver la clé
% secrète il faut être capable de résoudre ce problème de façon efficace. 


% À l'époque et encore jusqu'à récemment (source?) le problème de la factorisation des entiers
% était diffile à résoudre mais ceci est de moins en moins vrai avec les années. (source)



% En parallèle de l'histoire de la cryptographie. Une autre histoire fit son chemin de
% l'antiquité à nos jours. C'est celle du problème de la longueur du cercle et en particulier la longueur d'un arc
% de cercle.

% problème facto
% d-h log
% Alice bob eve
% fonction unique

% cercle diophante
% point ligne addition
% internet
% plan projectif

% En effet, dans un monde en constant évolution, notamment technique \cite{Kachigar2018}. Il est crucial de pouvoir
% améliorer, réinventer, ou même changer, des principes qui ont révolutionner à leur époque.
% C'est pourquoi,Koblitz \cite{Koblitz1987}, à présenter une façon concrète d'utiliser les courbes
% elliptiques dans le cadre de la cryptographie. Ceci a permit d'apporter une nouvelle façon
% de faire de la cryptographie, tout en conservant des concepts eprouvé basé sur le problème
% du logarithme discret \cite{Chabanol2021} et même le problème de la factorisation de entiers
% \cite{bibfacent}. Cette nouvelle cryptographie est basé sur le groupe abéliens des courbes
% elliptiques. Elle apporte bien évidemment ses lots d'avantages et d'inconvénients par rapport à
% la méthode, plus classique, qui elle est basé sur le groupe multiplicatif $(\Z /n\Z^{*})$, où $n$ est un entier.

% Tout d'abord, définition de ce qu'est la cryptographie. 

% D'après l'étymologie, qui est d'origine grec, c'est le fait de cacher ce qui à été écrit.

% On en déduit, ici que l'idée est de transmettre un éventuel message secret à une ou des
% personnes et l'on veut qu'elles soient les seuls à pouvoir comprendre le message.

% La première trace de la cryptographie dans l'histoire apparait déjà à l'époque des
% Égyptiens vers 2000 av. J.-C. en effet, on retrouver un

% Le premier exemple connue de cryptographie est appelé le chiffrement de César. Cette méthode consiste à
% prendre les lettres de l'aphabet et d'éffectuer une transposition $n \in \N$, on peut
% ainsi définir sur $\Z / 26\Z$ une bijection entre les lettres de l'alphabet et ce groupe.
% L'entier $n$ est alors ce que l'on appelle la clé secrète, qui va permettre à la fois de
% chiffrer et de déchiffrer un éventuel message. Quand la clé secrète a cette double utilité,
% on parle alors de cryptographie symétrique.

% La cryptographie symétrique à dominée une grande partie de notre histoire. En effet, il a fallu
% attendre la National Computer Conference de 1976, pour que W.Diffie et M.Hellman présente le
% concept de ce que l'on nomme aujourd'hui la cryptographie asymétrique. L'asymétrie ici
% décrit le décalage entre les informations. En effet, ici il y a deux clé qui entre en jeu. La
% clé publique et la clé secrete.

% La cryptographie moderne repose sur le principe de Kerckhoffs, énoncé en 1883 par Augustus
% Kerckhoffs

% De nos jours, traditionnelement nous désignons ces deux personnes par Alice et Bob. Mais
% il faut comprendre que Alice et Bob, sont comme deux variables dans l'ensemble de tous
% les entitées suceptible de communiquer entre elles. Autrement dit, elles peuvent représenter
% deux banques, une banque et un client, un ordinateur et un site internet, deux personnes qui
% souhaite communiquer entre-elles. Les exemples sont légion mais un premier problème va trés vite
% se poser et il faudra attendre les années 1970 avant que ce que l'on nomme aujourd'hui la
% cryptographie moderne apparaisent. 

% À notre epoque, on parle de cryptographie post-quantique qui comme on le verra va apporter
% sont lots de changement.


% Un des premiers problèmes que l'on rencontre est la transmission de la clé secrète
% entre Alice et Bob. En effet, non seulement la clé sert à chiffrer et déchiffrer mais elle
% doit être transmisse physiquement
