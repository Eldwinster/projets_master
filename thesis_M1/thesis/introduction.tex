\chapter{Introduction}

Dans ce mémoire, je vais m'intéresser au groupe des points rationnels d'une courbe elliptique
et donner des applications dans le
domaine de la cryptographie.

L'application première de notre construction étant la cryptographie, il me semble nécessaire
d'en poser les bases. Ceci nous permettra d'avoir une idée claire
des différents concepts et enjeux qui la composent.

\section{Cryptographie}

La cryptographie trouve ses origines avec l'invention de l'écriture, en effet on en
retrouve des traces dés l'époque des Égyptiens vers 2000 a.v. J.C.

Elle a longtemps été considérée comme un art. Un art bien souvent en relation avec l'art de
la guerre.

À ce stade, on est en droit de se demander ce que signifie la cryptographie. C'est un mot
d'étymologie d'origine grec. On peut le traduire par le fait de cacher ce qui est écrit.

On peut donc en conclure que c'est l'intention de transmettre un message de façon secrète.
Autrement dit, on souhaiterai transmettre par écrit un message dont seuls le destinataire
et l'expéditeur connaissent la signification du dit "message secret". 

On comprend donc tout l'importance de la cryptographie et son rôle important avec la
guerre.

Un première exemple bien connu de cryptographie est appelé le chiffrement de César. Utilisé par
Jules César pour chiffrer ses correspondances secrètes.

Cette méthode consiste à
prendre les lettres de l'alphabet et d'effectuer une transposition $n \in \N$ sur ce
dernier. On peut
ainsi définir sur $\Z / 26\Z$ une bijection entre les lettres de l'alphabet et ce groupe.
L'entier $n$ est alors ce que l'on appelle la clé secrète, qui va permettre à la fois de
chiffrer et de déchiffrer un éventuel message.

Il y a d'autres exemples, comme le chiffrement de Vigenère, inventé par Blaise de Vigenère en
1586 dans le traité des chiffres. Cette méthode de chiffrement appelé chiffrement de Vigenère repose sur le même principe que
le chiffrement de César. À ceci prés que que la transposition $n$ est remplacée explicitement
par une clé secrète que l'on peut noter $k$, qui est un mot secret ou bien une suite de
lettres.
Ainsi on effectue la même opération que pour le chiffrement de César à la différence prés que
notre $n$ cette fois ci varie dans $\Z / 26\Z$ selon les lettres qui composent notre clé secrète $k$.

Ces deux exemples ne sont plus sûrs. En effet, bien que le chiffrement de Vigenère essaye de
contourner le problème de l'analyse de fréquence d'apparition des lettres, qui permet de rendre
inefficace les chiffrement monoalphabétiques. Il aura tout de
même fallu trois
siècles après son apparition, pour qu'en 1863 le major prussien Friedrich Kasiski publie une
méthode pour percer le chiffrement de Vigenère.

Cependant encore récemment, la machine enigma utilisée par les Allemands lors de la seconde
guerre mondiale utilisait encore le principe lié au chiffrement Vigenère que l'on nomme chiffrement
par substitution polyalphabétique.

On retiendra que ces méthodes n'ont pas résistées à l'analyse de leurs
fonctionnements.

Ceci m'amène donc à parler d'un principe fondateur sur lequel est basé la cryptographie
moderne, qui repose essentiellement sur l'avènement de l'informatique qui à permis à la
cryptographie un renouveau historique. En effet, aujourd'hui elle n'est plus considérée comme
un art mais une vrai science avec tout le formalisme que l'on est en droit d'attendre.

On appelle ce principe, le principe de Kerckhoffs, énoncé par Augustus Kerckhoffs en 1883 dans
un article en deux parties, "La cryptographie millitaire". Ce principe nous dit que la sécurité
ne dépend pas de la méthode de chiffrement mais sur le secret de la clé. Autrement dit, d'après
Kerckhoffs, une bonne méthode de chiffrement ne doit pas se reposer sur le secret de sa
méthode mais sur le fait que même si elle est connue, tant que l'on ne peut pas à partir de
celle-ci en déduire une méthode efficace pour retrouver la clé. Notre système
cryptographique est considéré comme sûr. 

C'est ainsi, qu'en 1976 W.Diffie et M.Hellman, lors de la National Computer Conference,
énonces une nouvelle méthode basée sur le principe de Kerckhoffs. 

Cette nouvelle méthode appelée protocole de Diffie-Hellman est la pierre fondatrice de la
cryptographie moderne basée sur le
principe de clé publique et clé secrète, qui sont deux clés distinctes. On parle alors de cryptographie asymétrique ou
cryptographie à clé publique.
L'asymétrie, ici est une asymétrie de l'information entre les clés où l'une est publique, donc
connue, et l'autre non publique donc inconnue. De plus, chaque clé a sa propre fonction,
c'est à dire que la clé publique sert au chiffrement et la clé secrète au déchiffrement.

On peut se représenter le principe, en considérant deux personnes, traditionnellement
nommées
Alice et Bob.

Soit $\mathcal{M}$ un ensemble de chiffrements. On prend souvent pour $\mathcal{M}$
l'ensemble $\Z / n \Z$ ou bien un corps fini comme $\mathbb{F}_{p}$, où $p$ est
premier. Alice souhaite pouvoir se faire envoyer des
messages chiffrés de $\mathcal{M}$ de façon privée. Elle choisit une bijection $f_{A}\ :\
\mathcal{M}\to \mathcal{M}$ qui sera rendue publique, et elle seule en connaît la réciproque
$f_{A}^{-1}$. Le principe repose sur la grande difficulté de trouver $f_{A}^{-1}$ à partir de
$f_{A}$. 

Dans la situation où Bob envoie un message $x \in \mathcal{M}$. Il lui suffit
d'envoyer à Alice en clair l'élément $y = f_{A}(x)$. Pour déchiffrer le message Alice
calcule donc $f_{A}^{-1}(y)$, et retrouve le message $x$ de Bob. On appelle ces fonction des fonctions à sens unique, car leurs réciproques sont difficiles à expliciter.

L'enjeu de la cryptographie à clé publique est donc de trouver ce type de fonction.
C'est à dire des opérations faciles à calculer mais dont le cheminement inverse est le plus
difficile possible.

La cryptographie d'aujourd'hui est basée sur une hypothèse mathématique éprouvée et sur deux
problèmes issus de la théorie des nombres. On a d'un côté l'hypothèse qu'il existe des
fonctions à sens unique, c'est à dire dont la réciproque est inexistante ou très difficile à
expliciter. Et de l'autre, on a
le problème de la factorisation d'un entier et celui du logarithme discret.

C'est sur ce problème de la factorisation d'un entier, qu'est basé le système cryptographique RSA, inventé en 1977 par Rivest,
Shamir et Adleman. Son efficacité repose sur le fait que connaissant un entier $n$, qui
est produit de deux entiers premiers $p$ et $q$. Il est difficile de déterminer $p$ et
$q$. Cependant, les algorithmes de factorisation ayant énormément évolué, la taille des
clés de chiffrement obtenue par RSA doit être de plus en plus grande. Ainsi, pour des
tailles de clé plus petites, disons 256-bit, le groupe des points rationnels offre une sécurité équivalente à
des clés obtenue à l'aide du groupe $\Z / n\Z$ de taille 4096-bit. Ainsi, plus l'environnement
est contraignant et plus l'avantage des courbes elliptiques se fait ressentir.

L'efficacité du groupe des points rationnels d'une courbe elliptique est basée sur le
problème du logarithme discret. 

Le problème du logarithme discret est le suivant: 

Soit $(G,.)$ un groupe abélien. Étant donné $g \in G$ et $n \in \N^*$, connaissant $g$ et
$g^{n}$, trouver $n$.

\section{Les courbes elliptiques}

En parallèle de l'histoire de la cryptographie, se déroulaient deux histoires tout aussi
anciennes liées à deux problèmes qui trouvent leurs sources dans l'antiquité grecque.

\subsection{Origine des courbes elliptiques}
La première histoire est celle du cercle. En effet, depuis l'antiquité grecque, l'homme
s'est fortement intéressé à l'étude du cercle. Ce qui de nos jours, revient à étudier la courbe
algébrique d'équation $x^2 + y^2 = 1$. Un problème qui a irrigué les commencements des
mathématique est de déterminer la longueur d'un arc de cercle. Ceci revient à calculer
l'intégrale
\[
\int \frac{\dx}{\sqrt{1-x^2} }
.\] 

Ce qui nous permet d'obtenir la fonction $\arcsin(x)$ et par la méthode de Jacobi d'obtenir la
fonction réciproque $\sin(x)$. Cette recherche de la réciproque de l'intégrale obtenue
à partir de l'équation d'un cercle, amène naturellement à la même question mais plus générale,
sur les ellipses. Ainsi, en étudiant la courbe algébrique d'équation $ \frac{x^2}{a^2} +
\frac{y^2}{b^2} = 1$, les mathématiciens du XVIIème et XVIIIème siècle ont pu déterminer
à l'aide d'une série convergente l'intégrale obtenue à partir de l'équation algébrique de
l'ellipse. Toujours à cette époque, ils ont cherché à savoir si cette intégrale pouvait être
exprimée en termes de fonctions élémentaires. Liouville en 1837 prouva que cela n'est pas
possible.

Ainsi, ils leur a fallu considérer cette intégrale comme une fonction à part entière,
cependant comme pour le cercle la fonction la plus naturelle est sa réciproque. 

C'est ainsi qu'Abel en 1827 et indépendamment Jacobi en 1829 étudient la question de cette
intégrale associée à une ellipse du point de vue des fonctions complexes, en considérant la
réciproque. 

Ce qui amène l'étude de l'intégrale elliptique de la forme
\[
\int \frac{\dx}{\sqrt{4x^2-g_2-g_3} }
,\] 
et l'introduction de la fonction elliptique $\rho$ solution de l'équation
\[
\rho'^2 = 4 \rho^3 - g_2 \rho - g_3
,\] 
où $\rho$ est la réciproque de l'intégrale elliptique.

Ce qui en étudiant cette nouvelle équation est celle d'une courbe elliptique. Son étude dans
le plan complexe permet la construction du groupe abélien des points rationnels d'une
courbe elliptique.

Ainsi en 1985, indépendamment l'un de l'autre N.Koblitz \ref{article} et V.Miller, ont fourni un exemple
d'application à la cryptographie, celui du groupe des points rationnels d'une courbe elliptique dans un corps fini. C'est ce
groupe qui va nous intéresser.

\subsection{Diophante}

La seconde histoire est celle des équations diophantiennes. Ce parallèle permet d'introduire
l'idée derrière la construction du groupe abélien des points rationnels d'une courbe
elliptique.

Le principe est de trouver tous les solutions entières d'une équation à une ou
plusieurs indéterminées, dont les solutions sont des entiers.

Par exemple, une des équations diophantiennes la plus simple à résoudre est l'équation $ax+by =
c$, avec
les coefficients $a,b,c$ des entiers relatifs et les indéterminées $x,y \in \Z$ également. Sa résolution s'appuie
sur l'algorithme d'Euclide, le théorème de Bachet-Bézout et le lemme de Gauss.

Cependant, certaines équations diophantiennes ont nécessité les efforts conjugués de nombreux
mathématiciens sur plusieurs siècles pour les résoudre.

Ainsi, comme on peut s'en douter elles jouent un rôle prépondérant dans la
cryptographie moderne qu'il s'agisse des plus connues comme l'équation présentée ci-dessus, ou
des plus sophistiquées, comme celles étudiées par L.Mordell du type $y^2 = x^3 + ax+b$ qui va
nous intéresser.

Les courbes elliptiques sont à la fois un problème facile à décrire, c'est "l'ensemble des
solutions d'une cubique" et pourtant bien qu'elles semblent simples à première vue, de profond
théorèmes régissent
leur comportement, et beaucoup de questions naturelles à propos des courbes elliptiques sont encore
ouvertes.

Par exemple pour compter le nombre de sphères nécessaire pour former une
pyramide. On est amené à calculer la somme suivante
\[
1 + 4 + 9 + 16 + \ldots + n^2 = \frac{n\left( n+1 \right) \left( 2n+1 \right) }{6}
,\] 
pour une pyramide de $4$ étages ont à besoin de 30 sphères 
\[
1 + 4 + 9 + 16 = \frac{4\left( 4 + 1 \right) \left( 2 \times 4 + 1 \right) }{6} = 30
.\] 

Ainsi, si l'on étudie les solutions entières de l'équation suivante
\[
y^2 = \frac{x\left( x+1 \right) \left( 2x+1 \right) }{6}
,\] 
cela revient à calculer les
points rationnels de la courbe elliptique associée. 

Mis à part, les points $(0,0)$ et $(1,1)$, quelles sont les autres solutions entières que l'on
peut trouver ? L'idée est de partir de la droite qui intersecte les deux points à
savoir $y=x$ et de résoudre le système d'équation que l'on obtient, c'est à dire
\[
\begin{cases}
    y^2 &= \frac{x\left( x+1 \right) \left( 2x+1 \right) }{6} \\
    y &= x 
\end{cases}
.\] 

On obtient alors le polynôme 
\[
x^3 - \frac{3x^2}{2}+\frac{x}{2} = 0
,\] 
et l'on cherche alors sa troisième racine à l'aide des relations de Viète, on sait que
la somme des racines vaut 
\[
r + 0 + 1 = \frac{3}{2}
.\] 

Ainsi, on obtient le point $(\frac{1}{2},\frac{1}{2})$ et par symétrie de la courbe par rapport à l'axe des
abscisse le point
$(\frac{1}{2},-\frac{1}{2})$. Bien que ce dernier ne soit pas une solution entière. On peut en
répétant le même procédé en prenant les points $\left( \frac{1}{2},-\frac{1}{2} \right)
$ et $\left( 1,1 \right) $ on trouve d'autres solutions entières à savoir le point (24,70) qui
est donc solution de l'équation de la courbe. Cependant, l'intérêt de cet exemple n'est pas le calcul en
lui-même mais la méthode en elle-même. J'entends par là, le fait de prendre deux points de la
courbe et de tracer la droite entre ces deux points. C'est cette action qui va nous intéresser et
qui amène la question suivante.

Si l'on prend deux points d'une courbe elliptique, et que l'on trace la droite entre ces deux
points,
cette droite intersecte elle toujours la courbe en un 3ème point ?

À première vue, la réponse à cette question est non. En effet, une tangente verticale à la
courbe, c'est à dire parallèle à l'ordonnée n'intersecte pas la courbe elliptique. Cependant, à l'aide de la géométrie
projective, on peut rendre cela possible. Ce qui rend possible la création du groupe
des points rationnels d'une courbe elliptique.

Ce qui aujourd'hui, nous permet d'utiliser les courbes elliptiques pour construire des
protocoles de chiffrement robustes et largement répandus.

Le groupe des points rationnels d'une courbe elliptique est le fruit de la rencontre entre ces trois
histoires que sont la cryptographie, le cercle et les équations diophantiennes.

\begin{center}
    \textbf{Références}
    La liste complète est présente à la fin de ce mémoire. Les références principales
    dont je me suis servit pour écrire ce mémoire sont les suivantes :
    \begin{itemize}
        \item \cite[]{KrausCF}, \cite[]{KrausCP} et \cite[]{KrausCE}. 
        \item \cite[]{Delaunay} 
        \item \cite[]{Baigneres2003} 
        \item \cite{Deglise2013} 
    \end{itemize}

\end{center}

\begin{center}
    Merci à Mme Abdelatif pour avoir supervisé ce mémoire.
\end{center}

\begin{center}
    \textbf{Calculs numérique}

    Les calculs disponibles dans les différents exemples, ont été réaliser à l'aide des
    différents script disponible sur cet artcile de vulgarisation sur le sujet \cite[]{Kun2014}.
\end{center}

