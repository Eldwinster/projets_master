\chapter{Les courbes elliptiques: histoire et liens avec la cryptographie}
le cryptosystème RSA, inventé par Ronald \textbf{R}ivet, Adi
\textbf{S}hamir et Leonard \textbf{A}dleman en 1977
\section{Introduction}

Dans ce mémoire, nous allons nous intéresser aux courbes elliptiques et plus particulièrement
au groupe abélien des courbes elliptiques. Ce qui va nous permettre d'utiliser ce groupe en
cryptographie et présenter une façon plus efficace d'utiliser l'algorithme d'El-Gamal (EG.) (date) et le
protocol de Diffie-Hellman (D-H.) (date).

Dans un monde en constant évolution, notamment technique. Il est crucial de pouvoir
améliorer, réinventer, ou même changer, des principes qui ont révolutionner à leur époque.
C'est pourquoi, en (date) Klobnitz, à présenter une façon concrète d'utiliser les courbes
elliptiques dans le cadre de la cryptographie. Ceci a permit d'apporter une nouvelle façon
de faire de la cryptographie, tout en conservant des concepts eprouvé basé sur le problème
du logarithme discret. Cette nouvelle approche, que l'on nommera
version elliptique, contrairement à la version dite classique de chiffrement, qui est basé
sur le groupe multiplicatif $\left( \Z /p\Z \right) ^{*}$ et sa commutativité, celle ci
présente une diversité et complexité non négligeable. 

En effet, que ce soit l'algorithme d'El Gamal ou le protocol de Diffie-Hellman, leur version
classique est basé sur le générateur du groupe $\left( \Z / p\Z \right) ^{*}$, alors que
leur version elliptique est basé sur les courbes elliptiques qui comme on le verra sont en grand
nombre pour leur part. De plus, par construction du groupe des courbes elliptiques, plus
abstrait, la résolution du logarithme discret est quasiment impossible sans l'aide d'ordinateur
quantique extremement puissant (ref article Mme Abdelatif).

\section{plan ?}

\begin{itemize}
    \item explication cryptographie
        \begin{itemize}
            \item sym et asym
            \item rsa
            \item El Gamal et D-H
        \end{itemize}
    \item histoire des fonctions elliptiques 
\end{itemize}

\section{La cryptographie}
L'application première de notre construction étant la cryptographie, il me semble nécessaire de
poser les bases de cette branche des mathématiques. Ceci nous permettra d'avoir une idée clair
des différents concepts et enjeux qui la compose.

Tout d'abord, définition ce qu'est la cryptographie. Métaphysiquement, c'est le fait de vouloir
communiquer des messages, entre diverses entités et ceci de façon à ce que seul ces dernières
n'aient connaissances du contenue du message.

Cette définition personnel et trivial est basé sur comment dans l'histoire la cryptographie est
apparu.

De nos jours, le concept c'est énormément diversifié. La transmission de message reste un élément
majeur de ce qu'est la cryptographie mais 
