\newpage
\begin{center}
    \textbf{Perspectives}

    La suite naturelle de ce qui a été étudié dans ce mémoire est l'étude des structures de
    groupes sur le groupe $(E,+)$ défini sur un corps fini. Pour cela, on utilise les
    morhpismes de groupes de $E(\overline{K})$ pour ensuite utiliser le morphisme de Frobenius
    et le corps des points de torsion. Ainsi à l'aide du théorème fondamental et du théorème de
    Hasse, dont je n'ai pas parlé mais dont les énoncés sont disponibles dans ce cours
    \cite[p15-30]{KrausCE}, on peut étudier l'ordre du groupes $E(K)$. Bien qu'on en n'ait pas
    une formule explicite on obtient une borne supérieur ainsi que l'intervalle de Hasse. Ceci
    nous permet à partir de l'ordre d'un point retrouver l'ordre du groupe. La question du
    cardinal du groupe reste cependant encore une question ouverte. Grâce à cette étude on peut
    par ailleurs donner un critère pour différencier les courbes singulières des courbes
    ordinaires.

    Dans une autre optique on pourrait s'intéresser à la cryptographie post-quantique, en
    commençant par lire cet article de vulgarisation sur le sujet \cite{Kachigar2018}, ce qui
    permettrait de comprendre les enjeux relatif au fonctionnement des calculateurs quantiques.
    Et comprendre les défis qu'amène le développement des ordinateurs quantiques. On peut
    notamment cité l'agorithme de Shor qui permet de résoudre le problème de la facorisation
    des entiers sur lequel est basé le système RSA.
\end{center}
