\chapter{Applications}
Le groupe abélien $(E,+)$ des points rationnels d'une courbe elliptique et même les courbes elliptiques en général,
ont de nombreuses applications que ce soit dans le domaine pratique, ou bien dans le domaine
théorique.

En effet, on peut notamment citer leurs utilisations dans la mécanique classique dans la
description du mouvement des toupies. Elles interviennent également en théorie des nombres, dans la démonstration
du dernier théorème de Fermat.

Enfin, on les retrouve aussi en cryptologie, dans le problème de la factorisation des entiers.

Dans ce mémoire, on s'intéresse à leur application en cryptographie. Où elles ont
permis notamment la réduction de la taille des clés cryptographiques.

\section{Cryptosystèmes elliptiques}

Aujourd'hui, le groupe $E$ des points rationnels d'une courbe elliptique intervient notamment pour l'échange de clé
et les signatures numériques.

Nous allons nous intéresser à deux cryptosystèmes asymétriques, à savoir le protocole d'échange de
clés Diffie-Hellman, ainsi que l'algorithme d'El-Gamal, basé sur le principe du protocole de
Diffie-Hellman, qui permet d'échanger des messages chiffrés à l'aide d'une clé
publique et de déchiffrer les messages avec la clé secrète de chaque utilisateur.

La force des cryptosytèmes asymétriques réside dans la difficulté, voir l'impossibilité
actuelle dans le cas elliptique, de résoudre le problème du logarithme discret de façon
générale.

Dans le cas des cryptosystèmes à clé publique classique, on s'appuie sur le groupe multiplicatif d'un corps fini et de son groupe des inversibles. Ce qui réduit grandement notre choix comparé aux versions elliptique des algorithmes équivalent.

En effet, dans le cas elliptique, on remplace le groupe multiplicatif sur un corps fini par le
groupe des points rationnels d'une courbe elliptique. L'avantage de cette méthode est que pour
un corps fini  $K$ donné, on dispose généralement de nombreux choix de courbes elliptiques $E$
sur $K$. Autrement dit, on a de nombreux groupes $E(K)$, pour utiliser efficacement un
cryptosystème asymétrique elliptique contrairement aux versions classiques comme énoncé plus
haut, où
l'on ne dispose que du groupe des inversibles $K^{*}$.

\subsection{Problème du logarithme discret elliptique}

Le problème du logarithme discret pour les courbes elliptiques est analogue à celui
des corps fini, dont on peut trouver un énoncé dans ce cours \cite[p18]{KrausCf}.

Soit $K$ un corps et $E$ une courbe elliptique définie sur $K$. Les points $K$-rationnels
formant un groupe abélien, donne un cadre pour le problème du logarithme discret.

\begin{definition}
    Soit $E$ une courbe elliptique définie sur $K$ et $Q \in E(K)$. Connaissant le point $P
    \in E(K)$, le problème du logarithme discret consiste à trouver $n \in \N$, s'il existe.
    tel que $P = nQ$.
\end{definition}

Un tel entier $n$ n'existe pas toujours. De plus $E(K)$ n'est pas nécessairement cyclique. Afin
d'essayer de résoudre ce problème, on peut utiliser l'algorithme de Silver, Pohlig et
Hellman \cite[p20]{KrausCF} ou l'algorithme Baby step - Giant step \cite[p38]{KrausCE}.

\begin{remarque}
    Le problème du logarithme discret est généralement beaucoup plus difficile à
    résoudre dans le groupe des points rationnels d'une courbe elliptique $E$ sur un
    corps fini $K$, que celui dans $K^*$. 
\end{remarque}

Cela est dû au fait que, les algorithmes de résolution du problème du logarithme discret
pour le groupe multiplicatif d'un corps fini, sont de plus en plus efficaces pour résoudre le
problème comme on peut le voir dans cet article \cite{Lipton2013}. Ainsi, il est
de plus en plus clair que la taille des clés, requises pour maintenir un haut niveau de
sécurité pour le protocole RSA, se doivent d'être de plus en plus grande (i.e. 4096-bit).
Alors que les courbes elliptiques s'en sortent avec des clés beaucoup plus petite de l'ordre
de 256-bit. La raison est essentiellement mathématique, c'est-à-dire, que l'addition des points
rationnels d'une courbe elliptique est moins bien comprise que la multiplication pour les
entiers. Ainsi, la complexité apparente du groupe rend le problème intrinsèquement plus
difficile. C'est pourquoi tant qu'il n'y a pas de méthode générale efficace pour résoudre
le problème dans le contexte des courbes elliptiques. On peut avoir une sécurité aussi efficace
que RSA pour des clés beaucoup plus petites. 

Bien que l'on ne connaisse pas tous les paramètres qui rendent ou non une courbe
mathématiquement sûre. On connaît tout de même un certain nombres d'attaques contre certains
paramètres bien spécifiques.
Ainsi, il est recommander :
\begin{itemize}
    \item D'être sûr que l'ordre du point choisi ait une factorisation courte (i.e. $2p,3p$ ou
        $4p$, pour $p$ premier). Autrement on est vulnérable à une attaque basé sur le théorème
        des restes chinois, la plus importante étant celle de Pohlig-Hellman.
    \item D'être sûr que la courbe choisie ne soit pas supersingulière. Sinon on peut réduire
        le problème du logarithme discret à un problème différent dans un groupe plus simple.
    \item Si la courbe $E$ est définie sur $\Z / p \Z$, avec $p$ premier, on doit vérifier que
        le nombre de points de la courbe ne soit pas égale à $p$. Ce type de courbe est dite
        "anormale" et on peut réduire le problème du logarithme à la version additive sur les
        entiers.
    \item Ne pas choisir $\mathbb{F}_{2^{m}}$ avec $m$ petit. On peut utiliser l'algorithme de
        rho Pollard qui est très efficace contre ce genre de corps fini.
    \item Si on utilise le corps fini $\mathbb{F}_{2^{m}}$ alors il est plus sûr de choisir
        $m$ premier.
\end{itemize}

On peut retrouver dans ce cours \cite[p17-18]{Delaunay} des exemples d'algorithmes pour 
choisir convenablement des points rationnels ou choisir de bonnes courbes. Il y a également,
d'autres recommandation sur le choix de $E$.

Ainsi, comme on peut s'y attendre quant à l'énoncé du problème du logarithme discret les points
qui vont nous intéresser sont les multiples d'un point. On peut retrouver un algorithme pour le
calcul de ce point dans ce cours \cite[p10]{Delaunay}. Il est basé sur la décomposition de $n$
dans la base $2$. Ainsi, on amener à effectuer peu d'opération pour obtenir un multiple d'un
point rationnels de la courbe. On à la formule suivante pour le calcul de $nP$ 
\[
nP = \sum_{i= 0}^{n} a_{i}2^{i}P
,\] 
avec les $a_{i}$ dans ${0,1}$.

Autrement dit, pour calculer $19P$, il vient $19 = 1 \times 2^{0} + 1 \times 2^{1} + 0 \times
2^{2} + 0 \times 2^{3} + 1 \times 2^{4} $ et ainsi $19P = P + 2P + 16P$ et on calcul 9
doublements et 3 additions.

Ainsi, quand on effectue le rapport entre le nombre d'opération nécessaire pour calculer le
multiple d'un point et celui pour calculer sont logarithme. On est amener à effectuer beaucoup
plus d'opération pour le logarithme et ceci est la base de la sécurité des protocole suivant.


\subsection{Protocole Diffie-Hellman}

Dans ce tout ce qui suit Alice et Bob sont deux personnes qui souhaite s'échanger soit un message, soit
une clé secrète. Cependant, il faut bien comprendre qu'il peuvent également représenter deux
entités qui souhaitent communiquer via des messages chiffrés ou bien s'échanger une clef
secréte via des cannaux publics. Par entité, j'entends soit des banques, des entreprises ou
tout ce qui serait suceptible de vouloir communiquer secrètement entre eux.

Alice et Bob souhaitent s'échanger publiquement une clé secrète commune. Pour cela ils se mettent d'accord pour la construire selon le procédé suivant:

\begin{description}
    \item[1)] Ils choisissent un corps fini $K$ et une courbe elliptique $E$ définie sur $K$, pour que le problème du logaritme discret soit difficile à résoudre dans le groupe $E(K)$. Ils choisissent un point $P \in E(K)$. Ils rendent alors publique le triplet $(K,E,P)$.

    \item[2)] Alice choisit un entier naturel secret non nul $a$ et calcule le point $P_a=aP$, qu'elle transmet publiquement à Bob.

    \item[3)] Bob procède de la même façon en choisissant un entier naturel secret, non nul,
        $b$, et il calcule de son côté le point $P_b=bP$, qu'il transmet publiquement à Alice.

    \item[4)] Alice calcule le point $aP_b=a(bP)$.

    \item[5)] Bob calcule le point $bP_a=b(aP)$.
\end{description}

Ils ont ainsi construit leur clé secrète commune qui est le point $abP$.

\begin{probleme}[Diffie-Hellman]
   Connaissant $P$, $aP$ et $bP$ dans $E(K)$, comment déterminer $abP$ ? 
\end{probleme}

On ne sait pas à ce jour résoudre ce problème sans calculer $a$ ou $b$, autrement dit, sans
savoir résoudre le problème du logarithme discret dans $E(K)$. Cela étant, on n'a pas la preuve
qu'il n'existe pas d'autres moyens pour y parvenir. Ainsi le problème de Diffie-Hellman est une
hypothèse plus forte que le problème du logarithme discret car elle en dépend mais elle dépend
aussi du fait qu'on ne sache pas s'il existe un autre moyen pour résoudre ce problème.

\begin{exemple}
    Soit la courbe définit par 
    \[
    E : y^2 = x^3 + 324x + 1287
    ,\] 
    sur le corps $\mathbb{F}_{p}$, avec $p=3851$ qui est premier. 

    Soit le point $P \in E(\mathbb{F}_{p})$, avec $P = (920;303)$.

    La courbe et le point sont publiques.

    Alice choisit l'entier secret $a=1194$ et calcule $aP=1194P=(2067,2178) \in
    E(\mathbb{F}_{3851})$ et l'envoie à Bob.

    Bob choisit l'entier secret $b = 1759$ et calcul $bP = 1759P = (3684,3125) \in
    E(\mathbb{F}_{p})$.

    Finalement,

    Alice calcule $a(bP)=1194.(3684,3125)=(3347,1242) \in E(\mathbb{F}_{p})$ et

    Bob calcule $b(aP) = 1759.(2067,2178) = (3347,1242) \in E(\mathbb{F}_{p})$.

    Ils peuvent alors déduire du point échangé à l'aide de la coordonnée $x=3347$ une clé
    secrète pour la cryptographie symétrique.
\end{exemple}

\subsection{Algorithme d'El Gamal}

Alice souhaite envoyer un message chiffré à Bob. Pour se faire elle choisit un corps fini $K$,
une courbe elliptique $E$ définie sur $K$ de sorte que le problème du logarithme discret soit
difficile à résoudre dans le groupe $E(K)$. Elle choisit ensuite un point $P \in E(K)$. Enfin
elle choisit son entier naturel secret, non nul, $s$ et calcule le point $A=sP$.

Elle rend ainsi public le quadruplet 
\[
    (K,E,P,A)
.\] 

C'est la base de ce qui va permettre à Alice et Bob de pouvoir communiquer de façon
confidentielle entre eux.

Ainsi, pour que Bob puisse envoyer un message chiffré $M \in E(K)$ à Alice, il choisit secrétement un entier non nul $k$ et calcule les points
\[
M_1=kP \quad \text{et} \quad M_2=M+kA
.\] 
Il transmet alors publiquement à Alice le couple $(M_1,M_2)$. C'est donc la phase d'encryptage du message $M$.

Pour qu'Alice puisse déchiffrer le message $M$, elle doit calculer le point
\[
M_2-sM_1
.\] 
Ce qui lui permet grâce au calcul suivant de retrouver $M$:
\[
M_2-sM_1=M+kA-s(kP)=M+kA-kA=M
.\] 
