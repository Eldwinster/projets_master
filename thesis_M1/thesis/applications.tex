\chapter{Applications}
Le groupe abélien $(E,+)$ des points rationnels d'une courbes elliptiques et même les courbes elliptiques en générale,
ont de nombreuses applications que ce soit dans le domaine pratique, ou bien dans le domaine
théorique.

En effet, on peut notamment citer leurs utilisation dans la mécanique classique dans la
description du mouvement des toupies. Elles interviennent également en théorie des nombres, dans la démonstration
du dernier théorème de Fermat.

Enfin, on les retrouve aussi en cryptologie, dans le problème de la factorisation des entiers.

Dans ce mémoire, on s'intéresse à leur application en cryptographie. Où elles ont
permit notamment la réduction de la taile des clés cryptographique.

\section{Cryptosystèmes elliptiques}

Aujourd'hui, le groupe $E$ des points rationnels d'une courbe elliptique intervient notamment pour l'échange de clé
et les signatures numériques.

Nous allons, nous intéressé à deux cryptosystème asymétrique, à savoir le protocol d'échange de
clés Diffie-Hellman, ainsi que l'algorithme d'El-Gamal, basé sur le principe du protocol de
Diffie-Hellman, qui permet d'échanger des messages messages chiffré à l'aide d'une clé
publique et de déchiffrer les messages avec la clé secrète de chaque utilisateur.

La force des cryptosytèmes asymétrique réside dans la difficulté, voir l'impossibilité actuel dans le cas elliptique, de résoudre le problème du logarithme discret que nous énoncerons par la suite.

Dans le cas des cryptosystèmes à clé publique classique, on s'appuie sur le groupe multiplicatif d'un corps fini et de son groupe des inversibles. Ce qui réduit grandement notre choix comparé aux versions elliptique des algorithmes équivalent.

En effet, dans le cas elliptique, on remplace le groupe multiplicatif sur un corps fini par le groupe des points rationnels d'une courbe elliptique. L'avantage de cette méthode est que pour un corps fini  $K$ donné, on dispose généralement de nombreux choix de courbes elliptiques $E$ sur $K$. Autrement dit, on a de nombreux groupes $E(K)$, pour utiliser efficacement un cryptosystème asymétrique élliptique contrairement aux versions classique comme énoncé plus haut où
l'on ne dispose que du groupe des inversible $K^{*}$.

Dans ce qui suit Alice et Bob sont deux personnes qui souhaite s'échanger soit un message, soit une clef secrète. Cependant, il faut bien comprendre qu'il peuvent également représenter deux entité qui souhaitent communiqué via des messages chiffrés ou bien s'échanger une clef secréte via cannaux publique. Par entité, j'entends soit des banques, des entreprises ou tout ce qui serait suceptible de vouloir communiqué secretement entre eux.

De plus le choix des clés secret s'effectue de façon aléatoire dans le respect des conditions de chaque cas.

\subsection{Protocol Diffie-Hellman}

Alice et Bob souhaite s'échanger publiquement une clé secrète commune. Pour cela ils se mettent d'accord pour la construire selon le procédé suivant:

\begin{description}
    \item[1)] Ils choisissent un corps fini $K$ et une courbe elliptique $E$ définie sur $K$, pour que le problème du logaritme discret soit difficile à résoudre dans le groupe $E(K)$. Ils choisissent un point $P \in E(K)$. Ils rendent alors publique le triplet $(K,E,P)$.

    \item[2)] Alice choisit un entier naturel secret non nun $a$ et calcul le point $P_a=aP$, qu'elle transmet publiquement à Bob.

    \item[3)] Bob procède de la même façon en choisissant un entier naturel secret, non nul, $b$, et il calcul de son côté le point $P_b=bP$, qu'il transmet publiquement à Alice.

    \item[4)] Alice calcul le point $aP_b=a(bP)$.

    \item[5)] Bob calcul le point $bP_a=b(aP)$.
\end{description}

Ils ont ainsi construit leur clé secret commun qui est le point $abP$.

\begin{probleme}[Diffie-Hellman]
    
\end{probleme}

\subsection{Algorithme d'El Gamal}

Alice souhaite envoyer un message chiffré à Bob. Pour se faire elle choisit un corps fini $K$, une courbe elliptique $E$ définie sur $K$ de sorte que le problème du logarithme discret soit difficile à résoudre dans le groupe $E(K)$. Elle choisit ensuite un point $P \in E(K)$. Enfin elle choisit sont entier naturel secret, non nul, $s$ et calcul et calcul le point $A=sP$.

Elle rend ainsi public le quadruplet 
\[
    (K,E,P,A)
.\] 

C'est la base de ce qui va permettre à Alice et Bob de pouvoir communiquer de façon confidentiel entre eux.

Ainsi, pour que Bob puisse envoyer un message chiffré $M \in E(K)$ à Alice, il choisit secrétement un entier non nul $k$ et calcules les points
\[
M_1=kP \quad \text{et} \quad M_2=M+kA
.\] 
Il transmet alors publiquement à Alice le couple $(M_1,M_2)$. C'est donc la phase d'encryptage du message $M$.

Pour qu'Alice puisse déchiffrer le message $M$, elle doit calculer le point
\[
M_2-sM_1
.\] 
Ce qui lui permet grâce au calcul suivant de retrouver $M$:
\[
M_2-sM_1=M+kA-s(kP)=M+kA-kA=M
.\] 
