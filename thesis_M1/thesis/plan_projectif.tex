\chapter{Le plan projectif et courbes sur le plan projectif}

La définition que nous allons utilisé pour les courbes elliptiques étant dans le plan
projectif.

Introduisons brièvement, ce qu'est un espace projectif, ainsi que les objets dont nous aurons
besoin à savoir des points et des droites.

Intuitivement, un espace projectif permet de rendre homogène un espace vectoriel. On entend
par là, de raisonner indépendammenent des proportionalités pour ne plus considérer que les
directions. L'idée nous vient de la formalisation mathématique de la perspective. 
L'espace projectif nous permet d'identifier des droites à des points. Ce qui rend possible
le fait de raisonner en termes de coordonnées et de pouvoir effectuer des calculs formel.

Dans un premier temps, voici la définition du plan projectif.

\begin{definition}
    Le plan projectif sur $\overline{K}$, que l'on note $\mathbb{P}^2(\overline{K})$ ou
    $\mathbb{P}^2$, est l'ensemble quotient
    \[
    \overline{K}^3 - \left\{ (0,0,0) \right\} / \sim
    \] 
    où $\sim$ est la relation d'équivalence telle que pour tous $\left( x,y,z \right) $ et
    $\left( x',y',z'\right) $ non nuls de $\overline{K}^3$,

    \[
    \left( x,y,z \right) \sim \left( x',y',z' \right) \iff \exists \lambda \in
    \overline{K}^{*} \quad \left( x',y',z' \right) = \lambda \left( x,y,z \right) 
    .\] 
    Pour tous $\left( x,y,z \right) $ non nuls dans $\overline{K}^2$, on note $\left[
    x,y,z \right] $ sa classe d'équivalence appelée coordonnées homogènes.
\end{definition}

Pour définir la notion de courbe sur le plan projectif, on utilise pour cela des polynômes à
trois variables. La définition du plan projectif, nous dit qu'un point peut être représenté par
plusieurs triplets différents mais équivalents. Il semble alors naturel de ne considérer
que des polynôme $F(X,Y,Z) \in K[X,Y,Z]$ tels que si $F(x,y,z) = 0$ alors $F(\lambda x, \lambda
y, \lambda z) = 0$ pour tout $\lambda$ non nul.

\begin{definition}
    Un polynôme $F(X,Y,Z)$ est homogène de degré $d$ s'il vérifie l'égalité suivante :
    \begin{align}
        \label{eq:homogene}
    F(\lambda X, \lambda Y, \lambda Z) = \lambda^{d}F(X,Y,Z)
    .\end{align}
    Ces polynôme sont une somme de monômes de la forme $\sum_{i+j+k=d} X^{i}Y^{j}Z^{k}$
    et vérifie l'égalité \eqref{eq:homogene}.
\end{definition}

On peut maintenant énoncé la définition d'une courbe sur le plan projectif.

\begin{definition}
    \label{def:courbe}
    Une courbe $C$ sur le plan projectif $\mathbb{P}^2$ est l'ensemble des solutions d'une
    équation polynomiale 
    \[
    C : \quad F(X,Y,Z)=0
    ,\] 
    où $F$ est un polynôme homogène de degré supérieur ou égal à 1. Le degré de la courbe
    est le degré de ce polynôme.
\end{definition}

Un point $P = (x,y,z) \in C$, s'il vérifie que $F(x,y,z)=0$. En effet si on choisit une autre
représentation de ce point dans le plan projectif $\mathbb{P}^2$, par exemple pour
$P'=\lambda P$, on a
\[
F(P')=F(\lambda P) = \lambda^{d}F(P) = 0
.\] 
Toute représentation d'un point de la courbe est un zéro du polynôme $F$.

\begin{definition}
    Une courbe $D \in \mathbb{P}^2$ définie par un polynôme homogène de degré 1 est appelée
    une droite.

    Une courbe $C \in \mathbb{P}^2$ définie par un polynôme homogène de degré $3$ est appelée
    une cubique.
\end{definition}

% Dans un premier temps pour comprendre les concepts liées aux espaces projectif. Partons de la
% droite projective qui est un espace projectif de dimension 1.

% \begin{definition}
%     La droite projective sur $\overline{K}$, que l'on note $\mathbb{P}^1(\overline{K})$ ou
%     $\mathbb{P}^1$, est l'ensemble quotient
%     \[
%     \overline{K}^2 - \left\{ (0,0) \right\} / \sim
%     \] 
%     où $\sim$ est la relation d'équivalence telle que pour tous $\left( x,y \right) $ et
%     $\left( x',y'\right) $ non nuls de $\overline{K}^2$,

%     \[
%     \left( x,y \right) \sim \left( x',y' \right) \iff \exists \lambda \in
%     \overline{K}^{*} \quad \left( x',y' \right) = \lambda \left( x,y \right) 
%     .\] 
    % Pour tous $\left( x,y \right) $ non nuls dans $\overline{K}^2$, on note $\left[ x,y
    % \right] $ sa classe d'équivalence appelée coordonnées homogènes.
% \end{definition}

% Un point de la droite projective est donc définie par les droites vectorielles privée de
% l'origine.

% On a alors deux types de point, les points de la forme $\left[ x,1 \right] $ et ceux de
% la forme $\left[ x,0 \right] $.

% Pour le premier type, comme $y \neq 0$ on a naturellement $\lambda = \frac{1}{y}$. On
% obtient alors l'intersection de tous les droites vectorielles avec la droite affine $y=1$, ceux
% qui forme la droite sur $K$.

% Le deuxième type correspond à l'ensemble des droites affines parallèle à l'ordonné, de plus
% pour tous $x \neq 0$, on a $\lambda = \frac{1}{x}$, donc peut importe la valeur de $x$ les
% droites s'intersectent au point $[1,0]$.

% Intuivement, la droite projective sur $K$ est une droite affine sur $K$ complétée par un
% point, appelé point à l'infini que l'on note $\mathcal{O}$. Ainsi dans l'espace projectif deux droites parallèles s'intersectent en
% un point à l'infini. Quand on parle de droite à l'infini, on désigne par cette droite
% l'ensemble des points à l'infini.

% Le point à l'infini étant l'intersection des droites paralléles.
% % On obtient donc l'ensemble $\mathbb{P}^1 = K \cup \mathcal{O}$, avec $\mathcal{O} = \left[ 1,0 \right] $, le point à
% % l'infini. 


% Donnons également, la définition de la droite projective espace projectif de dimension 1,



% L'espace projectif permet de formaliser et généraliser à toute dimension la notion de
% droite à l'infini dans le plan projectif.

% Le plan projectif permet d'introduire la notion d'homogénéisation des équations de courbes algébrique.
% Ce procédé permet à partir de l'équation initial d'une courbe algébrique du plan usuel,
% admettant une équation de la forme $P(x,y)=0$ où $P \in K[X,Y]$, d'obtenir une équation d'une courbe qui
% est dans le plan projectif, et donc de prolonge la courbe initiale à la droite à
% l'infini. Autrement dit, on définit un plan affine en choisissant une droite projective
% quelconque associée à ce plan, qui est la droite à l'infini 

% Pour bien comprendre ce qu'est le plan projectif, parlons d'abord de la droite projective
% $\mathbb{P}^{1}$, qui est un espace projectif de dimension 1.

% Si $K$ est le corps des nombres réels, alors la droite projective réelle est obtenue en
% intersectant les droites vectorielles de $\R^2$ avec le cercle unité.

% On a donc l'ensemble des droites vectorielles de $\R^2$ qui sont de la forme $\left[ x,y
% \right] $ avec $y\neq 0$.


% Ceci permet d'obtenir à partir de l'équation initiale, une nouvelle équation qui prolonge la
% courbe initiale à la droite infini. C'est ce qui nous permet de construire un élément neutre
% qui soit bien définie pour la loi du groupe.

% Pour se donner une idée des parties qui compose le plan projectif raisonnons avec les réels.

% Pour $K=\R$, par définition le plan projectif $\mathbb{P}^2(\R)$, est l'ensemble quotient de
% tous les vecteurs colinéaire de $\R^3$ privée de l'origine. Ainsi, pour tous vecteurs $(x,y,z)
% \in \mathbb{P}^2$, un représentant de classe est de la forme $\left( \lambda x, \lambda y,
% \lambda z \right) $, et on note $\left[ x,y,z \right] $ sa classe d'équivalence. 


% Ainsi, on a les points de la forme $\left[ x,y,0 \right] $ qui forme la droite à l'infini et
% les autres points peuvent tous être écrit de la forme $\left[ x,y,1 \right] $ et il forme le
% plan affine.

% Pour comprendre ce que sont la droite à l'infini et le plan affine, parlons de
% l'homogénéisation des droites.

% Les équations d'une droite dans le plan affine sont de la forme $ax + by + c = 0$, où $a$ et
% $b$ deux réels non tous deux nuls et $c$ un réels quelconque. Ainsi, l'équation homogène
% associé est de la forme $P(x,y,z) = ax + by + cz = 0$, c'est un polynome à plusieurs
% indéterminées dont tous les monômes non nuls sont de même degré total, donc un polynôme
% homogène de degré 1.

% On a alors qu'un point du plan projectif de coordonées homogène $\left[ x,y,z \right] $ est sur
% la droite projective si et seulement si $P(x,y,z) = 0$. Autrement dit, les zéros du polynôme
% homogène sont les points du plan affine dans $\mathbb{P}^2$ et ceci indépendament du choix des
% coordonnées homogènes. 

% En effet, comme $\left[ x,y,z \right] = \left\{ (x,y,z) \in
% \R^3 \mid \exists \lambda \in \R^{*},\ (x,y,z) = \lambda(x,y,z) \right\}$, par homogénéité, on
% a
% \[
% P(\lambda x, \lambda y, \lambda z) = \lambda P(x,y,z) 
% \] 
% donc $P(x,y,z)=0$.

% Traditionnelement, on prend pour $z$ la valeur $1$, ainsi un point dans le plan affine est de
% la forme $\left[ x,y,1 \right] $ tel que $P(x,y,1)=P(x,y)$, ce qui montre que le point $\left[
% x,y,1\right] $ est sur la droite projective si et seulement si le point $(x,y)$ est sur la
% droite affine. Ainsi, on a bien opéré un prolongement de la droite affine initial en une droite
% projective.

% Si l'on pose 
% \[
% U_0 = \left\{ [x,y,z] \in \mathbb{P}^2(\R) \mid z \neq 0 \right\} 
% ,\] 

% on a l'application $\phi : U_0 \to \R^2$ définie par :
% \[
% \phi([x,y,z]) = \left( \frac{x}{z},\frac{y}{z} \right)
% .\] 
% C'est une bijection dont la réciproque est donnée par :
% \[
% \phi^{-1}(x,y) = [x,y,1]
% .\] 

% Le plan affine dans l'espace projectif est donc une représentation du plan euclidien $\R^2$.
% L'image suivante permet de comprendre ce que l'on obtient.

% Par suite, si $z=0$, on a deux choix qui s'offre à nous à savoir $x \neq 0$ ou $y \neq 0$. Pour
% fixer les idées prennons $y = 1$. Ainsi, un point de la droite projective est de la forme
% $\left[ x,1,0 \right] $ tel que P(x,1,0)=P(x,1). Ainsi, le point $[x,1,0]$ est sur la droite
% projective si et seulement si le point $(x,1)$ est sur la droite affine initial. 




% bijection 

% image

% droite à l'infini

% droite et point les cas

% presente une courbe la canonique

% parle du determinant et de son influence sur la courbe

% parle de la forme normal de W



