\chapter{Le plan projectif $\mathbb{P}^2(\overline{K})$}

La définition formel du plan projectif $\mathbb{P}^2$, est la suivante :

\begin{def}
    Le plan projectif sur $\overline{K}$, que l'on note $\mathbb{P}^2(\overline{K})$ ou
    $\mathbb{P}^2$, est l'ensemble quotient
    \[
    \overline{K}^3 - \left\{ (0,0,0) \right\} / \sim
    \] 
    où $\sim$ est la relation d'équivalence telle que pour tous $\left( x,y,z \right) $ et
    $\left( x',y',z'\right) $ non nuls de $\overline{K}^3$,

    \[
    \left( x,y,z \right) \sim \left( x',y',z' \right) \iff \exists \lambda \in
    \overline{K}^{*} \quad \left( x',y',z' \right) = \lambda \left( x,y,z \right) 
    .\] 
\end{def}

Il permet d'introduire la notion d'homogénéisation des équations de courbes algébrique. C'est
un procéder qui à partir d'une courbe usuel du plan, qui admet une équation de la forme
$P(x,y)=0$, où $P \in K[X,Y]$.

Ceci permet d'obtenir à partir de l'équation initiale, une nouvelle équation qui prolonge la
courbe initiale à la droite infini. C'est ce qui nous permet de construire un élément neutre
pour notre loi de groupe bien défini.

Pour se donner une idée des parties que compose le plan projectif raisonnons avec les réels.

Pour $K=\R$, par définition le plan projectif $\mathbb{P}^2(\R)$, est l'ensemble quotient de
tous les vecteurs colinéaire de $\R^3$ privée de l'origine. Ainsi, pour tous vecteurs $(x,y,z)
\in \mathbb{P}^2$, un représentant de classe est de la forme $\left( \lambda x, \lambda y,
\lambda z \right) $, et on note $\left[ x,y,z \right] $ sa classe d'équivalence. 

Ainsi, on a les points de la forme $\left[ x,y,0 \right] $ qui forme la droite à l'infini et
les autres points peuvent tous être écrit de la forme $\left[ x,y,1 \right] $ et il forme le
plan affine.

Pour comprendre ce que sont la droite à l'infini et le plan affine, parlons de
l'homogénéisation des droites.

Les équations d'une droite dans le plan affine sont de la forme $ax + by + c = 0$, où $a$ et
$b$ deux réels non tous deux nuls et $c$ un réels quelconque. Ainsi, l'équation homogène
associé est de la forme $P(x,y,z) = ax + by + cz = 0$, c'est un polynome à plusieurs
indéterminées dont tous les monômes non nuls sont de même degré total, donc un polynôme
homogène de degré 1.

On a alors qu'un point du plan projectif de coordonées homogène $\left[ x,y,z \right] $ est sur
la droite projective si et seulement si $P(x,y,z) = 0$. Autrement dit, les zéros du polynôme
homogène sont les points du plan affine dans $\mathbb{P}^2$ et ceci indépendament du choix des
coordonnées homogènes. 

En effet, comme $\left[ x,y,z \right] = \left\{ (x,y,z) \in
\R^3 \mid \exists \lambda \in \R^{*},\ (x,y,z) = \lambda(x,y,z) \right\}$, par homogénéité, on
a
\[
P(\lambda x, \lambda y, \lambda z) = \lambda P(x,y,z) 
\] 
donc $P(x,y,z)=0$.

Traditionnelement, on prend pour $z$ la valeur $1$, ainsi un point dans le plan affine est de
la forme $\left[ x,y,1 \right] $ tel que $P(x,y,1)=P(x,y)$, ce qui montre que le point $\left[
x,y,1\right] $ est sur la droite projective si et seulement si le point $(x,y)$ est sur la
droite affine. Ainsi, on a bien opéré un prolongement de la droite affine initial en une droite
projective.

Si l'on pose 
\[
U = \left\{ [x,y,z] \in \mathbb{P}^2(\R) \mid z \neq 0 \right\} 
,\] 

On a une bijection qui nous permet d'identifier
parle de la bijection entre le plan euclidien et le plan affine

fait le point continue
