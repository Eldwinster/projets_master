\chapter{Géométrie projective}
\begin{center}
    Dans ce chapitre, je défini et donne quelques propositions qui m'ont permis de
    mieux comprendre l'intéret de l'étude des courbes elliptiques dans le plan
    projectif. Ainsi que d'introduire les propositions qui me permettent de faire le parallèle
    entre la partie algébrique du raisonnement et l'interprétation géométrique
    associé. Notamment, j'introduis une proposition qui me permet de donner une la
    démonstration géométrique de l'associativité de la loi du groupe des points rationnels
    d'une courbe elliptique.

\end{center}

\section{Le plan projectif et les courbes sur le plan projectif}

La définition que nous allons utilisé pour les courbes elliptiques étant dans le plan
projectif.

Introduisons brièvement, ce qu'est un espace projectif, ainsi que les objets dont nous aurons
besoin à savoir le plan, des points et des droites.

Intuitivement, un espace projectif permet de rendre homogène un espace vectoriel. On entend
par là, de raisonner indépendammenent des proportionalités pour ne plus considérer que les
directions (i.e. les droites de l'espace). L'idée nous vient de la formalisation mathématique de la perspective. 
% L'espace projectif nous permet d'identifier des droites à des points.

Dans un premier temps, voici la définition du plan projectif.

\begin{definition}
    \label{def:planP2}
    
    Le plan projectif sur $\overline{K}$, que l'on note $\mathbb{P}_{2}(\overline{K})$ ou
    $\mathbb{P}_{2}$, est l'ensemble quotient
    \[
    \overline{K}^3 - \left\{ (0,0,0) \right\} / \sim
    \] 
    où $\sim$ est la relation d'équivalence telle que pour tous $\left( x,y,z \right) $ et
    $\left( x',y',z'\right) $ non nuls de $\overline{K}^3$,

    \[
    \left( x,y,z \right) \sim \left( x',y',z' \right) \iff \exists \lambda \in
    \overline{K}^{*} \quad \left( x',y',z' \right) = \lambda \left( x,y,z \right) 
    .\] 
    Pour tous $\left( x,y,z \right) $ non nuls dans $\overline{K}^3$, on note $\left[
    x,y,z \right] $ sa classe d'équivalence et $(x,y,z)$ sont appelées les coordonnées homogènes.
\end{definition}

Pour définir la notion de courbe sur le plan projectif, on utilise pour cela des polynômes à
trois variables. La définition du plan projectif, nous dit qu'un point peut être représenté par
plusieurs triplets différents mais équivalents. Il semble alors naturel de ne considérer
que des polynôme $F(X,Y,Z)$ dans l'anneau de polynômes $K[X,Y,Z]$ tels que si $F(x,y,z) = 0$ alors $F(\lambda x, \lambda
y, \lambda z) = 0$ pour tout $\lambda$ non nul.

\begin{definition}
    \label{polyHomo}
    
    Un polynôme $F(X,Y,Z)$ est homogène de degré $d$ s'il vérifie l'égalité suivante :
    \begin{align}
        \label{eq:polyHomo}
    F(\lambda X, \lambda Y, \lambda Z) = \lambda^{d}F(X,Y,Z)
    .\end{align}
    Ces polynôme sont une somme de monômes de la forme $\sum_{i+j+k=d} X^{i}Y^{j}Z^{k}$
    et vérifie l'égalité \eqref{eq:polyHomo}.
\end{definition}

On peut maintenant énoncé la définition d'une courbe sur le plan projectif.

\begin{definition}
    \label{def:courbeP2}
    Une courbe $E$ sur le plan projectif $\mathbb{P}_{2}$ est l'ensemble des solutions d'une
    équation polynomiale 
    \[
    E : \quad F(X,Y,Z)=0
    ,\] 
    où $F$ est un polynôme homogène de degré supérieur ou égal à 1. Le degré de la courbe
    est le degré de ce polynôme. On a alors 
    \[
    E = \left\{ (x,y,z) \in \mathbb{P}_{2} \mid F(X,Y,Z)=0 \right\} 
    .\] 
\end{definition}


Un point $P = [x,y,z] \in E$, s'il vérifie que $F(x,y,z)=0$ ne dépend que de sa classe
d'équivalence. En effet si on choisit un autre
représentant de classe de ce point dans le plan projectif $\mathbb{P}_{2}$, par exemple pour
$P'=\lambda P$, on a
\[
F(P')=F(\lambda x, \lambda y, \lambda z) = \lambda^{d}F(x,y,z) = 0
.\] 
Tous les représentant de classe d'un point de la courbe sont des zéro du polynôme $F$ donc un
point de la courbe.

\begin{definition}
    \label{def:cubique}
    
    Une courbe $D \in \mathbb{P}_{2}$ définie par un polynôme homogène de degré 1 est appelée
    une droite.

    Une courbe $E \in \mathbb{P}_{2}$ définie par un polynôme homogène de degré $3$ est appelée
    une cubique.
\end{definition}

\section{Lien avec la représentation affine}
On peut faire le lien entre une courbe du plan projectif telle qu'on vient de la définir
et une courbe du plan affine habituel, que l'on note $\mathbb{A}_{2}$ ou
$\mathbb{A}_{2}(\overline{K})$. 

Soit une courbe $E$ de $\mathbb{P}_{2}(\overline{K})$, donnée par un
polynôme homogène $F$ de degré $d$ tel que 

\[
E : F(X,Y,Z)=0
.\] 

Posons 
\[
U = \left\{ [x,y,z] \in \mathbb{P}_{2}(\overline{K}) \mid z \neq 0 \right\} 
.\] 

On dispose de l'application $\phi : U \to \mathbb{A}_{2}(\overline{K})$ définie par
\[
\phi([x,y,z])=\left( \frac{x}{z},\frac{y}{z} \right) 
.\] 

C'est une bijection, dont l'application réciproque est donnée par la formule
\[
\phi^{-1}(x,y)=\left[ x,y,1 \right] 
.\] 

En effet, si $P = (x,y) \in \mathbb{A}_{2}$, on a

\[
\phi \circ \phi^{-1} (x,y) = \phi([x,y,1]) = (x,y) \in \mathbb{A}_{2}
,\] 
et si $P = [x,y,z] \in U$ alors
\[
\phi^{-1} \circ \phi ([x,y,z]) = \phi^{-1}(\frac{x}{z},\frac{y}{z}) = [\frac{x}{z},\frac{y}{z},1] =
[x,y,z] \in U
.\] 

La bijection $\phi$ fait correspondre ce point du plan projectif avec un point
$\phi(P) = (\frac{x}{z},\frac{y}{z})$ du plan affine $K$. 

Il y a donc une correspondance entre les points du plan projectif et ceux du plan affine.
On peut également remarquer comme il a été dit plutôt que deux représentants de classes
distincts d'un même point $P$ dans $\mathbb{P}_{2}$ donne lieu à un unique point dans
$\mathbb{A}_{2}$.

Par ailleurs, si on a $F$ un polynôme homogène de degré $d$ et que $F(x,y,z) = 0$ alors 

\[
F(\frac{x}{z},\frac{y}{z},1) = \lambda^{d} F(x,y,z) = 0
,\] 
avec $\lambda = \frac{1}{z}$.

On peut alors définir une courbe dans le plan affine $\mathbb{A}_{2}$ à partir d'une
courbe dans le plan projectif $\mathbb{P}_{2}$, dont les points $(x,y)$ seront solution de
l'équation $f(x,y) = 0$, avec $f$ définie par

\begin{align}
    \label{eq:bijectionP2}
    f(x,y) = F(x,y,1)
.\end{align}


Autrement dit, si on identifie la partie affine à l'ensemble $U$, le plan projectif s'interprète comme la
réunion de $\overline{K}^2$ avec la droite à l'infini.
On note $\mathbb{P}_{1} = \left\{ (x,y,0) \in \mathbb{P}_{2} \mid F(x,y,0) = 0 \right\} $
l'ensemble des points à l'infini, on parle souvent de droite à l'infini quand on parle de
$\mathbb{P}_{1}$.

Dans ce cas, on a :
\[
\mathbb{P}_{2} \approx U \cup \mathbb{P}_{1}
.\] 


\begin{remarque}
    En ce qui concerne les courbes elliptique nous verrons que seul un point de la courbe
    appartient à $\mathbb{P}_{1}$.
\end{remarque}

\section{Courbes irréductibles}

\begin{definition}
    \label{def:polyIrr}
    Un polynôme $P$ est factorisable lorsqu'il existe deux polynômes $Q$ et $R$ non
    constants de degré strictement inférieur à celui de $P$ tels que
    \[
    P(X,Y,Z) = Q(X,Y,Z)R(X,Y,Z)
    .\] 

    Un polynôme est irréductible lorsqu'il n'est pas factorisable.

    On peut factorisé un polynôme en un produit de polynômes irréductibles. 

    Si $F
    \in K[X,Y,Z]$ est un polynôme, il existe $P_1,\ldots,P_{n} \in K[X,Y,Z]$ des polynômes
    tous
    irréductible tels que
    \[
    F(X,Y,Z)= P_1(X,Y,Z)\ldots P_{n}(X,Y,Z)
    .\] 

    Les $P_{i}$ sont appelées les composantes irréductibles du polymôme $F$.
\end{definition}

\begin{definition}
    \label{def:courbeIrr}
    
    Soit une courbe $E$ définie par le polynôme $F(X,Y,Z) = 0$. La courbe est irréductible si
    le polynôme $F$ est irréductible.
\end{definition}

On dit que deux courbes $E_1$ et $E_2$ n'ont pas de composante commune quand leur
composantes irréductibles sont distinctes.

\section{Intersection d'une cubique et d'une droite dans le plan projectif}

\begin{proposition}
    \label{prop:intersectionED}
    
    L'ensemble des points à l'intersection d'une cubique $E$ et d'une droite $D$ est fini si, et
    seulement si, ces deux courbes n'ont pas de composante irréductible en commun.
\end{proposition}

Autrement dit, cela revient à montrer l'ensemble $E \cap D$ est fini, si et seulement si, les
polynômes associés à $E$ et $D$ sont irréductibles. 

\begin{demonstration}
    \begin{itemize}
        \item 
    Dans un premier temps, on montre que si $E$ et $D$ ont une composante commune alors $E
    \cap D$ est infini. Autrement dit, comme ils ont une composante commune, cela revient à
    dire que l'un est produit de l'autre. Ce qui nous permettra de conclure par contraposée.

    Soient $E$ l'ensemble des solutions de $F_1(X,Y,Z) = 0$ et $D$ l'ensemble des solutions de
    $F_2(X,Y,Z) = 0$. 

    On a le degré de $F_1$ qui vaut trois et comme $D$ est une droite, $F_2$ est un polynôme
    homogène de degré 1, il est donc irréductible. 

    Dire que $E$ et $D$ ont une composante irréductible commune, revient à dire
    qu'il existe une courbe $C$ d'équation $F_3(X,Y,Z) = 0$ de degré $0<d<3$.

    Ainsi, on peut écrire $F_1$ sous la forme
    \[
    F_1(X,Y,Z) = F_2(X,Y,Z)F_3(X,Y,Z)
    ,\] 
    comme $F_1(X,Y,Z) = 0$, il vient 
    \begin{align*}
        F_2(X,Y,Z)F_3(X,Y,Z) = 0 & \iff \left( F_2(X,Y,Z)=0 \right)  \ou \left( F_3(X,Y,Z) = 0 \right)  \\
                                 & \iff \exists P \in \mathbb{P}_{2},\ P \in D \cup C = E 
    .\end{align*}
    Donc $D$ est contenu dans $E$. Comme il existe une infinité de droite, il existe une
    infinité de point intersection de la courbe et de la droite.

    Ainsi, on a montré que  si la courbe $E$ et la droite $D$ ont une composante irréductible
    commune alors il existe une infinité de points dans l'ensemble $D \cup C = E$, et
    comme $E \cap D = D$, par conséquent l'ensemble est infini.

    Autrement dit, par contraposée, si $E$ et $D$ se coupent en un nombre fini de points, elles n'ont pas
    de composante commune.

    \item Dans un second temps, supposons que $E$ et $D$ n'ont pas de composante commune, ce
        qui revient à dire que les polynômes $F_1$ et $F_2$ sont irréductibes. Donc à montrer
        que $F_1$ est irréductible ce qui permettra de conclure grâce au nombre fini de racine
        de $F_1$.

    La droite $D$ est défini par un polynôme homogène de degré 1
    \[
    D : F_2(X,Y,Z) = aX + bY + cZ
    .\] 
    Soit $P=[x,y,z]$ un point de l'intersection entre $E$ et $D$.

    \begin{itemize}
        \item Si $z \neq 0$,

            Les coordonnées affines du point $P$ vérifient alors l'équation 
            \[
            f_2(x,y) = ax+by+c = 0
            .\] 

            On peut supposer, par symétrie, que $b \neq 0$. Dans ce cas, on a

            \[
            y = - \frac{ax + c}{b}
            .\] 

            L'équation suivante doit alors être vérifié

            \[
            f_1(x, - \frac{ax+c}{b}) = 0
            .\] 

            C'est un polynôme en $x$ non nul car $E$ et $D$ n'ont pas de composante commune.
            Il admet donc un nombre fini de racines.

        \item Si $z = 0$ alors,

            les coordonnées homogène de $P$ vérifient alors le système d'équation suivante

            \[
                \begin{cases}
                    AX + bY &= 0 \\
                    F_1(X,Y,0) &= 0
                \end{cases}
            .\] 

            En supposant par symétrie, que $b \neq 0$, on voit que les coordonnées
            homogènes de $P$ doivent vérifier

            \[
            F_1(X,- \frac{a}{b}X,0)=0
            .\] 

            Cette équation est un polynôme en $X$ non nul puisque $E$ et $D$ n'ont pas de
            composante commune. il admet donc un nombre fini de racines.

            Ainsi, si $E$ et $D$ n'ont pas de composante commune. elles se coupent en un
            nombre fini de points.
    \end{itemize}
    \end{itemize}
\end{demonstration}

\begin{corollaire}
    Soit $E$ une cubique irréductible et $D$ une droite. La courbe plane $E$ et la droite
    projective $D$ se coupent en un nombre fini de points.
\end{corollaire}

\begin{demonstration}
    En effet, comme $E$ est irréductible, son polynôme homogène associé $F(X,Y,Z) = 0$
    est lui aussi irréductible et donc il ne possède pas de composante irréductible. D'où
    l'énoncé.
\end{demonstration}


% Dans un premier temps pour comprendre les concepts liées aux espaces projectif. Partons de la
% droite projective qui est un espace projectif de dimension 1.

% \begin{definition}
%     La droite projective sur $\overline{K}$, que l'on note $\mathbb{P}^1(\overline{K})$ ou
%     $\mathbb{P}^1$, est l'ensemble quotient
%     \[
%     \overline{K}_{2} - \left\{ (0,0) \right\} / \sim
%     \] 
%     où $\sim$ est la relation d'équivalence telle que pour tous $\left( x,y \right) $ et
%     $\left( x',y'\right) $ non nuls de $\overline{K}_{2}$,

%     \[
%     \left( x,y \right) \sim \left( x',y' \right) \iff \exists \lambda \in
%     \overline{K}^{*} \quad \left( x',y' \right) = \lambda \left( x,y \right) 
%     .\] 
%     Pour tous $\left( x,y \right) $ non nuls dans $\overline{K}_{2}$, on note $\left[ x,y
%     \right] $ sa classe d'équivalence appelée coordonnées homogènes.
% \end{definition}

% Un point de la droite projective est donc définie par les droites vectorielles privée de
% l'origine.

% On a alors deux types de point, les points de la forme $\left[ x,1 \right] $ et ceux de
% la forme $\left[ x,0 \right] $.

% Pour le premier type, comme $y \neq 0$ on a naturellement $\lambda = \frac{1}{y}$. On
% obtient alors l'intersection de tous les droites vectorielles avec la droite affine $y=1$, ceux
% qui forme la droite sur $K$.

% Le deuxième type correspond à l'ensemble des droites affines parallèle à l'ordonné, de plus
% pour tous $x \neq 0$, on a $\lambda = \frac{1}{x}$, donc peut importe la valeur de $x$ les
% droites s'intersectent au point $[1,0]$.

% Intuivement, la droite projective sur $K$ est une droite affine sur $K$ complétée par un
% point, appelé point à l'infini que l'on note $\mathcal{O}$. Ainsi dans l'espace projectif deux droites parallèles s'intersectent en
% un point à l'infini. Quand on parle de droite à l'infini, on désigne par cette droite
% l'ensemble des points à l'infini.

% Le point à l'infini étant l'intersection des droites paralléles.
% % On obtient donc l'ensemble $\mathbb{P}^1 = K \cup \mathcal{O}$, avec $\mathcal{O} = \left[ 1,0 \right] $, le point à
% % l'infini. 


% Donnons également, la définition de la droite projective espace projectif de dimension 1,



% L'espace projectif permet de formaliser et généraliser à toute dimension la notion de
% droite à l'infini dans le plan projectif.

% Le plan projectif permet d'introduire la notion d'homogénéisation des équations de courbes algébrique.
% Ce procédé permet à partir de l'équation initial d'une courbe algébrique du plan usuel,
% admettant une équation de la forme $P(x,y)=0$ où $P \in K[X,Y]$, d'obtenir une équation d'une courbe qui
% est dans le plan projectif, et donc de prolonge la courbe initiale à la droite à
% l'infini. Autrement dit, on définit un plan affine en choisissant une droite projective
% quelconque associée à ce plan, qui est la droite à l'infini 

% Pour bien comprendre ce qu'est le plan projectif, parlons d'abord de la droite projective
% $\mathbb{P}^{1}$, qui est un espace projectif de dimension 1.

% Si $K$ est le corps des nombres réels, alors la droite projective réelle est obtenue en
% intersectant les droites vectorielles de $\R^2$ avec le cercle unité.

% On a donc l'ensemble des droites vectorielles de $\R^2$ qui sont de la forme $\left[ x,y
% \right] $ avec $y\neq 0$.


% Ceci permet d'obtenir à partir de l'équation initiale, une nouvelle équation qui prolonge la
% courbe initiale à la droite infini. C'est ce qui nous permet de construire un élément neutre
% qui soit bien définie pour la loi du groupe.

% Pour se donner une idée des parties qui compose le plan projectif raisonnons avec les réels.

% Pour $K=\R$, par définition le plan projectif $\mathbb{P}_{2}(\R)$, est l'ensemble quotient de
% tous les vecteurs colinéaire de $\R^3$ privée de l'origine. Ainsi, pour tous vecteurs $(x,y,z)
% \in \mathbb{P}_{2}$, un représentant de classe est de la forme $\left( \lambda x, \lambda y,
% \lambda z \right) $, et on note $\left[ x,y,z \right] $ sa classe d'équivalence. 


% Ainsi, on a les points de la forme $\left[ x,y,0 \right] $ qui forme la droite à l'infini et
% les autres points peuvent tous être écrit de la forme $\left[ x,y,1 \right] $ et il forme le
% plan affine.

% Pour comprendre ce que sont la droite à l'infini et le plan affine, parlons de
% l'homogénéisation des droites.

% Les équations d'une droite dans le plan affine sont de la forme $ax + by + c = 0$, où $a$ et
% $b$ deux réels non tous deux nuls et $c$ un réels quelconque. Ainsi, l'équation homogène
% associé est de la forme $P(x,y,z) = ax + by + cz = 0$, c'est un polynome à plusieurs
% indéterminées dont tous les monômes non nuls sont de même degré total, donc un polynôme
% homogène de degré 1.

% On a alors qu'un point du plan projectif de coordonées homogène $\left[ x,y,z \right] $ est sur
% la droite projective si et seulement si $P(x,y,z) = 0$. Autrement dit, les zéros du polynôme
% homogène sont les points du plan affine dans $\mathbb{P}_{2}$ et ceci indépendament du choix des
% coordonnées homogènes. 

% En effet, comme $\left[ x,y,z \right] = \left\{ (x,y,z) \in
% \R^3 \mid \exists \lambda \in \R^{*},\ (x,y,z) = \lambda(x,y,z) \right\}$, par homogénéité, on
% a
% \[
% P(\lambda x, \lambda y, \lambda z) = \lambda P(x,y,z) 
% \] 
% donc $P(x,y,z)=0$.

% Traditionnelement, on prend pour $z$ la valeur $1$, ainsi un point dans le plan affine est de
% la forme $\left[ x,y,1 \right] $ tel que $P(x,y,1)=P(x,y)$, ce qui montre que le point $\left[
% x,y,1\right] $ est sur la droite projective si et seulement si le point $(x,y)$ est sur la
% droite affine. Ainsi, on a bien opéré un prolongement de la droite affine initial en une droite
% projective.

% Si l'on pose 
% \[
% U_0 = \left\{ [x,y,z] \in \mathbb{P}_{2}(\R) \mid z \neq 0 \right\} 
% ,\] 

% on a l'application $\phi : U_0 \to \R^2$ définie par :
% \[
% \phi([x,y,z]) = \left( \frac{x}{z},\frac{y}{z} \right)
% .\] 
% C'est une bijection dont la réciproque est donnée par :
% \[
% \phi^{-1}(x,y) = [x,y,1]
% .\] 

% Le plan affine dans l'espace projectif est donc une représentation du plan euclidien $\R^2$.
% L'image suivante permet de comprendre ce que l'on obtient.

% Par suite, si $z=0$, on a deux choix qui s'offre à nous à savoir $x \neq 0$ ou $y \neq 0$. Pour
% fixer les idées prennons $y = 1$. Ainsi, un point de la droite projective est de la forme
% $\left[ x,1,0 \right] $ tel que P(x,1,0)=P(x,1). Ainsi, le point $[x,1,0]$ est sur la droite
% projective si et seulement si le point $(x,1)$ est sur la droite affine initial. 




% bijection 

% image

% droite à l'infini

% droite et point les cas

% presente une courbe la canonique

% parle du determinant et de son influence sur la courbe

% parle de la forme normal de W



