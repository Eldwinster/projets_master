\chapter{Le plan projectif $\mathbb{P}(\overline{K})$}

Dans la totalité de ce qui suit la lettre $K$ désignera un corps de caractéristique $0$ ou un corps fini de caractéristique distincte de $2$ et $3$.

On désignera la clôture algébrique de $K$ par la notation $\overline{K}$.

Pour commencer, nous allons construire étape par étape le plan projectif $\mathbb{P}(\overline{K})$, ce qui nous permettra d'en donner une définition formel par la suite.

\section{La droite projective $\mathbb{P}(\R)$}
On commence par construire la droite projective, on se donne alors pour context, l'espace vectoriel $\R^2$. L'idée est de considérer l'ensemble de tous les droites vectorielles de ce dernier. C'est cette ensemble que l'on appelera la droite projective.

En effet, si l'on considère la droite affine d'équation $y=1$, chaque droite vectorielle de $\R^2$ coupe cette droite en un seul point. Sauf bien entendu la droite d'équation $y=0$ qui est parallèle à cette dernière. On peut alors convenir que la droite d'équation $y=0$ "coupe" la droite d'équation $y=1$ en un point à l'infini. On peut alors dire que la droite projective est composé d'une droite affine et d'un point à l'infini.

Plus formellement, on peut énoncer:

\begin{definition}
    Soit l'ensemble $\R^2-\left\{ (0,0) \right\} $ et soit la relation d'équivalence $ \sim$ sur ce plan tel que
    \[
        (x,y) \in\R^2- \left\{ (0,0) \right\} , (x,y) \sim (\lambda x, \lambda y) \text{ avec } \lambda \in \R^{*}
    ,\] 
    cette ensemble muni de sa relation d'équivalence, ainsi que du point à l'infini est un ensemble quotient appelé la droite projective $\mathbb{P}(\R)$.

    On le note $\R^2-\left\{ (0,0) \right\} / \sim$.

    Une classe d'équivalence est appelée un point projectif.
\end{definition}

\begin{remarque}
    Si $y\neq 0$ les point $(x,y)$ on pour représentant de classe les points de la forme $(\frac{x}{y},1)$.

    Si $y=0$, le point $(x,0)$ à pour représentant de classe $(1,0)$ c'est ce point que l'on appelle point à l'infini.
\end{remarque}

\section{Le plan projectif $\mathbb{P}^2(\R)$}

En partant du même principe, pour construire le plan projectif $\mathbb{P}^2(\R)$, on se sert de l'espace vectoriel $\R^3$.

On considère alors le plan affine d'équation $z=1$, ainsi tous les droites vectorielle de $\R^3$ coupe ce plan en un seul point, sauf les droites qui sont dans le plan $(O,x,y)$, où $O$ est l'origine du plan. Ainsi on peut convenir que les droites du plan $(O,x,y)$ coupe le plan $z=1$ en des points à l'infini. 

On a donc le plan projectif qui est formé d'un plan affine et d'une droite projective à l'infini.

On formalise algébriquement comme suit:

\begin{definition}
    Soit l'ensemble $\R^3-\left\{ (0,0,0) \right\} $ et soit la relation d'équivalence $\sim$ sur cet espace défini par
    \[
        (x,y,z) \in \R^3, \quad (x,y,z) \sim (\lambda x, \lambda y, \lambda z) \text{ avec } \lambda \in \R^{*}
    ,\] 
    cette ensemble muni de sa relation d'équivalence et de la droite projective à l'infini est appelé le plan projectif sur le corps $\R$.

    On le note $\mathbb{P}^2(\R)=\R^3-\left\{ (0,0,0) \right\} / \sim$.
\end{definition}

\begin{remarque}
    Si $z\neq_0$, les points (x,y,z) on pour représentant de classe les points de la forme $(\frac{x}{z},\frac{y}{z},1)$.

    Si $z=0$:

    Soit $y\neq 0$ et on prend pour représentant les point de la forme $(\frac{x}{y},1,0)$.

    Soit $y=0$ et on peut choisir comme représentant de classe le point $(1,0,0)$ qui n'est d'autre que le point à l'infini.
\end{remarque}

\section{Le plan projectif $\mathbb{P}(\overline{K})$}

