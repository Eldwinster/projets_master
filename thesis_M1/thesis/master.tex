\documentclass[a4paper]{report}
% Some basic packages
\usepackage[utf8]{inputenc}
\usepackage[T1]{fontenc}
\usepackage{textcomp}
\usepackage[french]{babel}
\usepackage{url}

\usepackage{graphicx}
\graphicspath{ {../images/} }

\usepackage{float}
\usepackage{booktabs}
\usepackage{enumitem}

% \usepackage[a4paper,left=1.5cm,right=1.5cm,top=1.5cm,bottom=1.5cm]{geometry}

\usepackage{hyperref}
\hypersetup{
	colorlinks=true,
	linkcolor=black,
	filecolor=magenta,
	urlcolor=purple,
	% pdftitle={pdftitle}, title of the window
	bookmarks=true,
	pdfpagemode=FullScreen,
}

\pdfminorversion=7

% Don't indent paragraphs, leave some space between them
\usepackage{parskip}

% Hide page number when page is empty
\usepackage{emptypage}
\usepackage{subcaption}
\usepackage{multicol}
\usepackage{xcolor}

% Other font I sometimes use.
% \usepackage{cmbright}

% References stuff
\usepackage[
backend=biber,
style=alphabetic,
sorting=nyt
]{biblatex}

% Math stuff
\usepackage{amsmath, amsfonts, mathtools, amsthm, amssymb, wasysym}
% Fancy script capitals
\usepackage{mathrsfs}
\usepackage{cancel}
% Bold math
\usepackage{bm}
% Some shortcuts
\newcommand\N{\ensuremath{\mathbb{N}}}
\newcommand\R{\ensuremath{\mathbb{R}}}
\newcommand\Z{\ensuremath{\mathbb{Z}}}
\renewcommand\O{\ensuremath{\emptyset}}
\newcommand\Q{\ensuremath{\mathbb{Q}}}
\newcommand\C{\ensuremath{\mathbb{C}}}
\newcommand\K{\ensuremath{\mathbb{K}}}

% Some new math command french made
\newcommand\congru{\ensuremath{\equiv}}
\newcommand\et{\ensuremath{\wedge}}
\newcommand\ou{\ensuremath{\vee}}
% Some new command french made
\newcommand\guillemeg{\guillemotleft}
\newcommand\guillemed{\guillemotright}

% New math operator french made
\DeclareMathOperator{\pgcd}{pgcd}
\DeclareMathOperator{\ppcm}{ppcm}
\DeclareMathOperator{\card}{Card}
\DeclareMathOperator{\car}{car}
\DeclareMathOperator{\trace}{trace}
\DeclareMathOperator{\id}{Id}

% Standard math operator
\DeclareMathOperator{\dx}{dx}
\DeclareMathOperator{\dt}{dt}
\DeclareMathOperator{\ds}{ds}

% Easily typeset systems of equations (French package)
\usepackage{systeme}

% Put x \to \infty below \lim
\let\svlim\lim\def\lim{\svlim\limits}

%Make implies and impliedby shorter
\let\implies\Rightarrow
\let\impliedby\Leftarrow
\let\iff\Leftrightarrow
\let\epsilon\varepsilon
\let\phi\varphi
\let\vec\overrightarrow

% Add \contra symbol to denote contradiction
\usepackage{stmaryrd} % for \lightning
\newcommand\contra{\scalebox{1.5}{$\lightning$}}

% Command for short corrections
% Usage: 1+1=\correct{3}{2}

\definecolor{correct}{HTML}{009900}
\newcommand\correct[2]{\ensuremath{\:}{\color{red}{#1}}\ensuremath{\to }{\color{correct}{#2}}\ensuremath{\:}}
\newcommand\green[1]{{\color{correct}{#1}}}

% horizontal rule
\newcommand\hr{
    \noindent\rule[0.5ex]{\linewidth}{0.5pt}
}

% hide parts
\newcommand\hide[1]{}

% si unitx
\usepackage{siunitx}
\sisetup{locale = FR}

% Environments
\makeatother
% For box around Definition, Theorem, \ldots
\usepackage{mdframed}
\mdfsetup{skipabove=1em,skipbelow=0em}
\theoremstyle{definition}
% \newmdtheoremenv[nobreak=true]{definition}{Définition}
\newtheorem{theoreme}{Théorème}[chapter]
\newtheorem{definition}[theoreme]{Définition}
\newtheorem{proposition}[theoreme]{Proposition}
\newtheorem{propriete}[theoreme]{Propriété}
\newtheorem{lemme}[theoreme]{Lemme}
\newtheorem{corollaire}[theoreme]{Corollaire}
\newtheorem{defthm}[theoreme]{Définition et théorème}

\newtheorem{exo}{Exercice}

% without numbering
\newtheorem*{demonstration}{Démonstration}
\newtheorem*{preuve}{Preuve}
\newtheorem*{but}{But}
\newtheorem*{consequence}{Conséquence}
\newtheorem*{rappel}{Rappel}
\newtheorem*{regle}{Règle de calcul}
\newtheorem*{postulat}{Postulat}
\newtheorem*{conclusion}{Conclusion}
\newtheorem*{bonus}{Bonus}
\newtheorem*{conjecture}{Conjecture}
\newtheorem*{recurrence}{Récurrence}
\newtheorem*{intermede}{Intermède}
\newtheorem*{observation}{Observation}
\newtheorem*{application}{Application}
\newtheorem*{probleme}{Problème}
\newtheorem*{terminologie}{Terminologie}
\newtheorem*{question}{Question}
\newtheorem*{exemple}{Exemple}
\newtheorem*{contrex}{Contre-Exemple}
\newtheorem*{notation}{Notation}
\newtheorem*{vuavant}{Comme vu précédemment}
\newtheorem*{remarque}{Remarque}
\newtheorem*{intuition}{Intuition}
\newtheorem*{motivation}{Motivation}
\newtheorem*{resume}{Résumé}

% End example and intermezzo environments with a small diamond (just like proof
% environments end with a small square)
\usepackage{etoolbox}
\AtEndEnvironment{exemple}{\null\hfill$\diamond$}%
\AtEndEnvironment{intermede}{\null\hfill$\diamond$}%
\AtEndEnvironment{demonstration}{\null\hfill$\square$} 
\AtEndEnvironment{preuve}{\null\hfill$\square$} 

% Fix some spacing
% http://tex.stackexchange.com/questions/22119/how-can-i-change-the-spacing-before-theorems-with-amsthm
\makeatletter
\def\thm@space@setup{%
  \thm@preskip=\parskip \thm@postskip=0pt
}


% Exercise 
% Usage:
% \exercice{5}
% \subexercice{1}
% \subexercice{2}
% \subexercice{3}
% gives
% Exercice 5
%   Exercice 5.1
%   Exercice 5.2
%   Exercice 5.3
\newcommand{\exercice}[1]{%
    \def\@exercice{#1}%
    \section{Exercice #1}
}

\newcommand{\subex}[1]{%
    \subsection{Exercice #1}
}

% \lecture starts a new lecture (les in dutch)
%
% Usage:
% \lecture{1}{di 12 feb 2019 16:00}{Inleiding}
%
% This adds a section heading with the number / title of the lecture and a
% margin paragraph with the date.

% I use \dateparts here to hide the year (2019). This way, I can easily parse
% the date of each lecture unambiguously while still having a human-friendly
% short format printed to the pdf.

\usepackage{xifthen}
\def\testdateparts#1{\dateparts#1\relax}
\def\dateparts#1 #2 #3 #4 #5\relax{
    \marginpar{\small\textsf{\mbox{#1 #2 #3 #5}}}
}

\def\@lecture{}%
\newcommand{\lecture}[3]{
    \ifthenelse{\isempty{#3}}{%
        \def\@lecture{Lecture #1}%
    }{%
        \def\@lecture{Lecture #1: #3}%
    }%
    \subsection*{\@lecture}
    \marginpar{\small\textsf{\mbox{#2}}}
}

\def\@td{}%
\newcommand{\td}[3]{
    \ifthenelse{\isempty{#3}}{%
        \def\@td{Td #1}%
    }{%
        \def\@td{Td #1: #3}%
    }%
    \subsection*{\@td}
    \marginpar{\small\textsf{\mbox{#2}}}
}


% These are the fancy headers
\usepackage{fancyhdr}
\pagestyle{fancy}

% \fancyhead[RO,LE]{\@lecture} % Right odd,  Left even
% \fancyhead[RO,LE]{\@td} % Right odd,  Left even
\fancyhead[L]{\leftmark}
% \fancyhead[R]{\rightmark}
\fancyhead[C,R]{}

\fancyfoot[L]{Yann-Arby BEBBA}
\fancyfoot[C]{\thepage}
\fancyfoot[R]{Mémoire master 1}

\makeatother




% Todonotes and inline notes in fancy boxes
\usepackage{todonotes}
\usepackage{tcolorbox}

% Make boxes breakable
\tcbuselibrary{breakable}

% Verbetering is correction in Dutch
% Usage: 
% \begin{verbetering}
%     Lorem ipsum dolor sit amet, consetetur sadipscing elitr, sed diam nonumy eirmod
%     tempor invidunt ut labore et dolore magna aliquyam erat, sed diam voluptua. At
%     vero eos et accusam et justo duo dolores et ea rebum. Stet clita kasd gubergren,
%     no sea takimata sanctus est Lorem ipsum dolor sit amet.
% \end{verbetering}
\newenvironment{correction}{\begin{tcolorbox}[
    arc=0mm,
    colback=white,
    colframe=green!60!black,
    title=Correction,
    fonttitle=\sffamily,
    breakable
]}{\end{tcolorbox}}

% Noot is note in Dutch. Same as 'verbetering' but color of box is different
\newenvironment{note}[1]{\begin{tcolorbox}[
    arc=0mm,
    colback=white,
    colframe=red!60!black,
    title=#1,
    fonttitle=\sffamily,
    breakable
]}{\end{tcolorbox}}




% Figure support as explained in my blog post.
\usepackage{import}
\usepackage{xifthen}
\usepackage{pdfpages}
\usepackage{transparent}
\newcommand{\incfig}[1]{%
    \def\svgwidth{\columnwidth}
    \import{./figures/}{#1.pdf_tex}
}

% Fix some stuff
% %http://tex.stackexchange.com/questions/76273/multiple-pdfs-with-page-group-included-in-a-single-page-warning
\pdfsuppresswarningpagegroup=1


% My name
% \author{Yann-Arby BEBBA}

% title page layout from
% https://latexref.xyz/titlepage.html
\begin{titlepage}
\vspace*{\stretch{1}}
\begin{center}
  % {\huge\bfseries Mémoire \\[1ex] 
  %                 Le groupe des courbes elliptiques et applications dans la
  %             cryptographie}                  \\[6.5ex]
  {\huge\bfseries Mémoire \\[1ex] 
                  Courbes elliptiques \\
              et \\ [1ex]
          cryptographie}                  \\[6.5ex]
  {\large\bfseries Bebba Yann-Arby}           \\
  \vspace{4ex}
  Mémoire rendu à                    \\[5pt]
  \textit{l'Université Picardie Jules Verne}                \\[1ex]
  dirigé par Mme R.Abdelatif \\ [2cm]
  dans le cadre de la première année de  \\[1ex]
  \textsc{\Large Master Mathématiques}    \\[6ex]
  % \textsc{\large Mathématique}             \\[6ex]
  \includegraphics[scale=0.1]{logoUPJV_Bleu}
  \vfill
  Département de Mathématiques               \\
  LAMFA \\
  CNRS 33 Rue Saint-Leu, 80000 Amiens, France                                 \\
  \today \\
  \vfill
\end{center}
\vspace{\stretch{2}}
\end{titlepage}

\addbibresource{../library.bib}
% \title{Mémoire}
\begin{document}
    \tableofcontents
    \newpage
    \begin{center}
        \textbf{Résumé}

        Dans ce mémoire, je m'intéresse essentiellement à la construction du groupe des points
        rationnels d'une courbe elliptique en vue d'en donner deux applications. 

        Je commence
        par introduire historiquement le sujet, puis je donne le contexte mathématique autour des
        courbes elliptiques. Ensuite, j'introduis les différentes définitions qui vont
        m'être utiles pour construire le groupe des points rationnels d'une courbe
        elliptique. Ces différentes définitions me permettent
        ainsi d'effectuer la construction géométrique et algébrique de la loi de groupe. Finalement, je donne deux
        systèmes cryptographiques appliqué au groupe que l'on a construit.
    \end{center}
    % start lectures
    \chapter{Les courbes elliptiques: histoire et liens avec la cryptographie}
le cryptosystème RSA, inventé par Ronald \textbf{R}ivet, Adi
\textbf{S}hamir et Leonard \textbf{A}dleman en 1977
\section{Introduction}

Dans ce mémoire, nous allons nous intéresser aux courbes elliptiques et plus particulièrement
au groupe abélien des courbes elliptiques. Ce qui va nous permettre d'utiliser ce groupe en
cryptographie et présenter une façon plus efficace d'utiliser l'algorithme d'El-Gamal (EG.) (date) et le
protocol de Diffie-Hellman (D-H.) (date).

Dans un monde en constant évolution, notamment technique. Il est crucial de pouvoir
améliorer, réinventer, ou même changer, des principes qui ont révolutionner à leur époque.
C'est pourquoi, en (date) Klobnitz, à présenter une façon concrète d'utiliser les courbes
elliptiques dans le cadre de la cryptographie. Ceci a permit d'apporter une nouvelle façon
de faire de la cryptographie, tout en conservant des concepts eprouvé basé sur le problème
du logarithme discret. Cette nouvelle approche, que l'on nommera
version elliptique, contrairement à la version dite classique de chiffrement, qui est basé
sur le groupe multiplicatif $\left( \Z /p\Z \right) ^{*}$ et sa commutativité, celle ci
présente une diversité et complexité non négligeable. 

En effet, que ce soit l'algorithme d'El Gamal ou le protocol de Diffie-Hellman, leur version
classique est basé sur le générateur du groupe $\left( \Z / p\Z \right) ^{*}$, alors que
leur version elliptique est basé sur les courbes elliptiques qui comme on le verra sont en grand
nombre pour leur part. De plus, par construction du groupe des courbes elliptiques, plus
abstrait, la résolution du logarithme discret est quasiment impossible sans l'aide d'ordinateur
quantique extremement puissant (ref article Mme Abdelatif).

\section{plan ?}

\begin{itemize}
    \item explication cryptographie
        \begin{itemize}
            \item sym et asym
            \item rsa
            \item El Gamal et D-H
        \end{itemize}
    \item histoire des fonctions elliptiques 
\end{itemize}

\section{La cryptographie}
L'application première de notre construction étant la cryptographie, il me semble nécessaire de
poser les bases de cette branche des mathématiques. Ceci nous permettra d'avoir une idée clair
des différents concepts et enjeux qui la compose.

Tout d'abord, définition ce qu'est la cryptographie. Métaphysiquement, c'est le fait de vouloir
communiquer des messages, entre diverses entités et ceci de façon à ce que seul ces dernières
n'aient connaissances du contenue du message.

Cette définition personnel et trivial est basé sur comment dans l'histoire la cryptographie est
apparu.

De nos jours, le concept c'est énormément diversifié. La transmission de message reste un élément
majeur de ce qu'est la cryptographie mais 

    \chapter{Le plan projectif $\mathbb{P}^2(\overline{K})$}

La définition que nous allons utilisé pour les courbes elliptiques étant dans le plan
projectif.

Introduisons brièvement, ce qu'est un espace projectif, ainsi que les objets dont nous aurons
besoin à savoir des points et des droites.

Intuitivement, un espace projectif permet de rendre homogène un espace vectoriel. On entend
par là, de raisonner indépendammenent des proportionalités pour ne plus considérer que les
directions. L'idée nous vient de la formalisation mathématique de la perspective. 
L'espace projectif nous permet d'identifier des droites à des points. Ce qui rend possible
le fait de raisonner en termes de coordonnées et de pouvoir effectuer des calculs formel.

Dans un premier temps pour comprendre les concepts liées aux espaces projectif. Partons de la
droite projective qui est un espace projectif de dimension 1.

\begin{definition}
    La droite projective sur $\overline{K}$, que l'on note $\mathbb{P}^1(\overline{K})$ ou
    $\mathbb{P}^1$, est l'ensemble quotient
    \[
    \overline{K}^2 - \left\{ (0,0) \right\} / \sim
    \] 
    où $\sim$ est la relation d'équivalence telle que pour tous $\left( x,y \right) $ et
    $\left( x',y'\right) $ non nuls de $\overline{K}^2$,

    \[
    \left( x,y \right) \sim \left( x',y' \right) \iff \exists \lambda \in
    \overline{K}^{*} \quad \left( x',y' \right) = \lambda \left( x,y \right) 
    .\] 
    Pour tous $\left( x,y \right) $ non nuls dans $\overline{K}^2$, on note $\left[ x,y
    \right] $ sa classe d'équivalence appelée coordonnées homogènes.
\end{definition}

Un point de la droite projective est donc définie par les droites vectorielles privée de
l'origine.

On a alors deux types de point, les points de la forme $\left[ x,1 \right] $ et ceux de
la forme $\left[ x,0 \right] $.

Pour le premier type, comme $y \neq 0$ on a naturellement $\lambda = \frac{1}{y}$. On
obtient alors l'intersection de tous les droites vectorielles avec la droite affine $y=1$, ceux
qui forme la droite sur $K$.

Le deuxième type correspond à l'ensemble des droites affines parallèle à l'ordonné, de plus
pour tous $x \neq 0$, on a $\lambda = \frac{1}{x}$, donc peut importe la valeur de $x$ les
droites s'intersectent au point $[1,0]$.

Intuivement, la droite projective sur $K$ est une droite affine sur $K$ complétée par un
point, appelé point à l'infini que l'on note $\mathcal{O}$. Ainsi dans l'espace projectif deux droites parallèles s'intersectent en
un point à l'infini. Quand on parle de droite à l'infini, on désigne par cette droite
l'ensemble des points à l'infini.

Le point à l'infini étant l'intersection des droites paralléles 
% On obtient donc l'ensemble $\mathbb{P}^1 = K \cup \mathcal{O}$, avec $\mathcal{O} = \left[ 1,0 \right] $, le point à
% l'infini. 

\begin{definition}
    Le plan projectif sur $\overline{K}$, que l'on note $\mathbb{P}^2(\overline{K})$ ou
    $\mathbb{P}^2$, est l'ensemble quotient
    \[
    \overline{K}^3 - \left\{ (0,0,0) \right\} / \sim
    \] 
    où $\sim$ est la relation d'équivalence telle que pour tous $\left( x,y,z \right) $ et
    $\left( x',y',z'\right) $ non nuls de $\overline{K}^3$,

    \[
    \left( x,y,z \right) \sim \left( x',y',z' \right) \iff \exists \lambda \in
    \overline{K}^{*} \quad \left( x',y',z' \right) = \lambda \left( x,y,z \right) 
    .\] 
\end{definition}

Donnons également, la définition de la droite projective espace projectif de dimension 1,



L'espace projectif permet de formaliser et généraliser à toute dimension la notion de
droite à l'infini dans le plan projectif.

Le plan projectif permet d'introduire la notion d'homogénéisation des équations de courbes algébrique.
Ce procédé permet à partir de l'équation initial d'une courbe algébrique du plan usuel,
admettant une équation de la forme $P(x,y)=0$ où $P \in K[X,Y]$, d'obtenir une équation d'une courbe qui
est dans le plan projectif, et donc de prolonge la courbe initiale à la droite à
l'infini. Autrement dit, on définit un plan affine en choisissant une droite projective
quelconque associée à ce plan, qui est la droite à l'infini 

Pour bien comprendre ce qu'est le plan projectif, parlons d'abord de la droite projective
$\mathbb{P}^{1}$, qui est un espace projectif de dimension 1.

Si $K$ est le corps des nombres réels, alors la droite projective réelle est obtenue en
intersectant les droites vectorielles de $\R^2$ avec le cercle unité.

On a donc l'ensemble des droites vectorielles de $\R^2$ qui sont de la forme $\left[ x,y
\right] $ avec $y\neq 0$.


Ceci permet d'obtenir à partir de l'équation initiale, une nouvelle équation qui prolonge la
courbe initiale à la droite infini. C'est ce qui nous permet de construire un élément neutre
qui soit bien définie pour la loi du groupe.

Pour se donner une idée des parties qui compose le plan projectif raisonnons avec les réels.

Pour $K=\R$, par définition le plan projectif $\mathbb{P}^2(\R)$, est l'ensemble quotient de
tous les vecteurs colinéaire de $\R^3$ privée de l'origine. Ainsi, pour tous vecteurs $(x,y,z)
\in \mathbb{P}^2$, un représentant de classe est de la forme $\left( \lambda x, \lambda y,
\lambda z \right) $, et on note $\left[ x,y,z \right] $ sa classe d'équivalence. 


Ainsi, on a les points de la forme $\left[ x,y,0 \right] $ qui forme la droite à l'infini et
les autres points peuvent tous être écrit de la forme $\left[ x,y,1 \right] $ et il forme le
plan affine.

Pour comprendre ce que sont la droite à l'infini et le plan affine, parlons de
l'homogénéisation des droites.

Les équations d'une droite dans le plan affine sont de la forme $ax + by + c = 0$, où $a$ et
$b$ deux réels non tous deux nuls et $c$ un réels quelconque. Ainsi, l'équation homogène
associé est de la forme $P(x,y,z) = ax + by + cz = 0$, c'est un polynome à plusieurs
indéterminées dont tous les monômes non nuls sont de même degré total, donc un polynôme
homogène de degré 1.

On a alors qu'un point du plan projectif de coordonées homogène $\left[ x,y,z \right] $ est sur
la droite projective si et seulement si $P(x,y,z) = 0$. Autrement dit, les zéros du polynôme
homogène sont les points du plan affine dans $\mathbb{P}^2$ et ceci indépendament du choix des
coordonnées homogènes. 

En effet, comme $\left[ x,y,z \right] = \left\{ (x,y,z) \in
\R^3 \mid \exists \lambda \in \R^{*},\ (x,y,z) = \lambda(x,y,z) \right\}$, par homogénéité, on
a
\[
P(\lambda x, \lambda y, \lambda z) = \lambda P(x,y,z) 
\] 
donc $P(x,y,z)=0$.

Traditionnelement, on prend pour $z$ la valeur $1$, ainsi un point dans le plan affine est de
la forme $\left[ x,y,1 \right] $ tel que $P(x,y,1)=P(x,y)$, ce qui montre que le point $\left[
x,y,1\right] $ est sur la droite projective si et seulement si le point $(x,y)$ est sur la
droite affine. Ainsi, on a bien opéré un prolongement de la droite affine initial en une droite
projective.

Si l'on pose 
\[
U_0 = \left\{ [x,y,z] \in \mathbb{P}^2(\R) \mid z \neq 0 \right\} 
,\] 

on a l'application $\phi : U_0 \to \R^2$ définie par :
\[
\phi([x,y,z]) = \left( \frac{x}{z},\frac{y}{z} \right)
.\] 
C'est une bijection dont la réciproque est donnée par :
\[
\phi^{-1}(x,y) = [x,y,1]
.\] 

Le plan affine dans l'espace projectif est donc une représentation du plan euclidien $\R^2$.
L'image suivante permet de comprendre ce que l'on obtient.

Par suite, si $z=0$, on a deux choix qui s'offre à nous à savoir $x \neq 0$ ou $y \neq 0$. Pour
fixer les idées prennons $y = 1$. Ainsi, un point de la droite projective est de la forme
$\left[ x,1,0 \right] $ tel que P(x,1,0)=P(x,1). Ainsi, le point $[x,1,0]$ est sur la droite
projective si et seulement si le point $(x,1)$ est sur la droite affine initial. 




bijection 

image

droite à l'infini

droite et point les cas

presente une courbe la canonique

parle du determinant et de son influence sur la courbe

parle de la forme normal de W




    \chapter{Définitions générales}

\begin{definition}
    \label{def:ell}
    Une courbe elliptique définie sur $K$ est une courbe projective plane d'équation
    \begin{align}
        \label{eq:ell}
    y^2z=x^3+axz^3+bz^2
    .\end{align}
    où $a$ et $b$ sont des éléments de $K$ vérifiant la condition
    \begin{align}
        \label{eq:delta}
    4a^3+27b^2\neq 0
    .\end{align}
\end{definition}

\begin{lemme}
    \label{lem:lemme1}
    Soit $\Delta= -(4a^3 + 27b^2)$ le discriminant $f = x^3 + ax + b$. 
    Les racines de $f$ sont simples, si et seulement si $\Delta \neq 0$.
\end{lemme}

\begin{demonstration}
    Montrons tout d'abord que le discriminant de $f$ est $\Delta= -(4a^3 + 27b^2)$.

    Soit $\Delta$ le discriminant de $f$. Soient $\alpha, \beta, \gamma $ les racines de $f$ dans $\overline{K}$ et $f'$ le polynôme dérivé de $f$.

    Tout d'abord montrons que :
    \begin{align*}
        \Delta &= (-1) ( \alpha - \beta )^2 ( \alpha - \gamma )^2 ( \beta - \gamma )^2 \\
          &= - f(\alpha)'f(\beta )'f(\gamma)'
    .\end{align*}
    D'après le théorème de d'Alembert on peut écrire $f$ sous la forme :
    \[
        f = \left( X - \alpha \right) \left( X - \beta \right) \left( X - \gamma \right) 
    .\] 
    En dérivant $f$ sous cette forme on obtient :
    \begin{align*}
        f &= ( X - \alpha ) \left( ( X - \beta ) ( X - \gamma ) \right)'  + ( X - \beta ) ( X - \gamma )\\
          &= ( X - \alpha ) \left( ( X - \beta ) + ( X - \gamma ) \right) + ( X - \beta ) ( X - \gamma ) 
    .\end{align*}

    Donc  
\[
f' = ( X - \alpha ) ( X - \beta ) + ( X - \alpha ) ( X - \gamma ) + ( X - \beta ) ( X - \gamma )
.\] 

On a alors successivement : 
\[
    f(\alpha)' = ( \alpha - \beta) ( \alpha - \gamma )
,\] 
\[
f(\beta )' = ( \beta - \alpha) ( \beta - \gamma)
,\] 
et
\[
f(\gamma)' = ( \gamma - \alpha) ( \gamma - \beta)
.\] 

En multipliant ces trois expressions, on obtient :
\begin{align*}
    f(\alpha)' f(\beta )' f(\gamma)' &= ( \alpha - \beta ) ( \alpha - \gamma ) ( \beta - \alpha ) ( \beta - \gamma) ( \gamma - \alpha ) ( \gamma - \beta ) \\
&= \left( X - \alpha \right) \left( \alpha - \gamma \right) \left( -1 \right) \left( \alpha - \beta  \right) \left( \beta - \gamma \right) \left( -1 \right) \left( \alpha - \gamma \right) \left( -1 \right) \left( \beta - \gamma \right) \\
&= \left( -1 \right) ^3 \left( \alpha - \beta  \right) ^2 \left( \alpha - \gamma  \right) ^2 \left( \beta - \gamma \right) ^2\\
 &= - \Delta
.\end{align*}

Et donc 

\[
\Delta = - f'(\alpha) f'(\beta ) f'(\gamma)
.\] 

En partant de la forme $f : x^3 + ax + b$, on remarque que $f' : 3x^2 + a$. Par suite on obtient,
\begin{align*}
    \Delta &= - f'(\alpha) f'(\beta ) f'(\gamma) \\
      &= - \left( 3 \alpha^2 + a \right) \left( 3 \beta^2 + a \right) \left( 3 \gamma^2 + a \right) 
.\end{align*}
Ce qui donne :
\begin{align*}
    \Delta  &= - \left( ( 9 \alpha^2 \beta^2 + 3a ( \alpha^2 + \beta^2 ) + a^2 ) ( 3 \gamma^2 + a ) \right)  \\
       &= - \left( 27 \left( \alpha \beta \gamma \right)^2  + 9a \left( \alpha^2 \beta^2 + \alpha^2 \gamma^2 + \beta^2 \gamma^2 \right) + 3a^2 \left( \alpha^2 + \beta^2 + \gamma^2 \right) + a^3 \right) 
.\end{align*}
On peut écrire
\[
\alpha^2 + \beta^2 + \gamma^2 = \left( \alpha + \beta + \gamma \right)^2 - 2 \left( \alpha \beta + \alpha \gamma + \beta \gamma \right)
,\] 
\[
\alpha^2 \beta^2 + \alpha^2 \gamma^2 + \beta^2 \gamma^2 = \left( \alpha \beta + \alpha \gamma + \beta \gamma \right)^2 - 2\alpha \beta \gamma \left( \alpha + \beta + \gamma \right)
.\] 
Donc d'après les relations entre coefficients et racine (i.e relation de Viète), pour un polynôme de la forme $ax^3 + bx^2 + cx + d$, on a :
\[
\alpha + \beta + \gamma = - \frac{b}{a}
,\] 
\[
\alpha \beta + \alpha \gamma + \beta \gamma = \frac{c}{a}
,\] 
\[
\alpha \beta \gamma = - \frac{d}{a}
.\] 
Donc pour $f$ on a $a = 1$, $b = 0$, $c = a$ et $d = b$.

D'où,
\[
\alpha + \beta + \gamma = 0 \text{, } \alpha \beta + \alpha \gamma + \beta \gamma = a \text{ et } \alpha \beta \gamma = - b
.\] 
Ce qui donne : 
\begin{align*}
    \alpha^2 + \beta^2 + \gamma^2 &= 0^2 - 2a = -2a \\
    \alpha^2 \beta^2 + \alpha^2 \gamma^2 + \beta^2 \gamma^2 &= a^2 + 2b \times 0
.\end{align*}

Donc le discriminant vaut :
\begin{align*}
    \Delta &= - \left( 27b^2 + 9a^3 - 6a^3 + a^3  \right) \\
        &= - \left( 4a^3 + 27b^2  \right)
.\end{align*}

Maintenant, supposons que $\Delta = 0$. On a alors :
 \begin{align*}
     - \left( 4a^3 + 27^2 \right) = 0 &\iff - f(\alpha)' f(\beta )' f(\gamma)' = 0 \\
                                      & \iff \left( f(\alpha)' = 0 \right) \ou \left( f(\beta )' = 0 \right) \ou \left( f(\gamma)' = 0 \right) \\
                                      &\iff \alpha \text{ ou } \beta \text{ ou } \gamma \text{ est une racine multiple}
.\end{align*}

D'où le résultat.
\end{demonstration}

\section{Partie affine et point à l'infini}

Si $(X,Y,Z)$ un point de $\overline{K}$ avec $Z\neq 0$ vérifie $F(X,Y,Z)=0$ alors de façon équivalente $(x,y)$ un point de $\overline{K}$ vérifie $f(x,y)=0$.

En effet, il suffit dans l'équation $y^2=x^3+ax+b$ de remplacer $x$ par $\frac{X}{Z}$ et $y$ par $\frac{Y}{Z}$ et de multiplier par la puissance de $Z$ qu'il faut pour obtenir l'équation polynômiale en terme de $(X,Y,Z)$.

Formalisons ce qu'est le point à l'infini, ainsi que ce qui vient d'être dit.

Posons 
\[
U = \left\{ [x,y,z] \in \mathbb{P}^2(\overline{K}) \mid z \neq 0 \right\} 
.\] 

On dispose de l'application $\phi : U \to \overline{K}^2$ définie par
\[
\phi([x,y,z])=\left( \frac{x}{z},\frac{y}{z} \right) 
.\] 

C'est une bijection, dont l'application réciproque est donnée par la formule
\[
\phi^{-1}(x,y)=\left[ x,y,1 \right] 
.\] 

Considérons des éléments $a$ et $b$ de $K$ tels que $4a^3+27b^2 \neq 0$. Soit $E$ la courbe elliptique définie sur $K$ d'équation 
\[
y^2z=x^3+axz^2+bz^3
.\] 

L'ensemble des points $[x,y,z] \in E$ tels que $z=0$ est réduit au singleton $\left\{ O \right\} $ où
\[
O = [0,1,0]
.\] 

En effet, dans l'équation \eqref{eq:ell}, il vient
\[
y^2\times 0 = x^3 + ax \times 0^2 + b \times 0^3
,\] 
donc $x=0$, ainsi on peut prendre pour représentant de classe de cet élément la classe de $O$.

Par ailleurs, $E \cap U$ s'identifie via $\phi$ à l'ensemble des éléments $(x,y)$ de $\overline{K}^2$ vérifiant l'égalité

\begin{align}
    \label{eq:ell1}
y^2 = x^3 + ax + b
.\end{align}

On dira que $E \cap U$ est la partie affine de $E$ et que $O$ est le point à la l'infini de $E$.

Dans toute la suite, on identifira $E \cap U$ et le sous-ensemble de $\overline{K}^2$ formé des éléments $(x,y)$ vérifiant \eqref{eq:ell1}. Avec cette identification, on a 
\[
E = \left\{ (x,y) \in \overline{K} \times \overline{K} \mid y^2=x^3+ax+b \right\} \cup \left\{ O \right\} 
.\] 

Ainsi, $E$ est la courbe affine d'équation \eqref{eq:ell1} à laquelle on adjoint le point à l'infini $O$. C'est pourquoi on définira souvent une courbe elliptique par sa partie affine, sans préciser le point $O$. Cela étant, il ne faudra pas perdre de vue l'importance du point à l'infini, comme on s'en rendra compte notamment dans la définition de la loi de groupe sur $E$ que l'on verra plus loin.

\begin{remarque}
    On retiendra qu'une courbe affine d'équation de la forme \eqref{eq:ell1} est une courbe
    elliptique si et seulement si, par définition, la condition \eqref{eq:delta} est satisfaite.
\end{remarque}

\section{Points rationnels d'une courbe elliptique}
Comme on va le voir par la suite par la construction d'une loi de composition interne sur l'ensemble des
courbes elliptique, dans un premier temps, et par la construction de la loi de groupe
sur les courbes elliptique, dans un second temps.

Les points construits à l'aide de ces deux
loi bien qu'au départ peuvent être des entiers deviennent, par construction trés vite des
rationnels. C'est pourquoi, il est important de définir que ces points sont toujours des
points de la courbe.

\textbf{(détail un peu la suite en reprennant les explications de Mme Abdelatif)}

Soit $L$ une extension de $K$ dans $\overline{K}$.

\begin{definition}
    Soit $P=\left[ x,y,z \right] $ un point de $\mathbb{P}^2$. On dit que $P$ est rationnel sur $L$ s'il existe $\lambda \in \overline{K}^{*}$ tel que $\lambda x$, $\lambda y$ et $\lambda z$ soient dans $L$. On note $\mathbb{P}^2(L)$ l'ensemble des points de $\mathbb{P}^2$ rationnels sur $L$.

    Cela justifie la notation $\mathbb{P}^2 = \mathbb{P}^2(\overline{K})$.
\end{definition}

\begin{remarque}
    Étant donné un point $[x_1,x_2,x_3] \in \mathbb{P}^2$, le fait qu'il soit rationnel sur $L$ n'implique pas que les $x_{i}$ soient dans $L$. Cela signifie qu'il existe $i$ tel que $x_{i}$ soit non nul, et que chaque $\frac{x_{j}}{x_{i}}$ appartienne à $L$.
\end{remarque}

Soit $E$ une courbe elliptique définie sur $K$ d'équation \eqref{eq:ell}.

\begin{definition}
    Un point de $E$ est dit rationnel sur $L$ s'il appartient à $E \cap \mathbb{P}^2(L)$. On note $E(L)$ l'ensemble des points de $E$ rationnels sur $L$.

    Par définition, on a donc
    \[
    E = E(\overline{K})
    .\] 
\end{definition}

Le point $O = [0,1,0]$ appartient à $E(K)$. Soit $(x,y) \in \overline{K}^2$ un point de la partie affine de $E$. Par définition, il est rationnel sur $L$ si et seulement si $x$ et $y$ sont dans $L$. Il en résulte que l'on a 
\[
E = \left\{ (x,y) \in L \times L \mid y^2 = x^3+ax+b \right\} \cup \left\{ O \right\} 
.\] 

\begin{exemple}
    mettre l'exemple avec la courbe et le corps fini pour les exemples des cryptosystèmes
\end{exemple}

    \chapter{Loi de groupe}
% \section{Loi de groupe}

Soit $E$ une courbe elliptique définie sur $K$. Pour toute extension $L$ de $K$ dans $\overline{K}$, on va munir $E(L)$ d'une structure naturelle de groupe abélien, d'élément neutre le point à l'infini.

\section{Point de vue géométrique}

\begin{proposition}
    \label{prop:secTanGeo}
    
    Soient une cubique irréductible non singulière $E$ et une droite $L$ définies sur
    $K$. Si la cubique $E$ a au moins deux points d'intersection (comptés avec leur
    multiplicité) avec la droite $L$, alors le nombre de points d'intersection (comptés avec
    leur multiplicité) entre $E$ et $L$ est exactement $3$.
\end{proposition}

\begin{demonstration}
   En effet, comme $E$ est irréductible, nous savons grâce à la proposition \textbf{ref} que le
   nombre de points à l'intersection de $E$ et $D$ est fini. Soit la droite $D : aX + bY + cZ =
   0$ où nous supposons $c \neq 0$. Les points $P = (X,Y,Z)$ sont racines du polynôme $F(X,Y,-
   \frac{aX+bY}{c})$ où $F$ est le polynôme homogène de degré 3 qui définit $E$. 

   Notons :
   \[
   q(X,Y)=F(X,Y,- \frac{aX + bY}{c})
   ,\] 
   et soient $P=(x_{P},y_{P},z_{P})$ et $Q=(x_{Q},y_{Q},z_{Q})$ deux point, non nécessairement
   distinct, à l'intersection
   entre $E$ et $D$. Comme $q(x_{P},y_{P}) = q(x_{Q},y_{Q})=0$, on peut écrire : 
   \[
   q(X,Y)=v(X,Y)(y_{P}X-x_{P}Y)(y_{Q}X-x_{Q}Y)
   ,\] 
   où $v$ est un polynôme homogène de degré 1. Il n'a donc qu'une racine que nous noterons
   $(x_{R},y_{R})$. Le point $R=(x_{R},y_{R},- \frac{ax_{R}+by_{Q}}{c})$ est alors le troisième
   point de l'intersection entre $E$ et $D$.
\end{demonstration}

\section{Droites de $\mathbb{P}^2$}
Le première objet dont l'on a besoin pour construire notre groupe et que le l'on va
manipuler tout on l'on du processus est la droite projective.

\begin{definition}
    Une droite de $\mathbb{P}^2$ est une partie de $\mathbb{P}^2$ formée des points $[x,y,z]$ tels que 
    \[
    D :\ ux+vy+wz=0
    ,\] 
    où $u$, $v$ et $w$ sont des éléments non tous nuls de $\overline{K}$.
\end{definition}

On parle alors de la droite d'équation $ux+vy+wz=0$. Une droite d'équation $x=\lambda z$, où $\lambda$ est dans $\overline{K}$, est dite verticale. Une telle droite passe par le point $O = [0,1,0]$. En fait, toute droite passant par $O$ a une équation de la forme $ux+wz=0$. On dit souvent que la droite d'équation $z=0$ est la droite à l'infini. En identifiant la partie de $\mathbb{P}^2$ formée des points $[x,y,z]$ tels que $z \neq 0$ avec $\overline{K}^2$, le plan projectif s'interprète
comme la réunion de $\overline{K}^2$ avec la droite à l'infini.

\begin{lemme}
    \label{lem:lemme2}
    
    % Soient $P = \left[ a_1, a_2, a_3 \right]$ et $Q = \left[ b_1, b_2, b_3 \right]$ deux points distincts de $\mathbb{P}^2$.

    % Il existe une unique droite de $\mathbb{P}^2$ passant par $P$ et $Q$. C'est la droite $D$ d'équation $ux + vy + wz = 0$ avec $\left[ x, y, z \right] \in \mathbb{P}^2$ et 
    % \[
    % u = a_2 b_3 - a_3 b_2, \ v = a_3 b_1 - a_1 b_3, \ w = a_1 b_2 - a_2 b_2 
    % .\] 

    % Énoncé originel : 

    Soient $P = \left[ a_1, a_2, a_3 \right]$ et $Q = \left[ b_1, b_2, b_3 \right]$ deux points distinct de $\mathbb{P}^2$. Il existe une unique droite de $\mathbb{P}^2$ passant par $P$ et $Q$. C'est l'ensemble des points $\left[ x, y, z \right] \in \mathbb{P}^2$ tels que le déterminant de la matrice
    \[
        M = 
    \begin{pmatrix}
        a_1 & b_1 & x \\ 
        a_2 & b_2 & y \\
        a_3 & b_3 & z
    \end{pmatrix}
    \] 
    soit nul. Autrement dit, c'est la droite d'équation $ux + vy + wz = 0$, avec
    \[
    u = a_2b_3 - a_3b_2, \ v = a_3b_1 - a_1b_3, \ w = a_1b_2 - a_2b_2
    .\] 
\end{lemme}

\begin{demonstration}

    Montrons qu'il existe une droite $D$ passant par $P$ et $Q$.

    Les éléments $u, \ v$ et $w$ ne sont pas tous nuls car $P$ et $Q$ sont distincts.
    L'équation $ux+vy+wz =0$ est donc celle d'une droite qui par définition contient $P$ et
    $Q$.

    Montrons que cette droite est unique.

    Soit une droite de $\mathbb{P}_{2}$ passant par $P$ et $Q$ d'équation
    \[
    u'x+v'y+w'z = 0 
    .\] 

    Soient $f$ et $g$ les formes linéaires $\overline{K}^3 \to \overline{K}$ définies par 
    \[
    f(x,y,z) = ux + vy + wz \quad \text{et} \quad g(x,y,z) = u'x+v'y+w'z
    .\] 

    Le noyau de $f$ et $g$ est le plan de $\overline{K}^3$ engendré par $\left( a_1,a_2,a_3
    \right) $ et $\left( b_1,b_2,b_3 \right) $. En particulier, $f$ et $g$ ont le
    même noyau. Dans le dual de $\overline{K}^3$, l'orthogonal du noyau de $f$ 
    (resp. g) est la droite engendrée par $f$ (resp g). Il existe $\lambda \in
    \overline{K}$ non nul tel que $f = \lambda g$, d'où l'unicité.
\end{demonstration}

\section{Tangente à $E$ en un point}

Le deuxième objet, dont l'on est amené à utilisé est la tangente.

Soit 
\[
E:\ y^2z = x^3 + axz^2 + bz^3
,\] l'équation de $E$, où $a,b \in K$.

Notons
\[
    F = Y^2Z - \left( X^3 + aXZ^2 + bZ^3 \right) \in K[X,Y,Z]
,\] 

\[
F_{X} = \frac{\partial F}{\partial X},\quad F_{Y} = \frac{\partial F}{\partial Y},\quad F_{Z} = \frac{\partial F}{\partial Z}
,\] 

c'est-à-dire, 

\[
F_{X} = - \left( 3X^2 + aZ^2 \right),\quad F_{Y} = 2YZ,\quad F_{Z} = Y^2 - \left( 2aXZ + 3bZ^2 \right)
.\] 

\begin{lemme}
    \label{lem:lemme3}
    
    Il n'existe pas de point $P \in E$ tel que
    \[
    F_{X}(P) = F_{Y}(P) = F_{Z}(P) = 0
    .\] 
\end{lemme}

\begin{demonstration}
   Supposons par l'absurde, qu'il existe un tel point $P \in E$. Remarquons que $F_{Z}(O) = 1 \neq 0 = F_{Z}(P)$ donc par hypothèse $P$ est distinct de $O$.

   Pour fixer les idées posons $P = [x,y,1]$.

   Puisque $\car(K) \neq 2$, on a
   \[
       (F_{Y} = 0) \iff (2YZ = 0) \iff (Y = 0) 
   ,\] 

   donc $y = 0$.
   
   Donc $P$ serait de la forme $[x,0,1]$.
   
   On obtient alors
   \[
   F_{X} = - \left( 3X^2 + a \right) = 0 \text{ et } F_{Z} = -\left( 2aX + 3b \right) = 0
   .\] 
   
   \begin{itemize}
       \item Supposons $a \neq 0$, on alors à partir de $F_{Z}$
   \[
       X = - \frac{3b}{2a} 
   .\] 

   Donc par $F_{X}$
\begin{align*}
    - \left( 3 (- \frac{3b}{2a})^2 + a \right) &= 0 \\
    - \left( 27b^2 + 4a^3 \right) &= 0
.\end{align*}
Ce qui est absurde car $E$ est elliptique. D'après le lemme \ref{lem:lemme1}
        \item Supposons que $a = 0$, alors
            \[
                (3b = 0) \underbrace{\implies}_{\car(K) \neq 3} (b = 0)
            .\] 

            Donc on $a = b = 0$ donc $-\left( 27b^2 + 4a^3 \right) = 0$ absurde car $E$ est elliptique. (lem \ref{lem:lemme1})

            D'où le résultat.
   \end{itemize}
\end{demonstration}

\begin{definition}
    Pour tout $P \in E$, la tangente à $E$ en $P$ est la droite d'équation
    \[
    F_{X}(P)x+F_{Y}(P)y+F_{Z}(P)z=0
    .\] 
\end{definition}

\begin{lemme}
    \label{lem:lemme4}
    
    \begin{description}
        \item[1)] L'équation de la tangente à $E$ au point $O$ est z = 0.
        \item[2)] Soit $P = \left[ x_0, y_0, 1 \right]$ un point de $E$ distinct de $O$. L'équation de la tangente à $E$ en $P$ est
            \[
            F_{X}(P)\left( x - x_0z \right) + F_{Y}(P)\left( y - y_0z \right) = 0
            .\] 
    \end{description}
\end{lemme}

\begin{demonstration}
    \begin{description}
        \item[1)] Soit  $O \in E$ le point à l'infini. D'après l'équation de la tangente à $E$ au point $O$. On a successivement 
            \[
            F_{X}(O) = 0, \ F_{Y}(O) = 0 \text{ et } F_{Z}(O) = 1
            .\] 
            Ainsi on retrouve bien $z = 0$.

        \item[2)] Soit $P$ un tel point, d'après l'équation (cite? set up snippet -nommé + cité) de la tangente et de l'égalité $y_0^2 = x_0^3 + ax_0 + b$ on a,
            \begin{align*}
                F_{X}(P)x + F_{Y}(P)y + F_{Z}(P)z &= 0 \\
                - \left( 3x_0^2 + a \right)x + \left( 2y_0 \right)y + \left( y_0^2 - \left( 2ax_0 + 3b \right) \right)z &= 0 \\
                - 3x_0^2x - ax + 2y_0y + y_0^2z - 2ax_0z - 3bz &= 0 \\ 
                - 3x_0^2x - ax + 2y_0y + y_0^2z - 2ax_0z - 3z\left( y_0^2 - x_0^3 - ax_0 \right) &= 0 \left( \text{i.e } b = y_0^2 - \left( x_0^3 + ax_0 \right) \right) \\
                - 3x_0^2x -ax + 2y_0y + y_0^2z - 2ax_0z - 3y_0^2z + 3x_0^3z + 3ax_0z &= 0 \\
                - \left( 3x_0^2 + a \right)x + 2y_0y - 2y_0^2z + \left( 3x_0^2 + a \right)x_0z &= 0\\
                - \left( 3x_0^2 + a \right)\left( x - x_0 \right) + 2y_0\left( y - y_0z \right) &= 0
            .\end{align*}
            D'où le résultat.
    \end{description}
    
\end{demonstration}


\section{Loi de composition des cordes-tangentes}

La proposition \ref{prop:secTanGeo} nous permet de définir la loi de composition des
cordes-tangentes qui satisfait :

\begin{enumerate}
    \item Si $P,Q \in E$, distinct, nous pouvons définir la droite $D = (PQ)$ la corde
        à la courbe passant par $P$ et $Q$. Grâce à la proposition \ref{prop:secTanGeo} on sait
        que cette corde prolongé à une droite intersecte la courbe $E$ en un unique troisième
        point qui appartient à $E \cap D$. Nous noterons ce troisième point $f(P,Q)$.
    \item Si $P \in E$, et que $Q=P$, on peut définir la tangente $D = (PP)$ à $E$ au point
        $P$. De nouveau, la proposition \ref{prop:secTanGeo} nous garantie l'existence d'un
        troisième point unique en comptant les multiplicités qui appartient à $E \cap D$. On
        notera ce dernier $f(P,P)$.
\end{enumerate}

\begin{figure}[h]
    \centering
    \includegraphics[width=0.8\textwidth]{cordesTangentes}
    \caption{Illustration de la loi des cordes-tangentes}
    \label{fig:cordesTangentes}
\end{figure}

On peut remarquer que sur la Figure \ref{fig:cordesTangentes}, une droite verticale
àla courbe $E$ ne semble
pas la couper en un troisième point. Ceci est lié à la difficulté de représenter le plan
projectif $\mathbb{P}_{2}$ sur un plan. Ce troisième point existe, et appartient à la droite infini
$\mathbb{P}_{1}$. Pour une courbe elliptique il correspond au $\mathcal{O}$.

\begin{remarque}
    Le meilleur moyen de considérer $\mathbb{P}_{1}$ est de se représenter ses éléments comme
    l'ensemble des directions possibles des droites du plan affine. Dans le cas particulier
    des courbes elliptiques, on a vu que $P_{1}$ se limite à un seul élément, que l'on a noté
    $\mathcal{O}$, qui correspond à la direction des droites verticales.
\end{remarque}

\begin{proposition}
    Pour tous points $P_1,P_2,Q_1$ et $Q_2$ de $E(K)$, on a :
    \[
    f(f(P_1,P_2),f(Q_1,Q_2)) = f(f(P_1,Q_1),f(P_2,Q_2))
    .\] 

    Pour une démonstration de ce résultat, voir 
\end{proposition}

% \subsection{Point de vue algébrique}
\begin{proposition}
    \label{prop:secTanAlg}
    
    Soient $P$ et $Q$ des points de $E$. Soit $D$ la droite de $\mathbb{P}^2$ passant par $P$ et $Q$ si $P \neq Q$, ou bien la tangente à $E$ en $P$ si $P = Q$. On a
    \[
    D \cap E = \left\{ P, Q, f(P,Q) \right\}
    ,\] 
    où $f(P,Q)$ désigne le point de $E$ défini par les conditions suivantes.
    \begin{description}
        \item[1)] Supposons $P \neq Q, \ P \neq O$ et $Q \neq O$.
            \begin{description}
                \item[i)] Supposons $x_{P} \neq x_{Q}$. Posons
                    \[
                    \lambda = \frac{y_{P} - y_{Q}}{x_{P} - x_{Q}} \text{ et } \nu = \frac{x_{P}y_{Q} - x_{Q}y_{P}}{x_{P} - x_{Q}}
                    .\] 

On a 
\begin{align}
    \label{eq:interne1}
f(P,Q) = \left[ \lambda - x_{P} - x_{Q}, \lambda \left( \lambda^2 - x_{P} - x_{Q} \right) + \nu, 1 \right]
.\end{align}
                \item[ii)] Si $x_{P} = x_{Q}$, on a $f(P,Q) = O$.
            \end{description}
        \item[2)] Supposons $P \neq O$ et $Q = O$. On a
            \begin{align}
                \label{eq:interne2}
            f(P,O) = \left[ x_{P}, -y_{P}, 1 \right]
            .\end{align}
            De même, si $P = O$ et $Q \neq O$, on a $f(O,Q) = \left[ x_{Q}, -y_{Q}, 1 \right]$
        \item[3)] Si $P = Q = O$, on a $f(O,O) = O$.
        \item[4)]  Supposons $P = Q$ et $P \neq O$.
            \begin{description}
                \item[i)] Si $y_{P} = 0$, on a $f(P,P) = O$.
                \item[ii)] Supposons $y_{P} \neq 0$. Posons
                    \[
                    \lambda = \frac{3x_{P}^3 + a}{2y_{P}} \text{ et } \nu = \frac{-x_{P}^3 + ax_{P} + 2b}{2y_{P}}
                    .\] 
On a
\begin{align}
    \label{eq:interne3}
f(P,P) = \left[ \lambda^2 - 2 x_{P}, \lambda\left( \lambda^2 - 2x_{P} \right) + \nu, 1 \right]
.\end{align}
            \end{description}
    \end{description}
\end{proposition}

\begin{demonstration}
    \begin{description}
        Soient $P = \left[ x_{P}, y_{P}, 1 \right]$ et $Q = \left[ x_{Q}, y_{Q}, 1 \right]$ des points de $E$ tels qu'ils sont distincts. Alors il existe une droite $D \in \mathbb{P} ^2$ qui passe par $P$ et $Q$.
        \item[1)] Supposons $P \neq Q$, $P \neq O$ et $Q \neq O$. Donc comme $D$ existe, il
            existe un point $M \in D \cap E$ et on cherche donc à trouver ses coordonnées.
            \begin{description}
        \item[i)] Supposons $x_{P} \neq x_{Q}$. 
            Comme $P,Q \neq O$, le point à l'infini n'appartient pas à $D$. Comme $M \in D$, il est de la même forme que $P$ et $Q$.
            Posons $M = \left[ x_0, y_0, 1 \right]$ avec $x_0$, $y_0$ des coordonnées sur $\overline{K}$.

            Comme $M \in E$, on a la première égalité
            \begin{align}
                \label{eq:droite} 
             y_0^2 = x_0^3 + ax_0 + b 
            .\end{align}
            Ensuite avec $M \in D$ d'après le lemme \ref{lem:lemme2}  on a la matrice suivante
            \[
                \begin{pmatrix}
                    x_P & x_Q & x_0 \\
                    y_P & y_Q & y_0  \\
                    1   & 1   & 1
                \end{pmatrix}
            ,\] 
           qui nous permet d'obtenir une seconde égalité.
           \begin{align*}
               \left( y_P - y_Q \right)x_0 - \left( x_P - x_Q \right)y_0 + \left( x_P y_Q - x_Q y_P \right) &= 0 \\
               y_0 = \frac{y_P - y_Q}{x_P - x_Q}x_0 + \frac{x_P y_Q - x_Q y_P}{x_P - x_Q}
           .\end{align*}
           Posons 
           \[
           \lambda = \frac{y_P - y_Q}{x_P - x_Q} \quad \text{et} \quad \nu = \frac{x_P y_Q - x_Q y_P}{x_P - x_Q}
           .\] 
           Donc l'équation de D est de la forme
           \[
           y = \lambda x + \nu z
           ,\] 
           c'est-à-dire dans notre cas on a
           \[
           y_0 = \lambda x_0 + \nu
           .\] 
           En remplacent dans \eqref{eq:droite}, il vient
           \begin{align*}
               \left( \lambda x_0 + \nu  \right)^2 = x_0^3 + ax_0 + b \\
               \lambda^2 x_0^2 + 2\lambda \nu x_0 + \nu^2 = x_0^3 + ax_0 + b \\
               x_0^3 - \lambda^2 x_0^2 + \left( a - 2\lambda \nu  \right)x_0 + b - \nu^2 = 0
           .\end{align*}
           Donc $x_0$ est une racine du polynôme
            \[
           H = X^3 - \lambda X^2 + \left( a - 2\lambda\nu \right)X + b - \nu^2
           .\] 
           On remarque que $H(x_P) = H(x_Q) = 0$ donc $x_p$ et $x_q$ sont aussi des racines de $H$.
           Par les relations coefficients racines obtient la valeur de $x_0$
           \begin{align*}
               x_0 + x_P + x_Q = - \left( - \lambda^2 \right) \\
               x_0 = \lambda^2 - x_P - x_Q
           .\end{align*} 
           Ainsi les racines de $H$ sont
           \[
           x_P, \quad x_Q \quad \text{et} \quad \lambda^2 - x_P - x_Q
           .\] 
           Il en résulte que $D \cap E$ est formé de $P$, et du point $M = f(P,Q)$.
           Donc  
           \begin{align*}
               f(P,Q) &= \left[ x_0, y_0, 1 \right] \\
                      &= \left[ \lambda^2 - x_P - x_Q, \lambda x_0 + \nu, 1 \right] \\
                      &= \left[ \lambda^2 - x_P - x_Q, \lambda\left( \lambda^2 - x_P - x_Q \right), 1\right]
           .\end{align*}
           D'où l'assertion.
       \item[ii)] Supposons $x_P = x_Q$. Comme $P$ et $Q$ sont distinct, on a alors $y_P = - y_Q$.
               D'après le lemme \ref{lem:lemme2} , la matrice suivante
               \[
                   \begin{pmatrix} x_P & x_Q & x \\
                   y_P & - y_Q & y \\
               1 & 1 & z
           \end{pmatrix} 
               .\] 

               D'où l'équation de la droite suivante
               \begin{align*}
                   2y_Px - 2y_Px_Pz &= 0 \\
                   x &= x_Pz
               .\end{align*}
               Donc le point $O$ est aussi un point de la droite $D$ donc de $D \cap E$.  Soit $M \in  D \cap E$ distincts de $O$. Si $M = \left[ 0,1,0 \right]$, d'après la situation on a $x_0 =
               x_P$ et $y_0 = \pm y_P$, donc $M = P$ ou $M = Q$. Or on a $P,Q \neq O$. Donc on a nécessairement $M = O$. Ainsi on a bien $D \cap E = \left\{ P, Q, f(P,Q)= O \right\}$, d'où
               l'assertion dans ce cas ci.
            \end{description}
        \item[2)] Supposons $P \neq O$ et $Q = O$. Donc d'après lemme \ref{lem:lemme2} , on a
            \[
            \begin{pmatrix}
                x_P & 0 & x \\
                y_P & 1 & y \\
                1   & 0 & z
            \end{pmatrix}
            .\] 
            À partir de la deuxième ligne on obtient l'équation de la droite suivante
            \begin{align*}
                x_Pz - x &= 0 \\
                x &= x_Pz
            .\end{align*}
            Si $M = \left[ x_0, y_0, 1 \right]$ est un point de $D \cap E$, on a donc $x_0 = x_P$ d'où $y_0 = \pm y_P$.
        On a ainsi $D \cap E = \left\{ P, O, f(P,O) \right\}$, où $f(P,O) = \left[ x_P, - y_P, 1 \right]$.
    \item[3)] Supposons $P = Q = O$, par le lemme \ref{lem:lemme4}  la tangente $D$ à E au point $O$ à pour $z = 0$. Par suite, $O$ est le seul point de $D \cap E$, d'où $f(O,O) = O$.
    \item[4)] Supposons $P = Q$ et $P \neq O$. L'équation de la tangente $D$ à $E$ en $P$ a donc pour équation 
        \[
        F_{X}(P)\left( x - x_Pz \right) + F_{Y}(P)\left( y - y_Pz \right) = 0
        .\] 
        \begin{description}
            \item[i)] Si $y_P = 0$, on a
                \[
                x_P^3 + ax_P + b = 0
                .\] 
                Donc $x_P$ est racine simple de ce polynôme. De plus, $F_{X}(P) \neq 0$.
                En effet, si $F_{X}(P) = 0$ on a
                \begin{align*}
                    - \left( 3x_P^2 + a \right) &= 0 \\
                    x_P^2 = - \frac{a}{3}
                ,\end{align*}
                ce qui est absurde.

                Ainsi à partir de l'équation de la tangente $D$ on a
                \begin{align*}
                    F_{X}(P)\left( x - x_Pz \right) = 0 &\implies \left( F_{X}(P) \right) = 0 \ou \left( x - x_Pz \right) = 0 \\
                                                        &\implies x - x_Pz = 0
                .\end{align*}
                Donc pour $D$ on a 
                \[
                D : x = x_Pz
                .\] 
                Le seul point de $D \cap E$ distinct de $P$ est donc le point $O$, d'où $D \cap E = \left( P,O \right)$, d'où l'assertion.
            \item[ii)] Supposons $y_P \neq 0$. Du lemme \ref{lem:lemme4}  et de l'équation $b = y_P^2 - x_P^3 - ax_P$ on obtient
                \begin{align*}
                    - \left( 3x_P^2 + a \right) \left( x - x_Pz \right) + 2y_P\left( y - y_Pz \right) &= 0 \\
                    - 3x_P^2x + 3 x_P^3z - ax + ax_Pz + 2y_Py - 2y_P^2z &= 0 \\
                    2y_Py = 3x_P^2x - 3x_P^3z + ax - ax_Pz +2y_P^2z \\
                    2y_Py - ax_Pz = 3x_P^2x + ax - x_P^3z + 2b \\
                    y = \frac{3x_P^2 + a}{2y_P}x + \frac{- x_P^3 + ax_P + 2b}{2y_P}z
                .\end{align*}
                On pose $\lambda = \frac{3x_P^2 + a}{2y_P}$ et $\nu = \frac{- x_P^3 + ax_P + 2b}{2y_P}$ et on obtient l'équation de $D$, c'est-à-dire
                \[
                y = \lambda x + \nu z
                .\] 
                Le point $O$ n'est donc pas sur $D$. Soit $M = \left[ x_0, y_0, 1 \right]$ un point de $E \cap D$. On a par le même raisonnement que dans le cas (1-i) (utilise ref?) les deux équations suivantes
                \[
                y_0^2 = x_0^3 + ax_0 + b \quad \text{et} \quad y_0 = \lambda x_0 + \nu
                .\] 
                Par suite $x_0$ est une racine du polynôme
                \[
                G = X^3 - \lambda^2 X^2 + \left( a - 2\lambda\nu  \right)X + b - \nu^2
                .\] 
                Le polynôme dérivé de $G$ est
                \[
                G' = 3X^2 - 2\lambda^2 X + a - 2\lambda\nu
                .\] 
                On a
                \[
                    \begin{cases}
                G(x_P)=(0) \iff x_P^3-\lambda^2x_P^2+\left( a-2\lambda\nu \right)X+b-nu^2=0 \\
                y_P^2=x_P^3+ax_P+b \implies b=y_P^2-x_P^3-ax_P \quad \text{et} \quad y_P=\lambda x_P + \nu \implies \nu = y_P - \lambda x_P
                    \end{cases}
                .\] 
                Donc,
                \begin{align*}
                    G(x_P) &=x_P^3-\lambda^2x_P^2+\left( a-2\lambda\left( y_p-\lambda x_P \right)  \right) x_P + y_P^2-x_P^3-ax_P-\left( y_P -\lambda x_P \right) ^2\\
                    &= x_P^3-\lambda^2 x_P^2+ax_P-2\lambda x_Py_P+2\lambda^2x_P^2+y_P^2-x_P^3-ax_P-y_P^2+2\lambda x_Py_P -\lambda^2x_P^2 \\
                    &= 2\lambda^2x_p^2 
                .\end{align*}
                Par suite,
                \begin{align*}
                    G'(x_P)=0 &\iff 3x_P^2-G(x_P)+a-2\lambda\nu =0 \\
                              & \iff G(x_P) = 3x_P^2+a-2\lambda\nu \\
                              & \iff G(x_P)=0 \\
                              & \iff x_P \text{ racine de G}
                .\end{align*}
Ainsi, $x_P$ est une racine d'ordre au moins $2$ de $G$. Les racines de $G$ sont donc 
\[
x_P \quad \text{et} \quad \lambda^2-2x_P
.\] 
On obtient donc par le même raisonnement que (1-i) la formule annoncé.
        \end{description}
    \end{description}
\end{demonstration}

On obtient alors une loi de composition interne sur $E$, appelée loi de composition des
cordes-tangentes, $f\ :\ E\times E \to E$ qui à tout couple de point $(P,Q)$ de la courbe
associe le point d'intersection de la corde ou tangente associé $f(P,Q) \in E$ défini
dans la proposition \ref{prop:secTanGeo} 

\begin{exemple}
\end{exemple}

\section{Loi de groupe}

% \subsection{Point de vue géométrique}

\begin{theoreme}
    Soit un corps $K$. Soit $E$ une courbe elliptique définie sur $K$. Soient $P$ et $Q$ deux
    points de cette courbe. Alors, 

    \[
    P + Q = f(f(P,Q),\mathcal{O})
    ,\] 
    définit une structure de groupe commutatif ayant $\mathcal{O}$ pour élément neutre
    de la loi.
\end{theoreme}

\begin{figure}[h]
    \centering
    \includegraphics[width=0.8\textwidth]{associativiteGeo}
    \caption{Illustration de l'associativité de la loi de groupe}
    \label{fig:associativiteGeo}
\end{figure}

\begin{demonstration}
    \begin{enumerate}
        \item La loi $+$ est bien interne puisque $P + Q$ est l'intersection entre la courbe et
            une droite, c'est à dire un point de la courbe.

        \item La loi $+$ est associative (voir Figure). En effet, si $P,Q$ et $R$ sont trois
            point de la courbe, on a

            \begin{align*}
            f(P,f(Q+R)) &= f(P,f(f(Q,R),\mathcal{O})) \\
            &= f(f(f(P,Q),Q),f(f(Q,R),\mathcal{O})) \text{ car } P = f(f(P,Q),Q) \\
            &= f(f(f(P,Q),\mathcal{O}),f(f(Q,R),Q)) \text{ voir la proposition \ref{prop:} } \\
            &= f(f(f(P,Q),\mathcal{O}),R) \text{ voir la Figure \ref{} } \\
            &= f(f(P+Q),R)
            .\end{align*}

            En appliquant $\mathcal{O}$ sur les deux membres de l'égalité, on trouve

            \[
            P + (Q+R) = (P+Q) + R
            .\] 

        \item L'élément $\mathcal{O}$ est l'élément neutre de la loi additive (voir Figure
            \ref{}). 

            En effet,

            \[
            P + \mathcal{O} = f(f(P,O)O) = P \quad \text{et} \quad \mathcal{O}+P =
            f(f(\mathcal{O},P),\mathcal{O}) = P
            .\] 

        \item Tout point $P$ possède un inverse pour la loi $+$. En effet, il faut vérifier que
            le point $-P= f(f(\mathcal{O},\mathcal{O}),P)$ est bien l'inverse de $P$ 

            \begin{align*}
                P + (-P) &= f(f(P,-P),\mathcal{O}) \\
                &= f(f(f(P,f(f(\mathcal{O},\mathcal{O}),P),\mathcal{O}))) =
                f(f(\mathcal{O},\mathcal{O}),\mathcal{O}) =
                \mathcal{O}+\mathcal{O}=\mathcal{O}  \\
                (-P) + P &= f(f(-P,P),\mathcal{O}) \\
                &= f(f(f(f(\mathcal{O},\mathcal{O}),P),P),\mathcal{O}) =
                f(\mathcal{O},f(\mathcal{O},\mathcal{O})) + \mathcal{O}+\mathcal{O}=
                \mathcal{O} \\
            .\end{align*}

        \item Enfin la loi $+$ est commutative. C'est à dire que si $P$ et $Q$ sont deux
            points de la courbe, on a

            \[
            P + Q = f(f(P,Q),\mathcal{O}) = f(f(Q,P),\mathcal{O}) = Q + P
            .\] 
    \end{enumerate}    
\end{demonstration}

Les propiétés de la loi de groupe sur une courbe elliptique sont représentées sur la Figure
\ref{}.

\begin{figure}[h]
    \centering
    \includegraphics[width=0.8\textwidth]{P+O}
    \caption{L'élément neutre de la loi de groupe}
    \label{fig:P-O}
\end{figure}

\begin{figure}[h]
    \centering
    \includegraphics[width=0.8\textwidth]{loiGroupe}
    \caption{La loi de groupe + sur l'ensemble des points rationnels d'une courbe
    elliptique}
    \label{fig:loiGroupe}
\end{figure}

% Considérons comme précédemment $a$ et $b$ des éléments de $K$ tels que $4a^3+27b^2\neq 0$ et $E$ la courbe elliptique définie sur $K$ d'équation
% \[
% y^2=x^3+ax+b
% .\] 
% Notons $+$ la loi de composition interne sur $E$, définie pour tous $P$ et $Q$ dans $E$ par l'égalité
% \begin{align}
%     \label{eq:groupe}
%     P+Q=f(f(P,Q),O)
% .\end{align}

\begin{theoreme}
    \label{th:theoreme1}
    Le couple $\left( E,+ \right) $ est un groupe abélien, d'élément neutre $O$. La loi interne $+$ est décrite explicitement par les formules suivantes.

    Soient $P$ et $Q$ des points de $E$ distincts de $O$. Posons $P=\left( x_P,y_P \right) $ et $Q=\left( x_Q,y_Q \right) $.

    \begin{description}
        \item[1)] Supposons $x_P\neqx_Q$. Posons 
            \[
            \lambda=\frac{y_P-y_Q}{x_P-x_Q} \quad \text{et} \quad \nu=\frac{x_Py_Q=x_Qy_P}{x_P-x_Q}
            .\] 
            On a 
            \begin{align}
                \label{eq:add1}
            P+Q=\left( \lambda^2-x_P-x_Q,-\lambda\left( \lambda^2-x_P-x_Q \right) -\nu  \right) 
            .\end{align}
            \item[2)] Si $x_P=x_Q$ et $P\neqQ$, on a $P+Q=O$.
            \item[3)] Supposons $P=Q$ et $y_P\neq 0$. Posons 
                \[
                \lambda=\frac{3x_P^2+a}{2y_P} \quad \text{et} \quad \nu=\frac{-x_P^3+ax_P+b}{2y_P}
                .\] 
                On a 
                \begin{align}
                    \label{eq:add2}
                2P=\left( \lambda^2-2x_P,\lambda\left( -\lambda^2-2x_P \right) -\nu \right) 
                .\end{align}
        \item[4)] Si $P=Q$ et $y_P=0$, on a $2P=O$.
        \item[5)] L'opposé de $P$ est le point
            \begin{align}
                \label{eq:add3}
                -P=\left( x_P,-y_P \right) 
            .\end{align}
    \end{description}
\end{theoreme}

\begin{demonstration}
    \begin{description}
        \item[1)] Supposons $x_P\neqx_Q$, compte tenu de \eqref{eq:groupe}, \eqref{eq:interne1} et \eqref{eq:interne2} on a
            \[
            \begin{cases}
                \eqref{eq:interne1} \iff f(\left[ \lambda^2-x_P-x_Q,\lambda\left( \lambda^2-x_P-x_Q \right) +\nu,1 \right] , \left[ 0,1,0 \right] ) \\
                \eqref{eq:interne2} \iff f(P,O)=\left[ x_P,-y_P,1 \right] 
            \end{cases}
            .\] 
            On retrouve bien la formule \eqref{eq:add1}.
        \item[2)] Supposons $x_P=x_Q$ et $P\neq Q$ c'est à dire $y_P\neq y_Q$.

            D'après la proposition \ref{prop:secTanGeo}  (1-i), on a $f(P,Q)=O$ donc $f(f(P,Q),O)=f(O,O)=O$. D'où la formule énoncé.
        \item[3)] Supposons $P=Q$ et $y_P\neq 0$, en prenant compte \eqref{eq:groupe} , \eqref{eq:interne2} et \eqref{eq:interne3} on obtient
            \[
            \begin{cases}
                \eqref{eq:interne3} \iff f(\left[ \lambda^2-2_x_P,\lambda\left( \lambda^2-2x_P \right) +\nu,1 \right] ,\left[ 0,1,0 \right] )\\
                \eqref{eq:interne2} \iff f(P,O)=\left[ x_P,-y_P,1 \right] 
            \end{cases}
            .\] 
            Ce qui permet de retrouver la formule \eqref{eq:add2}. 
        \item[4)] Supposons $P=Q$ et $y_P=0$, d'après l'assertion (4-i) de la proposition \ref{prop:secTanGeo} , on a $f(P,P)=O$ d'où $2P=f(f(P,P),O)=f(O,O)=O$. 
        \item[5)] Pour l'opposer on cherche un point $M \in E$ tel que $P\neq M$ et $P,Q\neq O$ d'après le théorème énoncé assertion 2) on a donc $x_P = x_M$ et donc nécessairement $y_M=-y_P$ donc le point recherché est $M=\left( x_M,y_M \right) = \left( x_P,-y_P \right) =-P$. 
    \end{description}
\end{demonstration}

\begin{remarque}
    Pour ce qui est de l'associativité de la loi de groupe, il faudrait traiter chaque cas et
    montrer que les formules sont bien associative. Ce qui long et fastidieux. 
\end{remarque}

\begin{exemple}
    mettre exemple de calcul de $2P$ pour la suite
\end{exemple} 


    \chapter{Applications}
Le groupe abélien $(E,+)$ des points rationnels d'une courbe elliptique et même les courbes elliptiques en général,
ont de nombreuses applications que ce soit dans le domaine pratique, ou bien dans le domaine
théorique.

En effet, on peut notamment citer leurs utilisations dans la mécanique classique dans la
description du mouvement des toupies. Elles interviennent également en théorie des nombres, dans la démonstration
du dernier théorème de Fermat.

Enfin, on les retrouve aussi en cryptologie, dans le problème de la factorisation des entiers.

Dans ce mémoire, on s'intéresse à leur application en cryptographie. Où elles ont
permis notamment la réduction de la taille des clés cryptographiques.

\section{Cryptosystèmes elliptiques}

Aujourd'hui, le groupe $E$ des points rationnels d'une courbe elliptique intervient notamment pour l'échange de clé
et les signatures numériques.

Nous allons nous intéresser à deux cryptosystèmes asymétriques, à savoir le protocole d'échange de
clés Diffie-Hellman, ainsi que l'algorithme d'El-Gamal, basé sur le principe du protocole de
Diffie-Hellman, qui permet d'échanger des messages chiffrés à l'aide d'une clé
publique et de déchiffrer les messages avec la clé secrète de chaque utilisateur.

La force des cryptosytèmes asymétriques réside dans la difficulté, voir l'impossibilité
actuelle dans le cas elliptique, de résoudre le problème du logarithme discret de façon
générale.

Dans le cas des cryptosystèmes à clé publique classique, on s'appuie sur le groupe multiplicatif d'un corps fini et de son groupe des inversibles. Ce qui réduit grandement notre choix comparé aux versions elliptique des algorithmes équivalent.

En effet, dans le cas elliptique, on remplace le groupe multiplicatif sur un corps fini par le
groupe des points rationnels d'une courbe elliptique. L'avantage de cette méthode est que pour
un corps fini  $K$ donné, on dispose généralement de nombreux choix de courbes elliptiques $E$
sur $K$. Autrement dit, on a de nombreux groupes $E(K)$, pour utiliser efficacement un
cryptosystème asymétrique elliptique contrairement aux versions classiques comme énoncé plus
haut, où
l'on ne dispose que du groupe des inversibles $K^{*}$.

\subsection{Problème du logarithme discret elliptique}

Le problème du logarithme discret pour les courbes elliptiques est analogue à celui
des corps fini, dont on peut trouver un énoncé dans ce cours \cite[p18]{KrausCf}.

Soit $K$ un corps et $E$ une courbe elliptique définie sur $K$. Les points $K$-rationnels
formant un groupe abélien, donne un cadre pour le problème du logarithme discret.

\begin{definition}
    Soit $E$ une courbe elliptique définie sur $K$ et $Q \in E(K)$. Connaissant le point $P
    \in E(K)$, le problème du logarithme discret consiste à trouver $n \in \N$, s'il existe.
    tel que $P = nQ$.
\end{definition}

Un tel entier $n$ n'existe pas toujours. De plus $E(K)$ n'est pas nécessairement cyclique. Afin
d'essayer de résoudre ce problème, on peut utiliser l'algorithme de Silver, Pohlig et
Hellman \cite[p20]{KrausCF} ou l'algorithme Baby step - Giant step \cite[p38]{KrausCE}.

\begin{remarque}
    Le problème du logarithme discret est généralement beaucoup plus difficile à
    résoudre dans le groupe des points rationnels d'une courbe elliptique $E$ sur un
    corps fini $K$, que celui dans $K^*$. 
\end{remarque}

Cela est dû au fait que, les algorithmes de résolution du problème du logarithme discret
pour le groupe multiplicatif d'un corps fini, sont de plus en plus efficaces pour résoudre le
problème comme on peut le voir dans cet article \cite{Lipton2013}. Ainsi, il est
de plus en plus clair que la taille des clés, requises pour maintenir un haut niveau de
sécurité pour le protocole RSA, se doivent d'être de plus en plus grande (i.e. 4096-bit).
Alors que les courbes elliptiques s'en sortent avec des clés beaucoup plus petite de l'ordre
de 256-bit. La raison est essentiellement mathématique, c'est-à-dire, que l'addition des points
rationnels d'une courbe elliptique est moins bien comprise que la multiplication pour les
entiers. Ainsi, la complexité apparente du groupe rend le problème intrinsèquement plus
difficile. C'est pourquoi tant qu'il n'y a pas de méthode générale efficace pour résoudre
le problème dans le contexte des courbes elliptiques. On peut avoir une sécurité aussi efficace
que RSA pour des clés beaucoup plus petites. 

Bien que l'on ne connaisse pas tous les paramètres qui rendent ou non une courbe
mathématiquement sûre. On connaît tout de même un certain nombres d'attaques contre certains
paramètres bien spécifiques.
Ainsi, il est recommander :
\begin{itemize}
    \item D'être sûr que l'ordre du point choisi ait une factorisation courte (i.e. $2p,3p$ ou
        $4p$, pour $p$ premier). Autrement on est vulnérable à une attaque basé sur le théorème
        des restes chinois, la plus importante étant celle de Pohlig-Hellman.
    \item D'être sûr que la courbe choisie ne soit pas supersingulière. Sinon on peut réduire
        le problème du logarithme discret à un problème différent dans un groupe plus simple.
    \item Si la courbe $E$ est définie sur $\Z / p \Z$, avec $p$ premier, on doit vérifier que
        le nombre de points de la courbe ne soit pas égale à $p$. Ce type de courbe est dite
        "anormale" et on peut réduire le problème du logarithme à la version additive sur les
        entiers.
    \item Ne pas choisir $\mathbb{F}_{2^{m}}$ avec $m$ petit. On peut utiliser l'algorithme de
        rho Pollard qui est très efficace contre ce genre de corps fini.
    \item Si on utilise le corps fini $\mathbb{F}_{2^{m}}$ alors il est plus sûr de choisir
        $m$ premier.
\end{itemize}

On peut retrouver dans ce cours \cite[p17-18]{Delaunay} des exemples d'algorithmes pour 
choisir convenablement des points rationnels ou choisir de bonnes courbes. Il y a également,
d'autres recommandation sur le choix de $E$.

Ainsi, comme on peut s'y attendre quant à l'énoncé du problème du logarithme discret les points
qui vont nous intéresser sont les multiples d'un point. On peut retrouver un algorithme pour le
calcul de ce point dans ce cours \cite[p10]{Delaunay}. Il est basé sur la décomposition de $n$
dans la base $2$. Ainsi, on amener à effectuer peu d'opération pour obtenir un multiple d'un
point rationnels de la courbe. On à la formule suivante pour le calcul de $nP$ 
\[
nP = \sum_{i= 0}^{n} a_{i}2^{i}P
,\] 
avec les $a_{i}$ dans ${0,1}$.

Autrement dit, pour calculer $19P$, il vient $19 = 1 \times 2^{0} + 1 \times 2^{1} + 0 \times
2^{2} + 0 \times 2^{3} + 1 \times 2^{4} $ et ainsi $19P = P + 2P + 16P$ et on calcul 9
doublements et 3 additions.

Ainsi, quand on effectue le rapport entre le nombre d'opération nécessaire pour calculer le
multiple d'un point et celui pour calculer sont logarithme. On est amener à effectuer beaucoup
plus d'opération pour le logarithme et ceci est la base de la sécurité des protocole suivant.


\subsection{Protocole Diffie-Hellman}

Dans ce tout ce qui suit Alice et Bob sont deux personnes qui souhaite s'échanger soit un message, soit
une clé secrète. Cependant, il faut bien comprendre qu'il peuvent également représenter deux
entités qui souhaitent communiquer via des messages chiffrés ou bien s'échanger une clef
secréte via des cannaux publics. Par entité, j'entends soit des banques, des entreprises ou
tout ce qui serait suceptible de vouloir communiquer secrètement entre eux.

Alice et Bob souhaitent s'échanger publiquement une clé secrète commune. Pour cela ils se mettent d'accord pour la construire selon le procédé suivant:

\begin{description}
    \item[1)] Ils choisissent un corps fini $K$ et une courbe elliptique $E$ définie sur $K$, pour que le problème du logaritme discret soit difficile à résoudre dans le groupe $E(K)$. Ils choisissent un point $P \in E(K)$. Ils rendent alors publique le triplet $(K,E,P)$.

    \item[2)] Alice choisit un entier naturel secret non nul $a$ et calcule le point $P_a=aP$, qu'elle transmet publiquement à Bob.

    \item[3)] Bob procède de la même façon en choisissant un entier naturel secret, non nul,
        $b$, et il calcule de son côté le point $P_b=bP$, qu'il transmet publiquement à Alice.

    \item[4)] Alice calcule le point $aP_b=a(bP)$.

    \item[5)] Bob calcule le point $bP_a=b(aP)$.
\end{description}

Ils ont ainsi construit leur clé secrète commune qui est le point $abP$.

\begin{probleme}[Diffie-Hellman]
   Connaissant $P$, $aP$ et $bP$ dans $E(K)$, comment déterminer $abP$ ? 
\end{probleme}

On ne sait pas à ce jour résoudre ce problème sans calculer $a$ ou $b$, autrement dit, sans
savoir résoudre le problème du logarithme discret dans $E(K)$. Cela étant, on n'a pas la preuve
qu'il n'existe pas d'autres moyens pour y parvenir. Ainsi le problème de Diffie-Hellman est une
hypothèse plus forte que le problème du logarithme discret car elle en dépend mais elle dépend
aussi du fait qu'on ne sache pas s'il existe un autre moyen pour résoudre ce problème.

\begin{exemple}
    Soit la courbe définit par 
    \[
    E : y^2 = x^3 + 324x + 1287
    ,\] 
    sur le corps $\mathbb{F}_{p}$, avec $p=3851$ qui est premier. 

    Soit le point $P \in E(\mathbb{F}_{p})$, avec $P = (920;303)$.

    La courbe et le point sont publiques.

    Alice choisit l'entier secret $a=1194$ et calcule $aP=1194P=(2067,2178) \in
    E(\mathbb{F}_{3851})$ et l'envoie à Bob.

    Bob choisit l'entier secret $b = 1759$ et calcul $bP = 1759P = (3684,3125) \in
    E(\mathbb{F}_{p})$.

    Finalement,

    Alice calcule $a(bP)=1194.(3684,3125)=(3347,1242) \in E(\mathbb{F}_{p})$ et

    Bob calcule $b(aP) = 1759.(2067,2178) = (3347,1242) \in E(\mathbb{F}_{p})$.

    Ils peuvent alors déduire du point échangé à l'aide de la coordonnée $x=3347$ une clé
    secrète pour la cryptographie symétrique.
\end{exemple}

\subsection{Algorithme d'El Gamal}

Alice souhaite envoyer un message chiffré à Bob. Pour se faire elle choisit un corps fini $K$,
une courbe elliptique $E$ définie sur $K$ de sorte que le problème du logarithme discret soit
difficile à résoudre dans le groupe $E(K)$. Elle choisit ensuite un point $P \in E(K)$. Enfin
elle choisit son entier naturel secret, non nul, $s$ et calcule le point $A=sP$.

Elle rend ainsi public le quadruplet 
\[
    (K,E,P,A)
.\] 

C'est la base de ce qui va permettre à Alice et Bob de pouvoir communiquer de façon
confidentielle entre eux.

Ainsi, pour que Bob puisse envoyer un message chiffré $M \in E(K)$ à Alice, il choisit secrétement un entier non nul $k$ et calcule les points
\[
M_1=kP \quad \text{et} \quad M_2=M+kA
.\] 
Il transmet alors publiquement à Alice le couple $(M_1,M_2)$. C'est donc la phase d'encryptage du message $M$.

Pour qu'Alice puisse déchiffrer le message $M$, elle doit calculer le point
\[
M_2-sM_1
.\] 
Ce qui lui permet grâce au calcul suivant de retrouver $M$:
\[
M_2-sM_1=M+kA-s(kP)=M+kA-kA=M
.\] 

    \newpage
\begin{center}
    \textbf{Perspectives}

    La suite naturelle de ce qui a été étudié dans ce mémoire est l'étude des structures de
    groupes sur le groupe $(E,+)$ défini sur un corps fini. Pour cela, on utilise les
    morhpismes de groupes de $E(\overline{K})$ pour ensuite utiliser le morphisme de Frobenius
    et le corps des points de torsion. Ainsi à l'aide du théorème fondamental et du théorème de
    Hasse, dont je n'ai pas parlé mais dont les énoncés sont disponibles dans ce cours
    \cite[p15-30]{KrausCE}, on peut étudier l'ordre du groupes $E(K)$. Bien qu'on en n'ait pas
    une formule explicite on obtient une borne supérieur ainsi que l'intervalle de Hasse. Ceci
    nous permet à partir de l'ordre d'un point retrouver l'ordre du groupe. La question du
    cardinal du groupe reste cependant encore une question ouverte. Grâce à cette étude on peut
    par ailleurs donner un critère pour différencier les courbes singulières des courbes
    ordinaires.

    Dans une autre optique on pourrait s'intéresser à la cryptographie post-quantique, en
    commençant par lire cet article de vulgarisation sur le sujet \cite{Kachigar2018}, ce qui
    permettrait de comprendre les enjeux relatif au fonctionnement des calculateurs quantiques.
    Et comprendre les défis qu'amène le développement des ordinateurs quantiques. On peut
    notamment cité l'agorithme de Shor qui permet de résoudre le problème de la facorisation
    des entiers sur lequel est basé le système RSA.
\end{center}

    \printbibliography[heading=bibintoc,title={Références}]
    % \chapter{Cryptosystèmes}
\section{context}
\textbf{À revoir si c'est le bonne endroit où placer tout ça. Il est peut-être préférable de
mettre tout cela dans l'introduction?}

En cryptographie parmis les deux types de cryptosystème à notre disposition. À savoir les cryptosytèmes symétriques (i.e. à clé secrète) et les cryptosystèmes asymétriques (i.e. à clé publique). On peut à l'aide de la théorie des courbes elliptique adapter les cryptosystèmes asymétriques dit classique à leur équivivalents, c'est-à-dire, les cryptosystèmes asymétriques dit elliptique.

La force des cryptosytèmes asymétrique réside dans la difficulté, voir l'impossibilité actuel dans le cas elliptique, de résoudre le problème du logarithme discret que nous énoncerons par la suite.

Dans le cas des cryptosystèmes à clé publique classique, on s'appuie sur le groupe multiplicatif d'un corps fini et de son groupe des inversibles. Ce qui réduit grandement notre choix comparé aux versions elliptique des algorithmes équivalent.

En effet, dans le cas elliptique, on remplace le groupe multiplicatif sur un corps fini par le groupe des points rationnels d'une courbe elliptique. L'avantage de cette méthode est que pour un corps fini  $K$ donné, on dispose généralement de nombreux choix de courbes elliptiques $E$ sur $K$. Autrement dit, on a de nombreux groupes $E(K)$, pour utiliser efficacement un cryptosystème asymétrique élliptique contrairement aux versions classique comme énoncé plus haut où
l'on ne dispose que du groupe des inversible $K^{*}$.


Dans ce qui suit Alice et Bob sont deux personnes qui souhaite s'échanger soit un message, soit une clef secrète. Cependant, il faut bien comprendre qu'il peuvent également représenter deux entité qui souhaitent communiqué via des messages chiffrés ou bien s'échanger une clef secréte via cannaux publique. Par entité, j'entends soit des banques, des entreprises ou tout ce qui serait suceptible de vouloir communiqué secretement entre eux.

De plus le choix des clés secret s'effectue de façon aléatoire dans le respect des conditions de chaque cas.
\section{Cryptosystème version classique}

\subsection{Algorithme d'El Gamal}
Une personne Alice, souhaite envoyer à quiconque des messages confidentiels. Pour ce faire, elle choisit au départ un couple qui sera public (i.e. accessible à tout le monde). Ce couple est $(K,g)$ où $K$ est un corps fini et $g$ un générateur du groupe des inversibles de ce corps à savoir $K^{*}$. 

Soit $q$ le cardinal de $K$.

L'algorithme d'El Gamal est alors le suivant:

\begin{description}
    \item[1)] Alice choisit un entier $a$ tel que $1<a<q-1$, qui sera sa clé secrète.

        Elle calcul alors $g^{a}$ qu'elle rend public, et qui sera considéré comme sa clé publique.

        On a donc au départ le triplet $(K,g,g^{a})$ qui est connue de tous.

    \item[2)] Pour qu'une personne Bob puisse envoyer un message $m \in K$ à Alice, il choisit un entier $b$ qui lui aussi est tel que $1<b<q-1$. Bob transmet alors à Alice le couple:
        \[
            (g^{b},mg^{ab})
        ,\] 
        où $g^{b}$ représente la clé publique de Bob.
        C'est ce qu'on appelle la phase d'encryptage du message $m$.

    \item[3)] Pour que Alice puisse déchiffrer le message reçu, elle passe par la phase dite de décryptage. C'est-à-dire, connaissant son entier secret $a$ et la clé publique de Bob, à savoir $g^{b}$, elle doit alors déterminer l'inverse de $(g^{b})^{a}$ dans $K$. C'est-à-dire l'entier $g^{-ab}$. 

        Il lui suffit alors d'effectuer la multiplication de $g^{-ab}$ par $mg^{ab}$, qui nous donne alors:
        \[
        g^{-ab}\left( mg^{ab} \right)=m 
        .\] 
        Ce qui permet donc à Alice de retrouver le message clair $m$ et Alice et Bob on donc pu communiquer de façon publique en toute discrétion.
\end{description}

\subsection{Protocole de Diffie-Hellman}

À la différence de l'agorithme d'El Gammal, ici deux personnes Alice et Bob souhaite se construire une clé secrete commune via cannaux public donc à la vue de tous, qui seront donc les seuls à connaître. Ceci leur permettra donc de pouvoir communiqué sur un canal non sûr en utilisant cette clé pour déchiffrer leur correspondance.

Comme pour l'algorithme d'El Gamal, on se donne un corps fini $K$, ainsi qu'un générateur $g \in K^{*}$, qui seront tout deux public. Donc $(K,g)$ est connu de tous.

Le procédé de construction de leur clé secret est ainsi le suivant:

\begin{description}
    \item[1)] Alice choisit sa clé secret qui est un entier $a$ tel que $1<a<q-1$, elle transmet ensuite publiquement à Bob l'entier $g^{a}$.

    \item[2)] Bob choisit de la même manière un entier $b$, et il transmet lui aussi publiquement l'élément $g^{b}$ à Alice.

    \item[3)] Alice pour sa part élève $g^{b}$ à la puissance $a$, ce qui lui permet d'obtenir l'élément $(g^{b})^{a}$.

    \item[4)] Bob d'autre part, élève $g^{a}$ à la puissance $b$, et il obtient donc l'élément $(g^{a})^{b}$.

        Ainsi Alice et Bob on pu se construire de façon public une clé secret commun qui est l'entier $g^{ab}$.
\end{description}

\section{Cryptosystème version elliptique}

\subsection{Algorithme d'El Gamal}

Alice souhaite envoyer un message chiffré à Bob. Pour se faire elle choisit un corps fini $K$, une courbe elliptique $E$ définie sur $K$ de sorte que le problème du logarithme discret soit difficile à résoudre dans le groupe $E(K)$. Elle choisit ensuite un point $P \in E(K)$. Enfin elle choisit sont entier naturel secret, non nul, $s$ et calcul et calcul le point $A=sP$.

Elle rend ainsi public le quadruplet 
\[
    (K,E,P,A)
.\] 

C'est la base de ce qui va permettre à Alice et Bob de pouvoir communiquer de façon confidentiel entre eux.

Ainsi, pour que Bob puisse envoyer un message chiffré $M \in E(K)$ à Alice, il choisit secrétement un entier non nul $k$ et calcules les points
\[
M_1=kP \quad \text{et} \quad M_2=M+kA
.\] 
Il transmet alors publiquement à Alice le couple $(M_1,M_2)$. C'est donc la phase d'encryptage du message $M$.

Pour qu'Alice puisse déchiffrer le message $M$, elle doit calculer le point
\[
M_2-sM_1
.\] 
Ce qui lui permet grâce au calcul suivant de retrouver $M$:
\[
M_2-sM_1=M+kA-s(kP)=M+k(sP)-s(kP)=M+skP-skP=M
.\] 

\subsection{Protocol Diffie-Hellman}

Alice et Bob souhaite s'échanger publiquement une clé secrète commune. Pour cela ils se mettent d'accord pour la construire selon le procédé suivant:

\begin{description}
    \item[1)] Ils choisissent un corps fini $K$ et une courbe elliptique $E$ définie sur $K$, pour que le problème du logaritme discret soit difficile à résoudre dans le groupe $E(K)$. Ils choisissent un point $P \in E(K)$. Ils rendent alors publique le triplet $(K,E,P)$.

    \item[2)] Alice choisit un entier naturel secret non nun $a$ et calcul le point $P_a=aP$, qu'elle transmet publiquement à Bob.

    \item[3)] Bob procède de la même façon en choisissant un entier naturel secret, non nul, $b$, et il calcul de son côté le point $P_b=bP$, qu'il transmet publiquement à Alice.

    \item[4)] Alice calcul le point $aP_b=a(bP)$.

    \item[5)] Bob calcul le point $bP_a=b(aP)$.
\end{description}

Ils ont ainsi construit leur clé secret commun qui est le point $abP$.

    % \section{sources}

Cours Delaunay: 
\begin{itemize}
    \item def 1: lisse
    \item prop 1: raison pour lem 7.1
    \item equation réduite raison de pourquoi $k$ diff de $3$ et $2$
    \item exemple des eq pour  $\car(k)=2$ ou $3$
    \item def 3: ensembles des points $k$-rationnels
    \item def 4: log
    \item th 5,6 et def 5: point rationnels et courbe supersingulière
    \item rq: pour parler de courbe singulière il faut introduire le morphisme de frobenius
    \item recherche des points rationnels différent algo
    \item protocol signature :
        \begin{itemize}
            \item algo choix de courbe

                cherche c'est quoi $l$, $t$ et $MOV$
            \item algo ecdsa (signature et vérification)
        \end{itemize}
    \item factorisation exemple d'algo info etc peut-être en parler brièvement tout en invitant
        à aller consulter pour cela il est peut-être bon d'en faire un résumer trés bref des
        différents lien.

        Je compte parler de l'algo de lenka et peut-être ce qu'il y a dans ce cours
\end{itemize}

Crypto asy et ce (chabanol)
\begin{itemize}
    \item el gammal basé sur $\left( \Z / p\Z \right) ^{*}$ muni de la multiplication
        constitue un groupe abélien
    \item enigma mm machine assortiede la clé (reglage de la machine) permet de dé/chiffrer
    \item désavantage de sym partage de la clé
    \item avantage de asy partage de la clé
    \item exemple concret paiement en ligne
    \item problème log discret version vulgarisé (oral)
    \item exponentiation rapide exemple rapport avec l'addition de $E$
    \item principe asym fonction à sens unique
    \item gros moyen mis en oeuvre pour dvp des algo de facto
    \item on ne sait pas prouver qu'une fonction est difficile à inverser
    \item plus sur que RSA
    \item histoire ec
    \item symétrie par rapport à $x$ 
    \item solution de $f$ sont les abscisses des points d'intersection de cette courbe
        avec l'axe $x$
    \item $\delta = 0$
    \item  $\delta > 0$ 
    \item $\delta < 0$
    \item example associativité
    \item remarque rationnels à partir d'un point $P$ au coordonnées rationnel on obtient
        que des multiple à coordonnées rationnel
    \item question ouverte cardinal
    \item désavantage face à RSA (donne les raisons voir delaunay)
    \item étude de l'allure de ces courbes
\end{itemize}

bib
\begin{itemize}
    \item courbe ell
        \begin{itemize}
            \item coordonnées selon équation de droites vérifie que c'est bien la même chose que ce
        que tu as et si c'est pas le cas pourquoi
            \item structure et ordre difficile à connaitre
            \item analogue plus compliqué que $\left( \Z / p\Z \right)^*$
        \end{itemize}
    \item chiffré à l'aide des c-e
        \begin{itemize}
            \item point de vue de eve
            \item difficulté de transmission de texte voir livre
            \item avantage inconvénients
                \begin{itemize}
                    \item log plus compliqué
                    \item taille des clé réduite
                    \item carte à puce peu de puissance, influence de la clé sur
                        performances
                    \item récent 
                    \item bcp de brevet donc potentiellement plus couteux
                \end{itemize}
        \end{itemize}
    \item factorisation c-e
        \begin{itemize}
            \item méthode $p-1$ de pollard
            \item facto via courbe elliptiques (Lenstra)
                \begin{itemize}
                    \item algo
                    \item pourquoi ca marche
                \end{itemize}
        \end{itemize}
    \item facto entier
        \begin{itemize}
            \item crucial par rapport à RSA
            \item différent algo possible
            \item principe derrière la facto
        \end{itemize}
\end{itemize}

Crypto et ordi quantique (Kachigar)
\begin{itemize}
    \item intro
        \begin{itemize}
            \item intro crypto
            \item antiquité romaine
            \item crypto moderne = crypto à clé publique
            \item hyp fonction à sens unique
            \item algo shor facto
            \item algo simon 1990
            \item casser les cryptosys
            \item discussion comment resister aux ordi quantique
            \item exemple sur impact sur la crypto à clé secrète
        \end{itemize}
    \item crypto état des lieux
        \begin{itemize}
            \item césar
            \item la vrai diff entre sym et asym c'est la difficulté de trouver $f^{-1}$
            \item def crypto systeme
            \item fonction à sens unique pas de preuve existence
            \item exemple détaillé RSA
            \item nombre de clé nécessaires
            \item rapidité sym $>$ asym
            \item exemple pratique d'hybride en footnote
        \end{itemize}
    \item asym vs quantique
        \begin{itemize}
            \item type d'attaque
            \item algo shor attaque de type math
            \item tps de calcul rsa rapport entre sec et facto
            \item explication compléxité rsa
            \item explication entre temps de calcul pour dé/chiffrer
            \item problème de recherche de période
            \item simon 1994 inspire shor ordi quantique efficaces problème de recherche
                de période
            \item shor permet de ramener le probleme de facto à un probleme de période grace a
                des propritété d'arithmétique
            \item exploite le fait que le problème soit doté d'une structure
            \item au contraire d'une simple recherche d'éléments
            \item raison calcul quantique capable de tel calcul
            \item resumé sur $P=NP$ liens avec quantique et crypto
        \end{itemize}
    \item sym vs quantique
        \begin{itemize}
            \item résultat de claude shanon longueur du texte et de la clé égale pour une
                sécurité parfaite
            \item parle un peu du principe moderne de sym
            \item présente le shéma d'enven-mansour (preuve de concept) de comment ramener
                les problèmes lier à la crypto sym à des problème de période
        \end{itemize}
\end{itemize}

wiki
\begin{itemize}
    \item el gamal
        \begin{itemize}
            \item GNUpg
            \item info
        \end{itemize}
    \item crypto hybride
        \begin{itemize}
            \item GnuPG
            \item PGP
            \item TLS
        \end{itemize}
    \item log
        \begin{itemize}
            \item pas d'algo pour reciproque mais addition nb de multi log en taille de l'arg
            \item dans certain groupe log difficile alors que expo facile algo expo rapide
            \item exemple d'algo pour resoudre log
        \end{itemize}
    \item crypto sur c-e
        \begin{itemize}
            \item intro quelques info utile peut-être
            \item histoire et motivation
                \begin{itemize}
                    \item algo ss-expo pour log (crible gene du corps de nb)
                    \item travail dans corps assez large implique cout implementation,
                        transmission et tps de calcul augment
                    \item tout par du groupes points rationnels de $E$ 
                    \item variétés algébrique collection de points satisfaisant une équation
                        à plusieurs indéterminées
                    \item groupe d'ordre premier donc cyclique cad qu'il existe $P$ qui
                        engendre le groupe. conséquence on peut passer des versions
                        classique aux version ell sur les algo type el et dh
                    \item seuls les algo générique comme l'algo de shanks pouvaient
                        résoudre le log discret dans $E$
                    \item cela rendait l'attaque bcp plus diff sur $E$ que sur un corps de
                        nombre
                    \item conséquence niveau de sec satisfaisant, diminution de la
                        taille des données manipulées
                    \item gain de vitesse reduction des couts d'implé et transmi
                    \item attaque MOV c-e supersingulière exemple d'attaque
                \end{itemize}
        \end{itemize}
    \item probleme dh
        \begin{itemize}
            \item def
            \item relation avec log
        \end{itemize}
    \item plongement de veil
        \begin{itemize}
            \item degré de plongement attaque mov
        \end{itemize}
    \item echange cle dh
        \begin{itemize}
            \item 2015 prix turing
            \item fondement math
        \end{itemize}
    \item calculateur quantique
        \begin{itemize}
            \item partie sur la crypto
        \end{itemize}
    \item crypto sur c-e
        \begin{itemize}
            \item liste d'algo
            \item liste de courbe
            \item liste de primaire cryptographique
        \end{itemize}
    \item echange de clé
        \begin{itemize}
            \item TLS le plus utilisé
            \item prix turing 2016 pour dh
            \item avant clé publique
            \item info protocole dh
            \item info signature
        \end{itemize}
    \item signature
    \item TLS
    \item fonction de hachage
    \item EC multiplication
    \item curve25519
    \item théoreme de bézout geo algé (en et fr)
    \item expo by squaring
    \item one-way function
    \item diophante 
    \item eq diophantienne
    \item time complexity
    \item asymptotic analysis
    \item chiffrement par substitution
    \item droite à l'infini
    \item plan pro réel
    \item plan pro
    \item espace pro
    \item droite pro
    \item géo pro
\end{itemize}

miximum sur algo courbe elliptique avec un exemple sur les signature

handbook sur les c-e presentation général du sujet notamment sur le plan projectif et
l'intersection des axe $x,y,z$ au point $\mathcal{O}$
 
une page sur stack exchange avec le meme principe qu'au dessus et une autre encore peut etre
encore plus simple à comprendre

image pour les transformation projective (perspective)

j'ai l'article de kerckhoff en lien


\section{Plan}

\begin{itemize}
    \item introduction
        \begin{itemize}
            \item crypto
                \begin{itemize}
                    \item origine grecs
                    \item considération historique
                    \item étymologie, signification porté historique (guerre)
                    \item crypto sym
                        \begin{itemize}
                            \item césar
                            \item vigenere
                            \item cryptanalise
                            \item enigma
                            \item ce qui amène kerckhoff
                        \end{itemize}
                    \item principe de kerckhoff
                    \item crypto asym
                        \begin{itemize}
                            \item intro principe kerckhoff
                            \item cryptosys
                            \item D-H 1976
                            \item intro asym
                            \item principe termes mathématiques
                            \item enjeux asym
                        \end{itemize}
                    \item prob fac et log
                \end{itemize}
        \end{itemize}
    \item courbes elliptiques
        \begin{itemize}
            \item parallèle historique dans le sens en même temps
        \end{itemize}
    \item plan projectif
        \begin{itemize}
            \item espace pro
            \item droite projective
                \begin{itemize}
                    \item def
                    \item classe d'équivalence (explicite ce que signifie les coordonnées pro
                        non ?)
                    \item point de la droite
                    \item division en deux type
                        \begin{itemize}
                            \item $[x,1]$ droite affine
                            \item $[x,0]$ point à l'infini
                        \end{itemize}
                    \item plan projectif
                        \begin{itemize}
                            \item def (à changer pour être dans la continuité de la def
                                précédente.
                        \end{itemize}
                \end{itemize}
        \end{itemize}
    \item def g
        \begin{itemize}
            \item def de la courbe
            \item poly homogène $F$
            \item conséquence def : zero de $F$ et $f$ racine simples
            \item lemme sur delta
            \item demo lemme complète
            \item partie fine et point à l'infini
            \item $U,\phi,\mathcal{O},E$
            \item def point rationnel sur L
            \item remarque pas trop compris
            \item rationnel sur E
            \item exemples pas encore mis à choisir
        \end{itemize}
    \item loi de groupe
        \begin{itemize}
            \item def droite dans $\mathbb{P}^2$
            \item lemme unicité et existence de la droite
            \item démo pas fini!
            \item tangente rajoute des info si possible
            \item lemme je crois c'est par rapport à l'existence d'une unique tangente
            \item demo complète
            \item def equation tangente
            \item lemme des cas tangente
            \item demo complète
            \item prop cordes-tangentes
            \item demo complète
            \item th loi de groupe
            \item demo complète
            \item exemple et remarque à choisir
        \end{itemize}
    \item applications
        \begin{itemize}
            \item application domaine pratique et th
            \item exemples
            \item ce qui m'intéresse
            \item application crypto
            \item DH
            \item el-gammal
            \item ecsda
        \end{itemize}
\end{itemize}

% pistes:
% \begin{itemize}
%     \item Factorisation (algo $\N$ et $E$)
%     \item
% \end{itemize}



% Dans ce mémoire, nous allons nous intéresser aux courbes elliptiques et plus particulièrement
% au groupe abélien des courbes elliptiques. Ce qui va nous permettre d'utiliser ce groupe en
% cryptographie et présenter une façon plus efficace d'utiliser l'algorithme d'El-Gamal (EG.) (date) et le
% protocol de Diffie-Hellman (D-H.) (date).

% Dans un monde en constant évolution, notamment technique. Il est crucial de pouvoir
% améliorer, réinventer, ou même changer, des principes qui ont révolutionner à leur époque.
% C'est pourquoi, en (date) Klobnitz, à présenter une façon concrète d'utiliser les courbes
% elliptiques dans le cadre de la cryptographie. Ceci a permit d'apporter une nouvelle façon
% de faire de la cryptographie, tout en conservant des concepts eprouvé basé sur le problème
% du logarithme discret. Cette nouvelle approche, que l'on nommera
% version elliptique, contrairement à la version dite classique de chiffrement, qui est basé
% sur le groupe multiplicatif $\left( \Z /p\Z \right) ^{*}$ et sa commutativité, celle ci
% présente une diversité et complexité non négligeable. 

% En effet, que ce soit l'algorithme d'El Gamal ou le protocol de Diffie-Hellman, leur version
% classique est basé sur le générateur du groupe $\left( \Z / p\Z \right) ^{*}$, alors que
% leur version elliptique est basé sur les courbes elliptiques qui comme on le verra sont en grand
% nombre pour leur part. De plus, par construction du groupe des courbes elliptiques, plus
% abstrait, la résolution du logarithme discret est quasiment impossible sans l'aide d'ordinateur
% quantique extremement puissant (ref article Mme Abdelatif).

% \section{plan ?}

% \begin{itemize}
%     \item explication cryptographie
%         \begin{itemize}
%             \item sym et asym
%             \item rsa
%             \item El Gamal et D-H
%         \end{itemize}
%     \item histoire des fonctions elliptiques 
% \end{itemize}

% \section{La cryptographie}
% L'application première de notre construction étant la cryptographie, il me semble nécessaire de
% poser les bases de cette branche des mathématiques. Ceci nous permettra d'avoir une idée clair
% des différents concepts et enjeux qui la compose.

% Tout d'abord, définition ce qu'est la cryptographie. Métaphysiquement, c'est le fait de vouloir
% communiquer des messages, entre diverses entités et ceci de façon à ce que seul ces dernières
% n'aient connaissances du contenue du message.

% Cette définition personnel et trivial est basé sur comment dans l'histoire la cryptographie est
% apparu.

% De nos jours, le concept c'est énormément diversifié. La transmission de message reste un élément
% majeur de ce qu'est la cryptographie mais 

    % end lectures
\end{document}
