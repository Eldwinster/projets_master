\lecture{5}{Mon 07 Feb 2022 21:04}{Proposition 7.1.} 

Dans cette proposition à l'aide du comportement de deux points du plan $E$ et de la droite qui les intersectent. On veut construit une loi de composition interne
\[
\begin{align*}
    \top : E \times E &\longrightarrow E \\
    (P,Q) &\longmapsto P  \ \top  \ Q
\end{align*}
\] 
Cette loi, comme on va le voir, n'est pas une loi de groupe. C'est ce qui va nous permettre cependant de donner, par la suite, de donner une structure de groupe au plan $E$ à l'aide d'une symétrie bien choisie.

\begin{proposition}
    Soient $P$ et $Q$ des points de $E$. Soit $D$ la droite de $\mathbb{P}^2$ passant par $P$ et $Q$ si $P \neq Q$, ou bien la tangente à $E$ en $P$ si $P = Q$. On a
    \[
    D \cap E = \left\{ P, Q, f(P,Q) \right\}
    ,\] 
    où $f(P,Q)$ désigne le point de $E$ défini par les conditions suivantes.
    \begin{description}
        \item[1)] Supposons $P \neq Q, \ P \neq O$ et $Q \neq O$.
            \begin{description}
                \item[i)] Supposons $x_{P} \neq x_{Q}$. Posons
                    \[
                    \lambda = \frac{y_{P} - y_{Q}}{x_{P} - x_{Q}} \text{ et } \nu = \frac{x_{P}y_{Q} - x_{Q}y_{P}}{x_{P} - x_{Q}}
                    .\] 

On a 
\begin{align}
    \label{eq:interne1}
f(P,Q) = \left[ \lambda - x_{P} - x_{Q}, \lambda \left( \lambda^2 - x_{P} - x_{Q} \right) + \nu, 1 \right]
.\end{align}
                \item[ii)] Si $x_{P} = x_{Q}$, on a $f(P,Q) = O$.
            \end{description}
        \item[2)] Supposons $P \neq O$ et $Q = O$. On a
            \begin{align}
                \label{eq:interne2}
            f(P,O) = \left[ x_{P}, -y_{P}, 1 \right]
            .\end{align}
            De même, si $P = O$ et $Q \neq O$, on a $f(O,Q) = \left[ x_{Q}, -y_{Q}, 1 \right]$
        \item[3)] Si $P = Q = O$, on a $f(O,O) = O$.
        \item[4)]  Supposons $P = Q$ et $P \neq O$.
            \begin{description}
                \item[i)] Si $y_{P} = 0$, on a $f(P,P) = O$.
                \item[ii)] Supposons $y_{P} \neq 0$. Posons
                    \[
                    \lambda = \frac{3x_{P}^3 + a}{2y_{P}} \text{ et } \nu = \frac{-x_{P}^3 + ax_{P} + 2b}{2y_{P}}
                    .\] 
On a
\begin{align}
    \label{eq:interne3}
f(P,P) = \left[ \lambda^2 - 2 x_{P}, \lambda\left( \lambda^2 - 2x_{P} \right) + \nu, 1 \right]
.\end{align}
            \end{description}
    \end{description}
\end{proposition}

Dans cette démonstration, on étudie le comportement des points $P$ et $Q$ selon qu'ils soient distincts ou égaux. Que ce soit pour la droite ou la tangente tous les deux vont éventuellement, soit
recouper la courbe elliptique et rester dans le plan $E$, soit "couper" le point à l'infini  $O$. C'est ce qu'on veut découvrir à l'aide du point que l'on a nommé $f(P,Q)$ qui désigne le comportement
par rapport à $P$ et $Q$ de ce troisième point.

\begin{demonstration}
    \begin{description}
        Soient $P = \left[ x_{P}, y_{P}, 1 \right]$ et $Q = \left[ x_{Q}, y_{Q}, 1 \right]$ des points de $E$ tels qu'ils sont distincts. Alors il existe une droite $D \in \mathbb{P} ^2$ qui passe par $P$ et $Q$.
        \item[1)] Supposons $P \neq Q$, $P \neq O$ et $Q \neq O$. Donc comme $D$ existe, il existe un point $M \in D \cap E$ et on cherche donc à connaitre son comportement dans le plan $E$.
            \begin{description}
        \item[i)] Supposons $x_{P} \neq x_{Q}$. 
            Comme $P,Q \neq O$, le point à l'infini n'appartient pas à $D$. Comme $M \in D$, il est de la même forme que $P$ et $Q$.
            Posons $M = \left[ x_0, y_0, 1 \right]$ avec $x_0$, $y_0$ des coordonnées sur $\overline{K}$.

            Comme $M \in E$, on a la première égalité
            \begin{align}
                \label{eq:droite} 
             y_0^2 = x_0^3 + ax_0 + b 
            .\end{align}
            Ensuite avec $M \in D$ d'après le lemme 7.2 on a la matrice suivante
            \[
                \begin{pmatrix}
                    x_P & x_Q & x_0 \\
                    y_P & y_Q & y_0  \\
                    1   & 1   & 1
                \end{pmatrix}
            ,\] 
           qui nous permet d'obtenir une seconde égalité.
           \begin{align*}
               \left( y_P - y_Q \right)x_0 - \left( x_P - x_Q \right)y_0 + \left( x_P y_Q - x_Q y_P \right) &= 0 \\
               y_0 = \frac{y_P - y_Q}{x_P - x_Q}x_0 + \frac{x_P y_Q - x_Q y_P}{x_P - x_Q}
           .\end{align*}
           Posons 
           \[
           \lambda = \frac{y_P - y_Q}{x_P - x_Q} \quad \text{et} \quad \nu = \frac{x_P y_Q - x_Q y_P}{x_P - x_Q}
           .\] 
           Donc l'équation de D est de la forme
           \[
           y = \lambda x + \nu z
           ,\] 
           c'est-à-dire dans notre cas on a
           \[
           y_0 = \lambda x_0 + \nu
           .\] 
           En remplacent dans \eqref{eq:droite}, il vient
           \begin{align*}
               \left( \lambda x_0 + \nu  \right)^2 = x_0^3 + ax_0 + b \\
               \lambda^2 x_0^2 + 2\lambda \nu x_0 + \nu^2 = x_0^3 + ax_0 + b \\
               x_0^3 - \lambda^2 x_0^2 + \left( a - 2\lambda \nu  \right)x_0 + b - \nu^2 = 0
           .\end{align*}
           Donc $x_0$ est une racine du polynôme
            \[
           H = X^3 - \lambda X^2 + \left( a - 2\lambda\nu \right)X + b - \nu^2
           .\] 
           On remarque que $H(x_P) = H(x_Q) = 0$ donc $x_p$ et $x_q$ sont aussi des racines de $H$.
           Par les relations coefficients racines obtient la valeur de $x_0$
           \begin{align*}
               x_0 + x_P + x_Q = - \left( - \lambda^2 \right) \\
               x_0 = \lambda^2 - x_P - x_Q
           .\end{align*} 
           Ainsi les racines de $H$ sont
           \[
           x_P, \quad x_Q \quad \text{et} \quad \lambda^2 - x_P - x_Q
           .\] 
           Il en résulte que $D \cap E$ est formé de $P$, et du point $M = f(P,Q)$.
           Donc  
           \begin{align*}
               f(P,Q) &= \left[ x_0, y_0, 1 \right] \\
                      &= \left[ \lambda^2 - x_P - x_Q, \lambda x_0 + \nu, 1 \right] \\
                      &= \left[ \lambda^2 - x_P - x_Q, \lambda\left( \lambda^2 - x_P - x_Q \right), 1\right]
           .\end{align*}
           D'où l'assertion.
       \item[ii)] Supposons $x_P = x_Q$. Comme $P$ et $Q$ sont distinct, on a alors $y_P = - y_Q$.
               D'après le lemme 7.2, la matrice suivante
               \[
                   \begin{pmatrix} x_P & x_Q & x \\
                   y_P & - y_Q & y \\
               1 & 1 & z
           \end{pmatrix} 
               .\] 

               D'où l'équation de la droite suivante
               \begin{align*}
                   2y_Px - 2y_Px_Pz &= 0 \\
                   x &= x_Pz
               .\end{align*}
               Donc le point $O$ est aussi un point de la droite $D$ donc de $D \cap E$.  Soit $M \in  D \cap E$ distincts de $O$. Si $M = \left[ 0,1,0 \right]$, d'après la situation on a $x_0 =
               x_P$ et $y_0 = \pm y_P$, donc $M = P$ ou $M = Q$. Or on a $P,Q \neq O$. Donc on a nécessairement $M = O$. Ainsi on a bien $D \cap E = \left\{ P, Q, f(P,Q)= O \right\}$, d'où
               l'assertion dans ce cas ci.
            \end{description}
        \item[2)] Supposons $P \neq O$ et $Q = O$. Donc d'après lemme 7.2, on a
            \[
            \begin{pmatrix}
                x_P & 0 & x \\
                y_P & 1 & y \\
                1   & 0 & z
            \end{pmatrix}
            .\] 
            À partir de la deuxième ligne on obtient l'équation de la droite suivante
            \begin{align*}
                x_Pz - x &= 0 \\
                x &= x_Pz
            .\end{align*}
            Si $M = \left[ x_0, y_0, 1 \right]$ est un point de $D \cap E$, on a donc $x_0 = x_P$ d'où $y_0 = \pm y_P$.
        On a ainsi $D \cap E = \left\{ P, O, f(P,O) \right\}$, où $f(P,O) = \left[ x_P, - y_P, 1 \right]$.
    \item[3)] Supposons $P = Q = O$, par le lemme 7.4 la tangente $D$ à E au point $O$ à pour $z = 0$. Par suite, $O$ est le seul point de $D \cap E$, d'où $f(O,O) = O$.
    \item[4)] Supposons $P = Q$ et $P \neq O$. L'équation de la tangente $D$ à $E$ en $P$ a donc pour équation 
        \[
        F_{X}(P)\left( x - x_Pz \right) + F_{Y}(P)\left( y - y_Pz \right) = 0
        .\] 
        \begin{description}
            \item[i)] Si $y_P = 0$, on a
                \[
                x_P^3 + ax_P + b = 0
                .\] 
                Donc $x_P$ est racine simple de ce polynôme. De plus, $F_{X}(P) \neq 0$.
                En effet, si $F_{X}(P) = 0$ on a
                \begin{align*}
                    - \left( 3x_P^2 + a \right) &= 0 \\
                    x_P^2 = - \frac{a}{3}
                ,\end{align*}
                ce qui est absurde.

                Ainsi à partir de l'équation de la tangente $D$ on a
                \begin{align*}
                    F_{X}(P)\left( x - x_Pz \right) = 0 &\implies \left( F_{X}(P) \right) = 0 \ou \left( x - x_Pz \right) = 0 \\
                                                        &\implies x - x_Pz = 0
                .\end{align*}
                Donc pour $D$ on a 
                \[
                D : x = x_Pz
                .\] 
                Le seul point de $D \cap E$ distinct de $P$ est donc le point $O$, d'où $D \cap E = \left( P,O \right)$, d'où l'assertion.
            \item[ii)] Supposons $y_P \neq 0$. Du lemme 7.4 et de l'équation $b = y_P^2 - x_P^3 - ax_P$ on obtient
                \begin{align*}
                    - \left( 3x_P^2 + a \right) \left( x - x_Pz \right) + 2y_P\left( y - y_Pz \right) &= 0 \\
                    - 3x_P^2x + 3 x_P^3z - ax + ax_Pz + 2y_Py - 2y_P^2z &= 0 \\
                    2y_Py = 3x_P^2x - 3x_P^3z + ax - ax_Pz +2y_P^2z \\
                    2y_Py - ax_Pz = 3x_P^2x + ax - x_P^3z + 2b \\
                    y = \frac{3x_P^2 + a}{2y_P}x + \frac{- x_P^3 + ax_P + 2b}{2y_P}z
                .\end{align*}
                On pose $\lambda = \frac{3x_P^2 + a}{2y_P}$ et $\nu = \frac{- x_P^3 + ax_P + 2b}{2y_P}$ et on obtient l'équation de $D$, c'est-à-dire
                \[
                y = \lambda x + \nu z
                .\] 
                Le point $O$ n'est donc pas sur $D$. Soit $M = \left[ x_0, y_0, 1 \right]$ un point de $E \cap D$. On a par le même raisonnement que dans le cas (1-i) (utilise ref?) les deux équations suivantes
                \[
                y_0^2 = x_0^3 + ax_0 + b \quad \text{et} \quad y_0 = \lambda x_0 + \nu
                .\] 
                Par suite $x_0$ est une racine du polynôme
                \[
                G = X^3 - \lambda^2 X^2 + \left( a - 2\lambda\nu  \right)X + b - \nu^2
                .\] 
                Le polynôme dérivé de $G$ est
                \[
                G' = 3X^2 - 2\lambda^2 X + a - 2\lambda\nu
                .\] 
                On a
                \[
                    \begin{cases}
                G(x_P)=(0) \iff x_P^3-\lambda^2x_P^2+\left( a-2\lambda\nu \right)X+b-nu^2=0 \\
                y_P^2=x_P^3+ax_P+b \implies b=y_P^2-x_P^3-ax_P \quad \text{et} \quad y_P=\lambda x_P + \nu \implies \nu = y_P - \lambda x_P
                    \end{cases}
                .\] 
                Donc,
                \begin{align*}
                    G(x_P) &=x_P^3-\lambda^2x_P^2+\left( a-2\lambda\left( y_p-\lambda x_P \right)  \right) x_P + y_P^2-x_P^3-ax_P-\left( y_P -\lambda x_P \right) ^2\\
                    &= x_P^3-\lambda^2 x_P^2+ax_P-2\lambda x_Py_P+2\lambda^2x_P^2+y_P^2-x_P^3-ax_P-y_P^2+2\lambda x_Py_P -\lambda^2x_P^2 \\
                    &= 2\lambda^2x_p^2 
                .\end{align*}
                Par suite,
                \begin{align*}
                    G'(x_P)=0 &\iff 3x_P^2-G(x_P)+a-2\lambda\nu =0 \\
                              & \iff G(x_P) = 3x_P^2+a-2\lambda\nu \\
                              & \iff G(x_P)=0 \\
                              & \iff x_P \text{ racine de G}
                .\end{align*}
Ainsi, $x_P$ est une racine d'ordre au moins $2$ de $G$. Les racines de $G$ sont donc 
\[
x_P \quad \text{et} \quad \lambda^2-2x_P
.\] 
On obtient donc par le même raisonnement que (1-i) la formule annoncé.
        \end{description}
    \end{description}
\end{demonstration}
