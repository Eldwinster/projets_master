\lecture{10}{Wed 16 Feb 2022 18:30}{Corollaire 7.2}

\begin{corollaire}
    Soit $\ell$ un nombre premier disctinct de $\car(K)$. Le groupe $E[\ell]$ est un $\mathbb{F}_{\ell}$-espace vectoriel de dimension $2$.

    Pour tout nombre premier $\ell$ distinct de $\car(K)$, si $\left( P_1,P_2 \right) $ est une base de $E[\ell]$ sur $\mathbb{F}_{\ell}$, tout point de $P \in E[\ell]$ s'écrit ainsi de manière unique sous la forme
    \[
    P=n_1P_1+n_2P_2
    ,\] 
    où $n_1$ et $n_2$ sont des entiers compris entre $0$ et $\ell-1$.
\end{corollaire}

\begin{demonstration}
    
\end{demonstration}
