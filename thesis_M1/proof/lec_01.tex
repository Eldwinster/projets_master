\lecture{1}{Sun 06 Feb 2022 02:51}{Lemme 7.1.}
\begin{lemme}
    Soit $g$ un polynôme unitaire à coefficients dans $K$ de degré $n \ge 1$. Soient $\alpha_1,\ldots,\alpha_{n}$ ses racines dans $\overline{K}$ comptées avec multiplicités. Le discriminant $\Delta$ de $g$ est défini par l'égalité
    \[
    \Delta = \prod_{i < j}^{} \left( \alpha_{i} - \alpha_{j} \right)^2 
    .\] 
    C'est un élément de $K$.
\end{lemme}

\begin{demonstration}
    
\end{demonstration}
\begin{lemme}
    Soit $\Delta= -(4a^3 + 27b^2)$ le discriminant $f : x^3 + ax + b$. 
    Les racines de $f$ sont simples, si et seulement si $\Delta \neq 0$.
\end{lemme}

\begin{demonstration}
    Montrons tout d'abord que le discriminant de $f$ est $\Delta= -(4a^3 + 27b^2)$.

    Soit $\Delta$ le discriminant de $f$. Soient $\alpha, \beta, \gamma $ les racines de $f$ dans $\overline{K}$ et $f'$ le polynôme dérivé de $f$.

    Tout d'abord montrons grâce au Lemme 1 que :
    \begin{align*}
        \Delta &= (-1) ( \alpha - \beta )^2 ( \alpha - \gamma )^2 ( \beta - \gamma )^2 \\
          &= - f(\alpha)'f(\beta )'f(\gamma)'
    .\end{align*}
    D'après le théorème de d'Alembert on peut écrire $f$ sous la forme :
    \[
        f = \left( X - \alpha \right) \left( X - \beta \right) \left( X - \gamma \right) 
    .\] 
    En dérivant $f$ sous cette forme on obtient :
    \begin{align*}
        f &= ( X - \alpha ) \left( ( X - \beta ) ( X - \gamma ) \right)'  + ( X - \beta ) ( X - \gamma )\\
          &= ( X - \alpha ) \left( ( X - \beta ) + ( X - \gamma ) \right) + ( X - \beta ) ( X - \gamma ) 
    .\end{align*}

    Donc  
\[
f' = ( X - \alpha ) ( X - \beta ) + ( X - \alpha ) ( X - \gamma ) + ( X - \beta ) ( X - \gamma )
.\] 

On a alors successivement : 
\[
    f(\alpha)' = ( \alpha - \beta) ( \alpha - \gamma ) \text{, } f(\beta )' = ( \beta - \alpha) ( \beta - \gamma) \text{ et } f(\gamma)' = ( \gamma - \alpha) ( \gamma - \beta)
.\] 

En multipliant ces trois expressions, on obtient :
\begin{align*}
    f(\alpha)' f(\beta )' f(\gamma)' &= ( \alpha - \beta ) ( \alpha - \gamma ) ( \beta - \alpha ) ( \beta - \gamma) ( \gamma - \alpha ) ( \gamma - \beta ) \\
&= \left( X - \alpha \right) \left( \alpha - \gamma \right) \left( -1 \right) \left( \alpha - \beta  \right) \left( \beta - \gamma \right) \left( -1 \right) \left( \alpha - \gamma \right) \left( -1 \right) \left( \beta - \gamma \right) \\
&= \left( -1 \right) ^3 \left( \alpha - \beta  \right) ^2 \left( \alpha - \gamma  \right) ^2 \left( \beta - \gamma \right) ^2\\
 &= - \Delta
.\end{align*}

Et donc 

\[
\Delta = - f'(\alpha) f'(\beta ) f'(\gamma)
.\] 

En partant de la forme $f : x^3 + ax + b$, on remarque que $f' : 3x^2 + a$. Par suite on obtient,
\begin{align*}
    \Delta &= - f'(\alpha) f'(\beta ) f'(\gamma) \\
      &= - \left( 3 \alpha^2 + a \right) \left( 3 \beta^2 + a \right) \left( 3 \gamma^2 + a \right) 
.\end{align*}
Ce qui donne :
\begin{align*}
    \Delta  &= - \left( ( 9 \alpha^2 \beta^2 + 3a ( \alpha^2 + \beta^2 ) + a^2 ) ( 3 \gamma^2 + a ) \right)  \\
       &= - \left( 27 \left( \alpha \beta \gamma \right)^2  + 9a \left( \alpha^2 \beta^2 + \alpha^2 \gamma^2 + \beta^2 \gamma^2 \right) + 3a^2 \left( \alpha^2 + \beta^2 + \gamma^2 \right) + a^3 \right) 
.\end{align*}
D'après l'identité suivante (sûrement faux):
\[
a^2 + b^2 = \left( a + b \right)^2 - 2ab 
.\] 
On peut écrire
\[
\alpha^2 + \beta^2 + \gamma^2 = \left( \alpha + \beta + \gamma \right)^2 - 2 \left( \alpha \beta + \alpha \gamma + \beta \gamma \right)
,\] 
\[
\alpha^2 \beta^2 + \alpha^2 \gamma^2 + \beta^2 \gamma^2 = \left( \alpha \beta + \alpha \gamma + \beta \gamma \right)^2 - 2\alpha \beta \gamma \left( \alpha + \beta + \gamma \right)
.\] 
Donc d'après les relations entre coefficients et racine (i.e relation de Viète), pour un polynôme de la forme $ax^3 + bx^2 + cx + d$, on a :
\[
\alpha + \beta + \gamma = - \frac{b}{a}
,\] 
\[
\alpha \beta + \alpha \gamma + \beta \gamma = \frac{c}{a}
,\] 
\[
\alpha \beta \gamma = - \frac{d}{a}
.\] 
Donc pour $f$ on a $a = 1$, $b = 0$, $c = a$ et $d = b$.

D'où,
\[
\alpha + \beta + \gamma = 0 \text{, } \alpha \beta + \alpha \gamma + \beta \gamma = a \text{ et } \alpha \beta \gamma = - b
.\] 
Ce qui donne : 
\begin{align*}
    \alpha^2 + \beta^2 + \gamma^2 &= 0^2 - 2a = 2a \\
    \alpha^2 \beta^2 + \alpha^2 \gamma^2 + \beta^2 \gamma^2 &= a^2 + 2b \times 0
.\end{align*}

Donc le discriminant vaut :
\begin{align*}
    \Delta &= - \left( 27b^2 + 9a^3 - 6a^3 + a^3  \right) \\
        &= - \left( 4a^3 + 27b^2  \right)
.\end{align*}

Maintenant, supposons que $\Delta = 0$. On a alors :
 \begin{align*}
     - \left( 4a^3 + 27^2 \right) = 0 &\iff - f(\alpha)' f(\beta )' f(\gamma)' = 0 \\
                                      & \iff \left( f(\alpha)' = 0 \right) \ou \left( f(\beta )' = 0 \right) \ou \left( f(\gamma)' = 0 \right) \\
                                      &\iff \alpha \text{ ou } \beta \text{ ou } \gamma \text{ est une racine multiple}
.\end{align*}

D'où le résultat.
\end{demonstration}
