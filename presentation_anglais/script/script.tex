\begin{center}
    \textbf{el\textbf{lip}tic \textbf{cur}ves cry\textbf{tog}raphy}
\end{center}
\begin{itemize}
    \item \textbf{Slide 1}: Hel\textbf{lo} \textbf{ev}eryone, let me \textbf{in}troduce
        my\textbf{self} \textbf{quick}ly. I'm
        Yann-Arby, I'm \textbf{stud}ying a \textbf{mas}ter de\textbf{gree} in
        ma\textbf{the}matics at the Uni\textbf{ver}sity of Picardie Jules
        Verne. To\textbf{day}, I'll speak about \hlred{el\textbf{lip}tic
        \textbf{cur}ves} and their
        appli\textbf{ca}tion in \textcolor{red}{cryp\textbf{tog}raphy}. It's a 10 min \textbf{pres}entation
        \textcolor{teal}{so} please bear with
        me until the end to ask any \textbf{ques}tion you might have during my speech. So let's just
        dive into it.
    \item \textbf{Slide 2}: Through this beamer I'd like to \textbf{an}swer the
        \textbf{fol}lowing \textbf{ques}tions.

        They are about to\textbf{day}'s topic \textcolor{teal}{which} is about the \textcolor{red}{group} of
        \textcolor{red}{\textbf{rat}ionals points} of
            an \textcolor{red}{el\textbf{lip}tic \textbf{cur}ves} defined over a \textcolor{red}{field} $K$. 

            In fact, \textcolor{red}{el\textbf{lip}tic \textbf{cur}ves} are to\textbf{day}'s
            trend in \textcolor{red}{cry\textbf{tog}raphy}. They are \textcolor{green}{broadly} use in
            \textcolor{red}{\textbf{sig}nature au\textbf{then}tification} and \textcolor{red}{key
            sharing \textbf{pro}tocol}. 

            \textcolor{brown}{Therefore}, I'd like to show you the underlying
            mechanism of its con\textbf{struc}tion \textcolor{teal}{in \textbf{or}der to} give an ex\textbf{am}ple of one
            \textcolor{green}{of the most} used
            \textcolor{red}{key
            sharing \textbf{pro}tocol} \textcolor{teal}{which} \textcolor{violet}{is known} as the \textcolor{red}{Diffie-Hellman
                \textbf{pro}tocol}. 
        \item \textbf{Slide 3}:
            In this \textbf{pres}entation
            \begin{itemize}
                \item I'll \textbf{an}swer the first \textbf{ques}tion in
                    section 2 and 3.
                \item \textcolor{brown}{Then} I'll de\textbf{scribe} the con\textbf{struc}tion of the \textcolor{red}{group} in section 4.
                \item \textcolor{brown}{Finally} in section 5, I'll ex\textbf{plain} why it
                    works, \textcolor{brown}{then} give an
                    ap\textbf{pli}cation \textcolor{teal}{and} con\textbf{clude} by giving the \textbf{up}sides and \textbf{down}sides of the
                    \textbf{the}ory.
                % \item \textcolor{brown}{finally} I'll give the cons and pros of the model in section 7.
            \end{itemize}
        \item \textbf{Slide 3}: So first thing first, the study of the  \textcolor{red}{group} of
            \textcolor{red}{\textbf{rat}ionals points} of an
            el\textbf{lip}tic \textbf{cur}ves, aimed towards its ap\textbf{pli}cation in cry\textbf{tog}raphy, \textcolor{violet}{was
                made} in
            parallel be\textbf{tween} N. Koblitz and Victor S. Miller and \textcolor{violet}{was
                published} in 1985.
        \item \textbf{Slide 4}: To lay the foun\textbf{da}tion of the \textcolor{red}{group} we'd like to
            build. We need a couple of tools.
            Such as:
            \begin{itemize}
                \item The \textcolor{red}{pro\textbf{jec}tive plane},
                    indeed as we'll see by \textbf{def}initions \textcolor{red}{el\textbf{lip}tic \textbf{cur}ves} are
                    \textcolor{red}{pro\textbf{jec}tive
                    geometry}'s objects.
                \item \textcolor{brown}{Then} we need to under\textbf{stand} \textcolor{red}{pro\textbf{jec}tive lines}
                    \textcolor{teal}{in \textbf{or}der to} under\textbf{stand} the \textcolor{red}{pro\textbf{jec}tive
                    plane} \textcolor{teal}{because}
                    they are what \textbf{gen}erate the \textcolor{red}{pro\textbf{jec}tive plane}.
                \item The next tools we need are \textcolor{red}{straight} and
                    \textcolor{red}{tangent} lines.
                    They are  the two \textcolor{red}{lines} \textcolor{teal}{that} we need to under\textbf{stand}
                    \textcolor{teal}{in \textbf{or}der to} treat
                    each case we would \textbf{stum}ble up\textbf{on} \textbf{stud}ying the \textcolor{red}{\textbf{a}belian \textbf{bi}nary
                        \textbf{op}eration} of the \textcolor{red}{group}.
                \item \textcolor{brown}{Finally} the last tool is the \textcolor{red}{\textbf{rat}ionals
                    points}.
                    They will be the \textcolor{red}{\textbf{el}ements} of our
                    \textcolor{red}{group} \textcolor{teal}{thanks to} them we can com\textbf{pute}
                    \textcolor{red}{ad\textbf{di}tions} and \textcolor{red}{doubles}
                    \textcolor{teal}{which} is the heart of our con\textbf{struc}tion.
            \end{itemize}
        \item \textbf{Slide 5}: The \textcolor{red}{pro\textbf{jec}tive plane} is a
            \textcolor{red}{quotient-set} defined over a
            \textcolor{red}{\textbf{vec}tor space} without its \textcolor{red}{origin},
            \textcolor{teal}{and} by an
            \textcolor{red}{e\textbf{quiv}alence re\textbf{la}tion} \textcolor{teal}{which} let
            two \textcolor{red}{\textbf{vec}tors} be the same if there are on the same
            \textcolor{red}{line}.

            More precisely the \textcolor{red}{pro\textbf{jec}tive plane} is the reunion be\textbf{tween} the
            \textcolor{red}{af\textbf{fine} plane}
            and the \textcolor{red}{in\textbf{fin}ity line}.

            The \textcolor{red}{af\textbf{fine} plane} \textcolor{violet}{is \textbf{gen}erated by} the \textcolor{red}{pro\textbf{jec}tive
                line}. We can see an ex\textbf{am}ple of a
            \textcolor{red}{pro\textbf{jec}tive line} on the figure 2. 

            The \textcolor{blue}{red line} is the \textcolor{red}{pro\textbf{jec}tive line} \textcolor{teal}{which} \textcolor{violet}{is
                \textbf{gen}erated by} the in\textbf{ter}section be\textbf{tween}
            each \textcolor{blue}{\textbf{vec}tor lines}
            of the \textcolor{red}{\textbf{vec}tor space} and in this case the \textcolor{blue}{line} $y=1$.
        % \item \textbf{Slide 6}: Over the \textcolor{red}{\textbf{vec}tor space} $\R^3$ we can see the
        %     \textcolor{red}{pro\textbf{jec}tive line} as the \textcolor{red}{line} that \textbf{gen}erates the
        %     \textcolor{red}{af\textbf{fine} plane}. As we can see in
        %     the \textbf{fol}lowing figure 3. 

        %     Here the \textcolor{blue}{red plane} is the \textcolor{red}{pro\textbf{jec}tive line}
        %     \textcolor{teal}{which}
        %     \textcolor{violet}{is
        %     \textbf{gen}erated by} the in\textbf{ter}section be\textbf{tween} \textbf{ev}ery \textcolor{blue}{\textbf{vec}tor lines} of the
        %     \textcolor{red}{\textbf{vec}tor space} and
        %     the \textcolor{blue}{plane} of equation $z=1$.
        \item \textbf{Slide 6}: So to re\textbf{sume} what I've said since the be\textbf{gin}ning, the
            \textcolor{red}{pro\textbf{jec}tive plane} is a \textcolor{blue}{sphere} of radius
            1 where we cut a
            \textcolor{red}{slice} of the \textbf{vec}tor space on top of it, in general at $z=1$.
            \textcolor{brown}{Therefore} the projection of the \textcolor{blue}{sphere} on this plane gives us the
            \textcolor{red}{af\textbf{fine} slice}
            \textcolor{teal}{which} is a \textcolor{blue}{circle}.
             \textcolor{brown}{Moreover} thanks to the
            \textcolor{red}{e\textbf{quiv}alence re\textbf{la}tion} and \textcolor{red}{\textbf{lim}its}, we have that the
            \textcolor{red}{in\textbf{fin}ity
            point} $\mathcal{O}$ is the in\textbf{ter}section of \textbf{ev}ery \textcolor{red}{\textbf{ver}tical line} of the
            \textcolor{red}{af\textbf{fine} slice} and the \textcolor{blue}{$y$-inter\textbf{cept}}
            \textcolor{teal}{which} is the perimeter of the \textcolor{blue}{circle}. \textcolor{brown}{Henceforth}
            the \textcolor{red}{in\textbf{fin}ity point} is the \textcolor{red}{\textbf{neu}tral
            \textbf{el}ement} of the \textcolor{red}{group}.
        \item \textbf{Slide 7}: Here is the \textbf{def}inition of \textcolor{red}{el\textbf{lip}tic \textbf{cur}ves}.

            The condition 2 guaranty us that our \textcolor{red}{\textbf{cur}ve} is
            \textcolor{red}{smooth} \textcolor{teal}{which} simply means that
            there is only one \textcolor{red}{tangent} per \textcolor{red}{point}.

            On the figure 5 we can see a re\textbf{pres}entation of an \textcolor{blue}{el\textbf{lip}tic \textbf{cur}ves}
            on the \textcolor{blue}{af\textbf{fine}
            slice} \textcolor{teal}{where} the \textcolor{red}{discriminant} is negative \textcolor{teal}{hence} there is only one
            \textcolor{blue}{root}.
        \item \textbf{Slide 8}: The \textcolor{red}{Weierstrass normal
                equation} give use \textcolor{red}{el\textbf{lip}tic
            \textbf{cur}ves} \textcolor{teal}{that} are \textcolor{red}{sym\textbf{met}ric} around the
            \textcolor{blue}{$x$-inter\textbf{cept}}.

            By \textbf{def}inition \textcolor{red}{\textbf{rat}ionals points} of an \textcolor{red}{el\textbf{lip}tic
                \textbf{cur}ves} are the \textcolor{red}{pro\textbf{jec}tive points}
            \textcolor{teal}{that} are \textcolor{red}{so\textbf{lu}tions} of the \textcolor{red}{Weierstrass normal
                equation}.

            Here is the \textcolor{red}{coordinates} of the \textcolor{red}{in\textbf{fin}ity point} $\mathcal{O}$.
        \item \textbf{Slide 9}: \textcolor{brown}{So} to build our \textcolor{red}{\textbf{a}belian \textbf{bi}nary
                \textbf{op}eration}, \textcolor{teal}{which} we'll call
            \textcolor{red}{ad\textbf{di}tion} \textcolor{brown}{henceforth}.

            We first need to
            look \textcolor{magenta}{what will happen} \textcolor{teal}{when} we take the \textcolor{red}{chords} be\textbf{tween}
            two
            \textcolor{red}{points} of the \textcolor{red}{\textbf{cur}ve} or
            the \textcolor{red}{tangent} of a \textcolor{red}{point} ?
            \textcolor{brown}{While} asking ourselves
            \textcolor{magenta}{are these lines} always giving us a third
            \textcolor{red}{point} on the \textcolor{red}{\textbf{cur}ve} ? The \textbf{an}swer is yes thanks to
            the \textcolor{red}{pro\textbf{jec}tive plane} and the
            \textcolor{red}{in\textbf{fin}ity point}.

            \textcolor{brown}{Therefore} this \textcolor{red}{non as\textbf{so}ciative \textbf{bi}nary \textbf{op}eration}
            give use the foun\textbf{da}tion
            to build the \textcolor{red}{ad\textbf{di}tion} \textcolor{teal}{that} we're looking for.
        \item \textbf{Slide 10}: Here is the \textcolor{red}{ad\textbf{di}tion} that we obtain
            thanks to a smart \textcolor{red}{sym\textbf{met}ry}. Indeed if we take the
            \textcolor{red}{opposite} of the \textcolor{red}{point} we've
            obtained
            through our \textcolor{red}{non as\textbf{so}ciative \textbf{bi}nary \textbf{op}eration}. We obtain the result
            of the \textcolor{red}{ad\textbf{di}tion} be\textbf{tween} two
            \textcolor{red}{points} of the \textcolor{red}{group}.
        \item \textbf{Slide 11}: \textcolor{red}{Modern cry\textbf{tog}raphy}'s foun\textbf{da}tion are
            build up\textbf{on} a tested \textcolor{red}{hy\textbf{poth}esis} and others as\textbf{sump}tions
            \textcolor{teal}{which} are still
            holding to\textbf{day}.
            
            In our case their is the \textbf{fol}lowing as\textbf{sump}tions \textcolor{teal}{that} are important:
            \begin{itemize}
                \item \textcolor{red}{One-way functions} exist and the
                    \textcolor{red}{ad\textbf{di}tion}, we've built, is one of them.
                \item The \textcolor{red}{discrete logarithm} problem is unsolvable in a
                    \textcolor{red}{polynomial time}.
                \item There is no other way to solve \textcolor{red}{Diffie-Hellman's
                        problem} without solving
                    the \textcolor{red}{discrete logarithm problem}.
            \end{itemize}
        \item \textbf{Slide 12}: Here is the \textcolor{red}{Diffie-Hellman
                \textbf{pro}tocol} \textcolor{teal}{which} is the
            basis of to\textbf{day}'s online transaction.

            Two person Alice and Bob would like to share a \textcolor{red}{\textbf{se}cret key
                publicly}. \textcolor{brown}{Hence} they
            proceed as followed:
            \begin{itemize}
                \item They choose a \textcolor{red}{finite field} $K$, an
                    \textcolor{red}{el\textbf{lip}tic \textbf{cur}ves} $E$, a base
                    \textcolor{red}{point} $P$ and publish this \textcolor{red}{triplet}.
                \item \textcolor{brown}{Then} they both chose a \textcolor{red}{\textbf{se}cret \textbf{in}teger} and
                    \textcolor{red}{com\textbf{pute}} this \textcolor{red}{\textbf{in}teger} times
                    the \textcolor{red}{base point}. \textcolor{brown}{Then} they respectively send their
                    result to the other.
                \item \textcolor{brown}{Lastly} they com\textbf{pute} again their \textcolor{red}{\textbf{se}cret
                    \textbf{in}teger} times \textcolor{teal}{what} \textcolor{violet}{they've
                    received}. Which give them their shared \textcolor{red}{\textbf{se}cret key}.
            \end{itemize}

            \textcolor{brown}{Hence} the \textcolor{red}{Diffie-Hellman \textbf{pro}tocol} is a se\textbf{cure} way to share a
            \textcolor{red}{key} publicly.

        \item \textbf{Slide 13}: The \textcolor{red}{group} of \textcolor{red}{\textbf{rat}ionals
                point} of an \textcolor{red}{el\textbf{lip}tic
            \textbf{cur}ve} have \textcolor{olive}{many} benefits
            \textcolor{green}{\textbf{com}pared to} the \textcolor{red}{\textbf{mul}tiplicative
                group} of the
            \textcolor{red}{invertible} of a \textcolor{red}{finite field} $K$. Among them there is:
            \begin{itemize}
                \item The \textcolor{red}{structure} that is more \textcolor{red}{abstract}.
                \item The \textcolor{red}{keys}' length is way shorter for
                    \textcolor{olive}{e\textbf{quiv}alent}
                    \textcolor{red}{se\textbf{cu}rity} \textcolor{green}{\textbf{com}pared to} \textcolor{red}{RSA}.
                \item It can be use on \textcolor{red}{low resources systems}.
                \item It can be implemented in an \textcolor{red}{hybrid
                        cry\textbf{pto}system} \textcolor{teal}{which} is the
                    combination be\textbf{tween} a \textcolor{red}{sym\textbf{met}ric
                        cry\textbf{pto}system} to \textcolor{red}{encrypt data}
                        \textcolor{teal}{and}
                    an \textcolor{red}{asym\textbf{met}ric cry\textbf{pto}system} to share
                    \textcolor{red}{\textbf{se}cret key}.
            \end{itemize}

            \textcolor{brown}{However}, as anything else there is always \textbf{down}sides. For ex\textbf{am}ple:
            \begin{itemize}
                \item There \textcolor{violet}{is} already a \textcolor{olive}{lot} of \textbf{cur}ves
                    \textcolor{violet}{patented} by companies.
                \item There is \textbf{al}ways hazards \textcolor{teal}{and} in our case \textcolor{teal}{if} the \textcolor{red}{triplet}
                    we're
                    using isn't 
                    \textbf{prop}erly, ef\textbf{fi}ciently and \textbf{ran}domly selected there \textcolor{teal}{will
                        always be} the
                    risks of premeditated 
                    use of a \textcolor{red}{back door}. 
            \end{itemize}
\end{itemize}

Which let's me con\textbf{clude} with the \textbf{fol}lowing statement ...

Thank you \textbf{ev}eryone for your at\textbf{ten}tion. My \textbf{pres}entation \textcolor{violet}{is done} and now is \textbf{ques}tion time's.

\setlength{\arrayrulewidth}{0.5mm}
% \setlength{\tabcolsep}{2pt}
\renewcommand{\arraystretch}{3}
\begin{table}[h]
    \centering
    \caption{Assessment grid}
    \label{tab:AssessmenttGrid}
    \begin{tabular}{|l|m{10cm}|}
        \hline
        \textcolor{blue}{voc-graph} & \\
        \hline
        \textcolor{red}{voc-your-field-of-research} & \\
        \hline
        \textcolor{green}{Gram-com/sup} & \\
        \hline
        \textcolor{magenta}{Gram-questions} & \\
        \hline
        \textcolor{violet}{Gram-passive} & \\
        \hline
        \textcolor{olive}{Gram-quantity} & \\
        \hline
        \textcolor{brown}{Syntax-link-words} & \\
        \hline
        \textcolor{teal}{Syntax-condition and complex-sentences} & \\
        \hline
        Word \textbf{stress}  & \\
        \hline
    \end{tabular}
\end{table}
% (bold print on the stressed syllable of polysyllabic words)
