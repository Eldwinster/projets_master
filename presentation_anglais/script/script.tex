\begin{center}
    \textbf{el\textbf{lip}tic curves cry\textbf{tog}raphy}
\end{center}
\begin{itemize}
    \item \textbf{Slide 1}: Hel\textbf{lo} \textbf{eve}ryone, let me \textbf{in}tro\textbf{duce}
        my\textbf{self} quicky. I'm
        Yann-Arby, I'm \textbf{stud}ying a \textbf{mas}ter de\textbf{gree} in
        \textbf{math}e\textbf{mat}ics at the Uni\textbf{ver}sity of Picardie Jules
        Verne. To\textbf{day}, I'll speak a\textbf{bout} \hlred{el\textbf{lip}tic
        curves} and their
        \textbf{ap}pli\textbf{ca}tion in cryp\textbf{tog}raphy. It's a 10 min
        \textbf{pres}en\textbf{ta}tion
        \hlteal{so} please bear with
        me until the end to ask any \textbf{ques}tion you might have during my speech.
        So let's just
        dive into it.
    \item \textbf{Slide 2}: Through this beamer I'd like to \textbf{an}swer the
        \textbf{fol}lowing \textbf{ques}tions.

        They are a\textbf{bout} to\textbf{day}'s \textbf{top}ic \hlteal{which} is a\textbf{bout} the \hlred{group} of
        \hlred{\textbf{rat}ional points} of
            an \hlred{el\textbf{lip}tic curve} de\textbf{fin}ed \textbf{o}ver a \hlred{field} $K$. 

            In fact, \hlred{el\textbf{lip}tic curves} are to\textbf{day}'s
            trend in cry\textbf{tog}raphy. They are \hlolive{broadly} use in
            \hlred{\textbf{sig}nature au\textbf{then}tification} and \hlred{key
            sharing \textbf{pro}tocol}. 

            \hlbrown{\textbf{Therefore}}, I'd like to show you the \textbf{un}der\textbf{ly}ing
            \textbf{mech}a\textbf{nis}m of its con\textbf{struc}tion \hlteal{in \textbf{or}der to} give an ex\textbf{am}ple of \hlgreen{one
            of the most} used
            \hlred{key
            sharing \textbf{pro}tocol} \hlteal{which} \hlpink{is known} as the \hlred{Diffie-Hellman
                \textbf{pro}tocol}. 
        \item \textbf{Slide 3}:
            In this \textbf{pres}en\textbf{ta}tion
            \begin{itemize}
                \item I'll \textbf{an}swer the first \textbf{ques}tion in
                    \textbf{sec}tion 2 and 3.
                \item \hlbrown{Then} I'll de\textbf{scribe} the con\textbf{struc}tion of the \hlred{group} in \textbf{sec}tion 4.
                \item \hlbrown{\textbf{Fi}naly} in \textbf{sec}tion 5, I'll ex\textbf{plain} why it
                    works, \hlteal{then} I'll give an
                    ap\textbf{pli}cation and con\textbf{clude} by giving a \hlolive{few} \textbf{up}sides and \textbf{down}sides of the
                    \textbf{the}ory.
                % \item \hlbrown{finally} I'll give the cons and pros of the model in \textbf{sec}tion 7.
            \end{itemize}
        \item \textbf{Slide 3}: \hlbrown{So first thing first}, the study of the  \hlred{group} of
            \hlred{\textbf{rat}ional points} of an
            \hlred{el\textbf{lip}tic curve}, aimed to\textbf{wards} its ap\textbf{pli}cation in cry\textbf{tog}raphy, \hlpink{was
                made} in
            parallel be\textbf{tween} N. Koblitz and Victor S. Miller and \hlpink{was
                \textbf{pub}lished} in 1985.
        \item \textbf{Slide 4}: To lay the foun\textbf{da}tion of the \hlred{group} we'd like to
            build. We need a \hlolive{couple} of tools.
            Such as:
            \begin{itemize}
                \item The \hlred{pro\textbf{jec}tive plane},
                    in\textbf{deed} as we'll see by \textbf{def}initions \hlred{el\textbf{lip}tic curves} are
                    \hlred{pro\textbf{jec}tive
                    geometry}'s objects.
                \item \hlbrown{Then} we need to under\textbf{stand} \hlred{pro\textbf{jec}tive lines}
                    \hlteal{in \textbf{or}der to} under\textbf{stand} the \hlred{pro\textbf{jec}tive
                    plane} \hlteal{be\textbf{cause}}
                    they are what \textbf{gen}erate the \hlred{pro\textbf{jec}tive plane}.
                \item The next tools we need are \hlred{straight} and
                    \hlred{\textbf{tan}gent} lines.
                    They are  the two lines \hlteal{that} we need to under\textbf{stand}
                    \hlteal{in \textbf{or}der to} treat
                    each case we would \textbf{stum}ble up\textbf{on} \textbf{stud}ying the \hlred{\textbf{a}belian \textbf{bi}nary
                        \textbf{op}eration} of the \hlred{group}.
                \item \hlbrown{\textbf{Fi}naly} the last tool is the \hlred{\textbf{rat}ional
                    points}.
                    They will be the \hlred{\textbf{el}ements} of our
                    \hlred{group} \hlteal{thanks to} them we can com\textbf{pute}
                    \hlred{ad\textbf{di}tions} and \hlred{\textbf{dou}bles}
                    \hlteal{which} is the heart of our con\textbf{struc}tion.
            \end{itemize}
        \item \textbf{Slide 5}: The \hlred{pro\textbf{jec}tive plane} is a
            \hlred{\textbf{quo}tient-set} de\textbf{fin}ed \textbf{o}ver a
            \hlred{\textbf{vec}tor space} with\textbf{out} its \hlred{\textbf{or}igin},
            and by an
            \hlred{e\textbf{quiv}alence re\textbf{la}tion} \hlteal{which} let
            two \hlred{\textbf{vec}tors} be the same if there are on the same
            line.

            More pre\textbf{cise}ly the \hlred{pro\textbf{jec}tive plane} is the
            re\textbf{un}ion be\textbf{tween} the
            \hlred{af\textbf{fine} plane}
            and the \hlred{in\textbf{fin}ity line}.

            The \hlred{af\textbf{fine} plane} \hlpink{is \textbf{gen}erated by} the \hlred{pro\textbf{jec}tive
                line}. We can see an ex\textbf{am}ple of a
            \hlred{pro\textbf{jec}tive line} on the \textbf{fig}ure 2. 

            The red line is the \hlred{pro\textbf{jec}tive line} \hlteal{which} \hlpink{is
                \textbf{gen}erated by} the \textbf{in}ter\textbf{sec}tion be\textbf{tween}
            each \hlred{\textbf{vec}tor lines}
            of the \hlred{\textbf{vec}tor space} and in this case the line $y=1$.
        % \item \textbf{Slide 6}: Over the \hlred{\textbf{vec}tor space} $\R^3$ we can see the
        %     \hlred{pro\textbf{jec}tive line} as the \hlred{line} that \textbf{gen}erates the
        %     \hlred{af\textbf{fine} plane}. As we can see in
        %     the \textbf{fol}lowing \textbf{fig}ure 3. 

        %     Here the \hlblue{red plane} is the \hlred{pro\textbf{jec}tive line}
        %     \hlteal{which}
        %     \hlpink{is
        %     \textbf{gen}erated by} the \textbf{in}ter\textbf{sec}tion be\textbf{tween} \textbf{ev}ery \hlblue{\textbf{vec}tor lines} of the
        %     \hlred{\textbf{vec}tor space} and
        %     the \hlblue{plane} of equation $z=1$.
        \item \textbf{Slide 6}: So to re\textbf{sume} \hlteal{what} I've said since the be\textbf{gin}ning, the
            \hlred{pro\textbf{jec}tive plane} is a \hlblue{sphere} of \textbf{ra}dius
            one \hlteal{where} we cut a
            \hlred{slice} of the \textbf{vec}tor space on top of it, in general at $z=1$.
            \hlbrown{\textbf{Therefore}} the projection of the \hlblue{sphere} on this plane gives us the
            \hlred{af\textbf{fine} slice}
            \hlteal{which} is a \hlblue{circle}.
             \hlbrown{More\textbf{o}ver} thanks to the
            \hlred{e\textbf{quiv}alence re\textbf{la}tion} and \hlred{\textbf{lim}its}, we have that the
            \hlred{in\textbf{fin}ity
            point} is the \textbf{in}ter\textbf{sec}tion of \textbf{ev}ery \hlred{\textbf{ver}tical line} of the
            \hlred{af\textbf{fine} slice} and the \hlblue{$y$-inter\textbf{cept}}
            \hlteal{which} is the perimeter of the \hlblue{circle}. \hlbrown{Hence\textbf{forth}}
            the \hlred{in\textbf{fin}ity point} is the \hlred{\textbf{neu}tral
            \textbf{el}ement} of the \hlred{group}.
        \item \textbf{Slide 7}: Here is the \textbf{def}inition of \hlred{el\textbf{lip}tic curves}.

            The con\textbf{di}tion 2 \textbf{guar}an\textbf{ty} us that our \hlred{curve} is
            \hlred{smooth} \hlteal{which} simply means that
            there is only one \hlred{\textbf{tan}gent} per \hlred{point}.

            On the \textbf{fig}ure 5 we can see a \textbf{rep}resen\textbf{ta}tion of an \hlred{el\textbf{lip}tic curve}
            on the \hlred{af\textbf{fine}
            slice} \hlteal{where} the \hlred{dis\textbf{crim}inant} is
            \textbf{neg}ative \hlteal{thus} there is only one
            \hlblue{root}.
        \item \textbf{Slide 8}: The \hlred{\textbf{Wei}erstrass
                \textbf{nor}mal
                e\textbf{qua}tion} give use \hlred{el\textbf{lip}tic
            curve} \hlteal{that} are \hlred{sym\textbf{met}ric} around the
            \hlblue{$x$-inter\textbf{cept}}.

            \hlbrown{Be\textbf{sides}} by \textbf{def}inition \hlred{\textbf{rat}ional points} of an \hlred{el\textbf{lip}tic
                curve} are the \hlred{pro\textbf{jec}tive points}
            \hlteal{that} are \hlred{so\textbf{lu}tions} of the \hlred{\textbf{Wei}erstrass
                \textbf{nor}mal
                e\textbf{qua}tion}.

            Here are the \hlred{coordinates} of the \hlred{in\textbf{fin}ity point}.
        \item \textbf{Slide 9}: So to build our \hlred{\textbf{a}belian \textbf{bi}nary
                \textbf{op}eration}, \hlteal{which} we'll call
            \hlred{ad\textbf{di}tion} hence\textbf{forth}.

            We first need to
            look \hlmagenta{what will \textbf{hap}pen} \hlteal{when} we take the \hlred{chord} be\textbf{tween}
            two
            \hlred{points} of the \hlred{curve} or
            the \hlred{\textbf{tan}gent} of a \hlred{point} ? \hlmagenta{Are these lines
                \textbf{al}ways
                giving} us a third
            point on the curve ?
            % \hlbrown{While} asking ourselves
            % \hlmagenta{are these lines} always giving us a third
            % \hlred{point} on the \hlred{curve} ?
             The \textbf{an}swer is yes thanks to
            the \hlred{pro\textbf{jec}tive plane} and the
            \hlred{in\textbf{fin}ity point}.

            \hlbrown{\textbf{Therefore}} this \hlred{non as\textbf{so}ciative \textbf{bi}nary \textbf{op}eration}
            give use the foun\textbf{da}tion
            to build the \hlred{ad\textbf{di}tion} \hlteal{that} we're looking for.
        \item \textbf{Slide 10}: Here is the \hlred{ad\textbf{di}tion} that we ob\textbf{tain}
            thanks to a smart \hlred{sym\textbf{met}ry}. In\textbf{deed} if we take the
            \hlred{\textbf{op}posite} of the \hlred{point} we've
            ob\textbf{tain}ed
            through our \hlred{non as\textbf{so}ciative \textbf{bi}nary \textbf{op}eration}. We ob\textbf{tain} the result
            of the \hlred{ad\textbf{di}tion} be\textbf{tween} two
            \hlred{points} of the \hlred{group}.
        \item \textbf{Slide 11}: \hlred{\textbf{Mod}ern cry\textbf{tog}raphy}'s foun\textbf{da}tion
            \hlpink{are
            built} up\textbf{on} a tested \hlred{hy\textbf{poth}esis} and \textbf{oth}er as\textbf{sump}tions
            \hlteal{which} are still
            holding to\textbf{day}.
            
            In our case there is the \textbf{fol}lowing as\textbf{sump}tions \hlteal{that} are
            im\textbf{por}tant:
            \begin{itemize}
                \item \hlred{One-way \textbf{func}tions} exist and the
                    \hlred{ad\textbf{di}tion}, we've built, is one of them.
                \item The \hlred{dis\textbf{crete} \textbf{lo}ga\textbf{rith}m}
                    \textbf{prob}lem is un\textbf{sol}vable in a
                    \hlred{\textbf{pol}y\textbf{no}mial time}.
                \item There is no \textbf{oth}er way to solve \hlred{Diffie-Hellman's
                        \textbf{prob}lem} \hlteal{with\textbf{out}} solving
                    the \hlred{dis\textbf{crete} \textbf{lo}ga\textbf{rith}m \textbf{prob}lem}.
            \end{itemize}
        \item \textbf{Slide 12}: Here is the \hlred{Diffie-Hellman
                \textbf{pro}tocol} \hlteal{which} is the
            basis of to\textbf{day}'s \textbf{online} tran\textbf{sac}tion.

            Two \textbf{per}son Alice and Bob would like to share a \hlred{\textbf{se}cret key
                \textbf{pub}licly} \hlteal{hence} they
            \textbf{pro}ceed as followed:
            \begin{itemize}
                \item They choose a \hlred{\textbf{finite} field} $K$, an
                    \hlred{el\textbf{lip}tic curve} $E$, a base
                    \hlred{point} $P$ and \textbf{pub}lish this \hlred{\textbf{trip}let}.
                \item \hlbrown{Then} they both choose a \hlred{\textbf{se}cret \textbf{in}teger} and
                    \hlred{com\textbf{pute}} this \hlred{\textbf{in}teger} times
                    the \hlred{base point}. \hlbrown{Then} they re\textbf{spec}tively send their
                    re\textbf{sult} to the \textbf{oth}er.
                \item \hlbrown{Lastly} they com\textbf{pute} again their \hlred{\textbf{se}cret
                    \textbf{in}teger} times \hlteal{what} \hlpink{they've
                    re\textbf{ceived}} \hlteal{which} give them their shared \hlred{\textbf{se}cret key}.
            \end{itemize}

            \hlbrown{Hence} the \hlred{Diffie-Hellman \textbf{pro}tocol} is a se\textbf{cure} way to share a
            \hlred{key} \textbf{pub}licly.

        \item \textbf{Slide 13}: The \hlred{group} of \hlred{\textbf{rat}ional
                points} of an \hlred{el\textbf{lip}tic
            curve} have \hlolive{\textbf{man}y} benefits
            \hlgreen{\textbf{com}pared to} the \hlred{\textbf{mul}tiplicative
                group} of the
            \hlred{in\textbf{vert}ible} of a \hlred{\textbf{finite} field} $K$.
            A\textbf{mong} them there is:
            \begin{itemize}
                \item The \hlred{\textbf{struc}ture} that is \hlgreen{way more}
                    \hlred{ab\textbf{stract}}.
                \item The \hlred{keys}' length is way \hlgreen{shorter} for
                    \hlolive{e\textbf{quiv}alent}
                    \hlred{se\textbf{cu}rity} \hlgreen{\textbf{com}pared to} \hlred{RSA}.
                \item It can be use on \hlred{low re\textbf{source} \textbf{sys}tems}.
                \item It can be \textbf{im}plemented in an \hlred{\textbf{hy}brid
                        cry\textbf{pto}system} \hlteal{which} is the
                    \textbf{com}bi\textbf{na}tion be\textbf{tween} a \hlred{sym\textbf{met}ric
                        cry\textbf{pto}system} to \hlred{en\textbf{crypt} data}
                        and
                    an \hlred{asym\textbf{met}ric cry\textbf{pto}system} to share
                    \hlred{\textbf{se}cret key}.
            \end{itemize}

            \hlbrown{How\textbf{ev}er}, as \textbf{an}y\textbf{thing} else there is \textbf{al}ways \textbf{down}sides. For ex\textbf{am}ple:
            \begin{itemize}
                \item There \hlpink{is} al\textbf{read}y a \hlolive{lot} of curves
                    \hlpink{\textbf{pat}ented} by \textbf{com}panies.
                \item There is \textbf{al}ways \textbf{haz}ard \hlteal{which} in our case
                    comes from the
                    \textbf{trip}let \hlteal{which} \hlpink{is chosen} by Alice and Bob. \hlteal{If} the \hlred{\textbf{trip}let}
                     \hlpink{isn't} 
                    \textbf{prop}erly, ef\textbf{fi}ciently and \textbf{ran}domly
                    \hlpink{se\textbf{lect}ed},
                    \hlteal{there is} the
                    risk of a pre\textbf{med}i\textbf{tated} 
                    use of \hlred{back doors}. 
            \end{itemize}
\end{itemize}

\hlbrown{Which} let's me con\textbf{clude} with the \textbf{fol}lowing \textbf{stat}ement
of Serge Lang in his book \underline{\textsl{Elliptic curve: Diophantine analysis}},
\hlteal{which} \hlpink{was published} in 1978:

\begin{quote}
    "It is possible to write endlessly on elliptic curves. (This is not a threat)"
\end{quote}

Thank you \textbf{ev}eryone for your at\textbf{ten}tion. My \textbf{pres}en\textbf{ta}tion
\hlpink{is done} and now it's \textbf{ques}tion time's.

\setlength{\arrayrulewidth}{0.5mm}
% \setlength{\tabcolsep}{2pt}
\renewcommand{\arraystretch}{3}
\begin{table}[h]
    \centering
    \caption{Assessment grid}
    \label{tab:AssessmenttGrid}
    \begin{tabular}{|l|m{10cm}|}
        \hline
        \hlblue{voc-graph} & \\
        \hline
        \hlred{voc-your-field-of-research} & \\
        \hline
        \hlgreen{Gram-com/sup} & \\
        \hline
        \hlmagenta{Gram-questions} & \\
        \hline
        \hlpink{Gram-passive} & \\
        \hline
        \hlolive{Gram-quantity} & \\
        \hline
        \hlbrown{Syntax-link-words} & \\
        \hline
        \hlteal{Syntax-condition and complex-sentences} & \\
        \hline
        Word \textbf{stress}  & \\
        \hline
    \end{tabular}
\end{table}
% (bold print on the stressed syllable of polysyllabic words)
