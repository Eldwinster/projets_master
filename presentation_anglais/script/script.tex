\begin{center}
    \textbf{Elliptic curves cryptography}
\end{center}
\begin{itemize}
    \item \textbf{Slide 1}: Hello everyone, let me introduce myself quickly. I'm
        Yann-Arby, I'm studying a master degree in mathematics at the University of Picardie Jules
        Verne. Today, I'll speak about elliptic curves and their
        application in cryptography. It's a 10 min presentation so please bear with
        me until the end to ask any question you might have during my speech. So let's just
        dive into it.
    \item \textbf{Slide 2}: Through this beamer my motivation is to answer the
        following questions.

        They are about today's topic which is about the group of rationals points of
            an elliptic curve defined over a field $K$. 

            In fact, Elliptic curves are today's
            trend in cryptography. It's broadly use in signature authentication and key
            sharing protocol. 

            Therefore, I'd like to show you the underlying
            mechanism of its construction in order to give an example of one of the most use key
            sharing protocol which is know as the Diffie-Hellman protocol. 
        \item \textbf{Slide 3}:
            In this presentation
            \begin{itemize}
                \item I'll answer the first question in
                    section 2 and 3.
                \item Then I'll describe the construction of the group in section 4.
                \item After that I'll explain why it works and give an application in section 5 and 6.
                \item Finally I'll give the cons and pros of the model in section 7.
            \end{itemize}
        \item \textbf{Slide 3}: The study of the group of rationals points of an
            elliptic curve, aimed towards its application in cryptography, was made in
            parallel between N. Koblitz and Victor S. Miller in 1985.
        \item \textbf{Slide 4}: To lay the foundation of the group we'd like to
            build. We need a couple of tools.
            Such as:
            \begin{itemize}
                \item The projective plane

                    Indeed as we'll see by definitions elliptic curves are projective
                    geometry's objects.
                \item The projective lines
                    
                    To understand projective plane we need to understand projective lines as
                    they are what generates the projective plane.
                \item Straight and tangent lines

                    That's the two lines that we will need to understand in order to treat
                    each case we would stumble upon studying the binary operation of the group
                    $\left( E,+ \right) $.
                \item The rationals points

                    They will be the elements of our group. Thanks to them we can compute
                    additions and doubles. Which is the heart of our construction.
            \end{itemize}
        \item \textbf{Slide 5}: The projective plane is a quotient-set define over a
            vector space without its origin, and define by an equivalence relation which let
            two vectors be the same if there are on the same line.

            More precisely the projective plane is the reunion between the affine plane
            and the infinity line.

            The affine plane is generated by the projective line. We can see an example of a
            projective line on the figure 2. 

            The red line is the projective line which is generated by the intersection between
            each vectors lines
            of the vector space and in this case the line $y=1$.
        \item \textbf{Slide 6}: Over the vector space $\R^3$ we can see the
            projective line as the line that generates the affine plane. As we can see in
            the following figure 3. 

            Here the red plane is the projective line which is
            generated by the intersection between every vectors of the vector space and
            the plane of equation $z=1$.
        \item \textbf{Slide 7}: So to resume what I've said since the beginning, the
            projective plane is a sphere where we cut a slice, in general at $z=1$ and
            the plane we get is our affine slice which is a circle. Therefore thanks to the
            equivalence relation and limits, we have that the infinity
            point $\mathcal{O}$ is the intersection of every vertical line of the
            affine slice and the $y$-intercept. Henceforth $\mathcal{O}$ is the neutral
            element of the group.
        \item \textbf{Slide 8}: Here is the definition of elliptic curves.

            The condition 2 guaranty us that our curve is smooth which simply means that
            there is only one tangent per point.

            On the figure 5 we can see a representation of an elliptic curve on the affine
            slice where the discriminant is negative hence there is only one roots.
        \item \textbf{Slide 9}: The Weierstrass normal equation give use elliptic
            curves that are symmetric around the $x$-intercept.

            By definition rationals points of an elliptic curves are the projective points
            that are solutions of the Weierstrass normal equation.

            Here is the coordinates of the infinity point $\mathcal{O}$.
        \item \textbf{Slide 10}: To build our abelian binary operation, which we'll call
            addition henceforth.

            First thing first we need to
            look what will happen when you take the chords between to points of the curve or
            the tangent of a point ? While asking ourselves are these lines always give us a third
            point on the curve. The answer is yes thanks to the projective plane and the
            infinity point.

            Therefore thanks to this non associative binary operation we have the foundation
            to build the addition that we're looking for.
        \item \textbf{Slide 11}: Here is the addition that we obtain
            thanks to a smart symmetry. Indeed if we take the opposite of the point we've
            obtained
            through our non associative binary operation. We obtain the result of the addition between two
            points of the group.
        \item \textbf{Slide 12}: Modern cryptography's foundation are
            build upon a tested hypothesis and others assumptions which are still
            holding today.
            
            In our case their is the following assumptions that are important:
            \begin{itemize}
                \item One-way functions exist and the addition, we've built, is one of them.
                \item The discrete logarithm problem is unsolvable in a polynomial time.
                \item There is no other way to solve Diffie-Hellman's problem without solving
                    the discrete logarithm problem.
            \end{itemize}
        \item \textbf{Slide 13}: Here is the Diffie-Hellman protocol which is the
            basis of today's online transaction.

            Two person Alice and Bob would like to share a secret key publicly. Hence they
            proceed as followed:
            \begin{itemize}
                \item They choose a finite field $K$, an elliptic curve $E$ and a base
                    point $P$ and publish this triplet.
                \item Then they both chose a secret integer and compute this integer times
                    the base point. Then they respectively send their result to the other.
                \item Lastly they compute again their secret integer times what they've
                    received. Which give them their secret key.
            \end{itemize}

            Hence the Diffie-Hellman protocol is a secure way to share a key publicly.

        \item \textbf{Slide 14}: The group of rationals point of an elliptic
            curve have many benefits compared to the multiplicative group of the
            invertible of a finite field $K$. Among them there is:
            \begin{itemize}
                \item The structure is more abstract.
                \item The keys' length is way shorter for equivalent security.
                \item It can be use on low resources systems.
                \item It can be implemented in an hybrid cryptosystem which is the
                    combination between a symmetric cryptosystem to encrypt data and
                    an asymmetric cryptosystem to share secret key.
            \end{itemize}

            However, as anything else there is restriction. For example:
            \begin{itemize}
                \item There is already a lot of patent rights own by companies.
                \item There is always hazard and in our case if the triplet we use is not
                    properly, efficiently and randomly selected there is the risk or
                    back door as the alleged claim of NSA using such a method.
            \end{itemize}
\end{itemize}

Which let's me conclude with the following statement ...

Thank you everyone for your attention. My presentation is done and now is question time.

\setlength{\arrayrulewidth}{0.5mm}
% \setlength{\tabcolsep}{2pt}
\renewcommand{\arraystretch}{3}
\begin{table}[h]
    \centering
    \caption{Assessment grid}
    \label{tab:AssessmenttGrid}
    \begin{tabular}{|l|m{10cm}|}
        \hline
        \textcolor{blue}{voc-graph} & \\
        \hline
        \textcolor{red}{voc-your-field-of-research} & \\
        \hline
        \textcolor{green}{Gram-com/sup} & \\
        \hline
        \textcolor{magenta}{Gram-questions} & \\
        \hline
        \textcolor{violet}{Gram-passive} & \\
        \hline
        \textcolor{olive}{Gram-quantity} & \\
        \hline
        \textcolor{brown}{Syntax-link-words} & \\
        \hline
        \textcolor{teal}{Syntax-condition and complex-sentences} & \\
        \hline
        Word \textbf{stress}  & \\
        \hline
    \end{tabular}
\end{table}
% (bold print on the stressed syllable of polysyllabic words)
