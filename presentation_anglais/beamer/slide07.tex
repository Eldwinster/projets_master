\section{Application}

\subsection{The discrete logarithm problem} 

\begin{frame}[t]
    \frametitle{Discrete logarithm}
    \begin{alertblock}{The problem of discrete logarithm}
        Let $E$ be an elliptic curve define over a field $K$.

        Let $Q \in E(K)$ be a rational point.

        Given a rational point $P \in E(K)$, the discrete logarithm problem is to find $n \in
        \N $, if its exists, such as $P = nQ$.
    \end{alertblock}
    % Modern cryptography's foundation are build upon a tested hypothesis and others assumptions
    % which still hold today.

    % In our case their is the following assumptions that are important:
    Why it works:
    \begin{itemize}
        \item One-way functions exist.
        % \item The discrete logarithm problem is unsolvable in a raisonable time.
        \item Unsolvability of the logarithm problem.
        % \item There is no other way to solve Diffie-Hellman's problem without solving
            % the discrete logarithm problem.
        \item No other way to solve the Diffie-Hellman's problem.
    \end{itemize}
\end{frame}

\subsection{Public key-sharing protocol}

\begin{frame}[t]
    \frametitle{Public key-sharing protocol}
    \begin{exampleblock}{Diffie-Hellman protocol}
        Alice and Bob would like to share a secret key (i.e. know only by them) over a non-secure channel.

        To do this they proceed the following way:
        \begin{description}
            \item[1)] 
                % They choose a finite field $K$ and an elliptic curve $E$ define over
                % $K$.
                % They choose a point $P \in E$. 
                They choose and publish the triplet $\left( K,E,P \right) $.

            \item[2)] Alice choose $a>0$ and compute $P_{a} =
                aP$, which she sends to Bob.

            % \item[3)] Bob choose a non zero secret integer $b$ and compute the point $P_{b} =
            %     bP$, which she sends to Bob.
            \item[3)] Bob $b>0$ and compute $P_{b} =
                bP$, which she sends to Bob.
            % \item[4)] Alice (resp. Bob) compute $aP_{b}$ (resp. $bP_{a}$) where both
                % computation's results is $P_{ab}$ which is the secret key only know by Alice
                % and Bob.
                \item[4)] Alice and Bob compute $aP_{b}$ and $bP_{a}$ which give $P_{ab}$.
        \end{description}
    \end{exampleblock}
\end{frame}

\subsection{Elliptic curves' pros and cons}

\begin{frame}[t]
    \frametitle{Pros and cons}
    % The group of rationals points of an elliptic curve have many benefits compared to the
    % multiplicative group of the inversible of a finite field $K$. Among them there is:
    Pros
    \begin{itemize}
        % \item The structure is more abstract
        % \item The key lengh is way shorter for equivalent security
        % \item Could be use on systems with low ressources
        % \item Could be use in an hybrid cryptosystem (i.e. symetric cryptosystem to encrypt
        %     data and asymetric cryptosystem to share secret key)
        \item Abstract structure.
        \item Shorter secret key lengh.
        \item Low ressources usage.
        \item Hybrid cryptosystem compatibility.
    \end{itemize}

    % However, as anything else there is restriction. For example: 
    Cons
    \begin{itemize}
        % \item More demanding resource wise to transfer data compared to RSA
        % \item There is already a lot of patent rights own by companies
        \item Many are patented
        \item Build-in trap doors? \cite{noauthor_how_2014} \cite{goodin_we_2013}
    \end{itemize}

    \begin{center}
    The following is a statement of Serge Lang in his book

    \textit{\underline{Elliptic curve:
    Diophantine analysis}}, 1978 \cite{lang_auth_elliptic_1978}: 
    \begin{quote}
       "It is possible to write endlessly on elliptic curves. (This is not a threat)"
    \end{quote}
    \end{center}
\end{frame}
